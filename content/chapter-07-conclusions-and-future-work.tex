\chapter{Conclusions and Future work}
\label{ch:conclusionsAndFutureWork}


% \sweExpl{Slutsats och framtida arbete}
% \generalExpl{Add text to introduce the subsections of this chapter.}


This chapter will provide conclusions from the research carried out in this thesis. It will reflect on the goals laid out in section~\ref{sec:goals} and what insights have been gained. It will describe the limitations to the result and discuss some future work.


\section{Conclusions}
\label{sec:conclusions}


One of the goals for this thesis was to understand the technological efficacy, i.e. speed and accuracy of various different tools and techniques commonly used when developing AI assistants. In addition to that, this thesis tried to understand what users preferred, various tools and the feasibility of operating \gls{LLM}on-premises. Lastly, a goal was to understand how AI assistants impacted education. All of this was encapsulated by the research questions, \textit{Which language model and which retrieval techniques do students prefer using?} and \textit{Is it possible to deploy an AI-assistant using a completely open source toolchain?}. My hypothesis was that the closed source alternatives would be better, however it would be feasible to build a completely open source and self-hosted AI assistant too.


Generally, I think the hypothesis turned out to be correct. The results laid out in \autoref{ch:resultsAndAnalysis}, and section~\ref{sec:survey_questions_injected_into_the_chat} specifically, showed that students preferred the model provided by OpenAI to the Mistral model. Even though that comparison isn’t very fair model-to-model, the fact that self-hosting models that are as large as those available by for profit vendors, means that building an assistant on proprietary models will yield a better assistant. However, the results show generally favourable opinions from the users who had to use the open source model.


Due to the size of the experiment, there weren't enough participants to test all the tools and techniques initially intended. The software constructed for the experiments was designed for testing more tools and models, such as some open source embedding functions. However, given these weren’t used in real chats, no conclusions were drawn regarding their effectiveness or effect on student satisfaction. For the same reason, no data was collected on how traditional search techniques, such as fulltext search, affected the metrics collected in the study.


The perceived speed, as reported by the users, was just slightly slower for the open source model used, as can be seen in \autoref{fig:feedback_01_frequency_of_answer_for_question_cbfea1}. This was also backed up by the recorded time each response took in \autoref{fig:performance_05_daily_average_response_time_including_pending_time}.


% \sweExpl{Slutsatser}
% \engExpl{Describe the conclusions (reflect on the whole introduction given in Chapter 1).}




% \engExpl{Discuss the positive effects and the drawbacks.\\
% Describe the evaluation of the results of the degree project.\\
% Did you meet your goals?\\
% What insights have you gained?\\
% What suggestions can you give to others working in this area?\\
% If you had it to do again, what would you have done differently?}


% \sweExpl{Uppfyllde du dina mål?\\
% Vilka insikter har du fått?\\
% Vilka förslag kan du ge till andra som arbetar inom detta område?
% Om du skulle göra detta igen, vad skulle du ha gjort annorlunda?}


\section{Limitations}
\label{sec:limitations}


% \sweExpl{Begränsande faktorer\\Vad gjorde du som begränsade dina ansträngningar? Vilka är begränsningarna i dina resultat?}
% \engExpl{What did you find that limited your efforts? What are the limitations of your results?}


\section{Future work}
\label{sec:futureWork}






% \sweExpl{Vad du har kvar ogjort?\\Vad är nästa självklara saker som ska göras?\\Vad tips kan du ge till nästa person som kommer att följa upp på ditt arbete?}
% \engExpl{Describe valid future work that you or someone else could or should do.\\
% Consider: What you have left undone? What are the next obvious things to be done? What hints can you give to the next person who is going to follow up on your work?}




Due to the breadth of the problem, only some of the initial goals have been
met. In these section we will focus on some of the remaining issues that
should be addressed in future work. ...


\subsection{What has been left undone?}
\label{what-has-been-left-undone}


The prototype does not address the third requirment, \ie a yearly unavailability of less than 3 minutes; this remains an open problem. ...


\subsubsection{Cost analysis}


% \generalExpl{Example of a missing component}


The current prototype works, but the performance from a cost perspective makes this an impractical solution. Future work must reduce the cost of this solution; to do so, a cost analysis needs to first be done. ...


\subsubsection{Security}


% \generalExpl{Example of a missing component}


A future research effort is needed to address the security holes that results from using a self-signed certificate. Page filling text mass. Page filling text mass. ...


\subsection{Next obvious things to be done}


In particular, the author of this thesis wishes to point out xxxxxx remains as a problem to be solved. Solving this problem is the next thing that should be done. ...


\section{Reflections}
\label{sec:reflections}


% \sweExpl{Reflektioner}
% \sweExpl{Vilka är de relevanta ekonomiska, sociala, miljömässiga och etiska aspekter av ditt arbete?}
% \engExpl{What are the relevant economic, social,
%   environmental, and ethical aspects of your work?
% }




One of the most important results is the reduction in the amount of
energy required to process each packet while at the same time reducing the
time required to process each packet.


% The thesis contributes to the \gls{UN}\enspace\glspl{SDG} numbers 1 and 9 by
% xxxx.


\noindent\rule{\textwidth}{0.4mm}


% \engExpl{In the references, let Zotero or other tool fill this in for you. I suggest an extended version of the IEEE style, to include URLs, DOIs, ISBNs, etc., to make it easier for your reader to find them. This will make life easier for your opponents and examiner. \\IEEE Editorial Style Manual: \url{https://www.ieee.org/content/dam/ieee-org/ieee/web/org/conferences/style_references_manual.pdf}}
% \sweExpl{Låt Zotero eller annat verktyg fylla i det här för dig. Jag föreslår en utökad version av IEEE stil - att inkludera webbadresser, DOI, ISBN osv. - för att göra det lättare för läsaren att hitta dem. Detta kommer att göra livet lättare för dina opponenter och examinator.}


\cleardoublepage
