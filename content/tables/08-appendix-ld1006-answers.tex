\begin{table}[h]
\centering
{\small
\begin{tabularx}{\textwidth}{@{}lX@{}}
\toprule
\textbf{ID} & \textbf{Answer} \\ \midrule
1 & För enklare frågor fungerar botten bra. \\ \hdashline
2 & Jag är helt ny användare så det är svårt att säga något nu :) \\ \hdashline
3 & Använde dem bara för att svara på AI verktyg frågorna. \\ \hdashline
4 & Jag är relativt ny i användandet av AI, därför prövade jag lite olika inlägg i AI-Copilot. \\ \hdashline
5 & Jag frågade om zoom-mötena spelades in och lades upp på sidan - jag fick svaret ja. Så där ger den mig fel information. \\ \hdashline
6 & Det fungerade bra tycker jag. Ställde bara enkla frågor som jag redan hade svaret på. \\ \hdashline
7 & Den svarar väldigt svepande och allmänt. \\ \hdashline
8 & Frågade och fick ett korrekt svar. \\ \hdashline
9 & En relativt positiv upplevelse. \\ \hdashline
10 & Den var oväntat positiv! För mig var det inte självklart med kopplingen mellan hämtningskapacitet och lagringskapacitet vilket jag ställde en fråga om. Jag fick ett utmärkt svar och förstod bättre än innan. \\ \hdashline
11 & Aldrig använt AI chat tidigare, men tycker det fungerade riktigt bra. Jag frågade bland annat om hur man undervisar lågpresterande elever och fick då både sammanhängande text och även en lista med råd på hur man kan arbeta med dessa elever. Jag frågade också om "den tysta läraren" och fick då svar på min fråga. \\ \hdashline
12 & Bra. Jag fick svar på min fråga. \\ \hdashline
13 & Bra upplevelse. Den hjälpte mig att besvara på frågor kring sena inlämningar. \\
\bottomrule
\end{tabularx}
}
\vspace{2mm}
\caption{Answers to the question: Vad var din upplevelse av att använda chattboten för kursrelaterade frågor?}
\label{tab:appendix_typeform_table_question_experience}
\end{table}

\begin{table}[h]
\centering
{\small
\begin{tabularx}{\textwidth}{@{}lX@{}}
\toprule
\textbf{ID} & \textbf{Answer} \\ \midrule
1 & För frågor som är relaterade till självaste kursmaterialet, dvs teoretiska frågor, är den näst intill inkapabel. \\ \hdashline
2 & Det är bra att få bra verktyg, men det är bra tiden som visar om det är bra/effektivt eller inte. \\ \hdashline
3 & Ser ut som det kan bli hjälpsamt när det gäller definitioner. Men för djupare frågor var kritisk tänkande krävs, chatbot svarar igen baserat på kursens definitioner. \\ \hdashline
4 & Jag skrev några olika inlägg med fråga om lärstilar kopplat till forskning och önskade svar skrivet av en 15-åring och senare ett svar skrivet av en forskare. Det var intressant att se hur boten anpassade texterna utifrån min önskan. Texten var välformulerade och anpassade utifrån vem som enligt min önskan skulle vara författaren. Dock fick jag endast en källhänvisning till forskning. \\ \hdashline
5 & Jag frågade om zoom-mötena spelades in och lades upp på sidan - jag fick svaret ja. Så där ger den mig fel information. \\ \hdashline
6 & Jag kunde söka efter ett specifikt avsnitt i boken för att få veta vilket kapitel jag skulle leta i. Det hade jag inte förväntat mig. \\ \hdashline
7 & Jag har inga direkta förväntningar. \\ \hdashline
8 & Kan inte säga att jag tog det vidare. Fick inte svar på djupare frågor i kursen. \\ \hdashline
9 & Det enda var väl att den någon gång svarade på engelska fast frågan ställdes på svenska. Sen svarade den antingen ganska kort och inte så uttömmande som andra chatbotar på vissa frågor och andra frågor svarade den inte på hela frågan. Exempelvis ställde jag fråga om den kunde nämna 3 inre samt 3 yttre motivationer, då nämnde den enbart 3 inre. \\ \hdashline
10 & Nej det var jag inte, den var precis så hjälpsam som jag behövde. \\ \hdashline
11 & Chatboten klarade att svara på alla de frågor jag ställde på ett tillfredsställande sätt. \\ \hdashline
12 & Vet ej, har bara testat 1 gång. \\ \hdashline
13 & den var mycket hjälpsam. \\
\bottomrule
\end{tabularx}
}
\vspace{2mm}
\caption{Answers to the question: Uppstod det situationer där chatboten var särskilt hjälpsam eller där den inte levde upp till dina förväntningar?}
\label{tab:appendix_typeform_table_question_helpful_or_fell_short}
\end{table}

\begin{table}[h]
\centering
{\small
\begin{tabularx}{\textwidth}{@{}lX@{}}
\toprule
\textbf{ID} & \textbf{Answer} \\ \midrule
1 & Teoretiska frågor från kursmaterialet. \\ \hdashline
2 & Svårt att svara på den frågan nu. \\ \hdashline
3 & Nej, hittills inte alls. \\ \hdashline
4 & Svårt att säga eftersom chattboten är helt ny för mig. \\ \hdashline
5 & Nej - är väl allmänt lite skeptisk till olika chatbotar (även på andra diverse sidor där man kan chatta med en chatbot). Jag mailar eller ringer hellre till en person för att få svar. \\ \hdashline
6 & Detta kan jag inte svara på i nuläget. \\ \hdashline
7 & Nej. \\ \hdashline
8 & Nej. \\ \hdashline
9 & Inga specifika frågor men jag kanske skulle välja att ställa frågor till chatboten i första hand då jag tänker att jag får svar fortare av den då den inte blir lika överbelastad som lärare kan vara. \\ \hdashline
10 & Kanske om det var frågor jag kände mig dum att fråga. \\ \hdashline
11 & Åh, svårt, men det finns det säkert. Frågor som man kanske tycker att man ska ha koll på ställer man säkert hellre till en chatbot än till en lärare. \\ \hdashline
12 & Frågor om kursen, administrativa frågor är bra att ställa till chatboten. Man får ju svar direkt vilket underlättar mycket. \\ \hdashline
13 & var inte helt säker på till vem man skulle vända sig till för frågor så då var det bra att den här fanns som kunde svara istället. \\
\bottomrule
\end{tabularx}
}
\vspace{2mm}
\caption{Answers to the question: Finns det frågor du hellre ställer till chatboten än till lärare eller lärarassistent? Vilka?}
\label{tab:appendix_typeform_table_question_chatbot_vs_teacher}
\end{table}

\begin{table}[h]
\centering
{\small
\begin{tabularx}{\textwidth}{@{}lX@{}}
\toprule
\textbf{ID} & \textbf{Answer} \\ \midrule
1 & Nej. \\ \hdashline
2 & Det är svårt att säga eftersom chattboten är helt ny för mig. \\ \hdashline
3 & Jo, kanske om, i framtiden, jag kommer inte att hinna läsa hela kursens material inom kort tid. \\ \hdashline
4 & Det är svårt att säga eftersom chattboten är helt ny för mig. \\ \hdashline
5 & Nej - inte som den fungerar nu. \\ \hdashline
6 & Kanske framöver när jag lärt mig mer om det. Just nu litar jag inte fullt ut på att det fungerar. \\ \hdashline
7 & Nej. \\ \hdashline
8 & Nej egentligen inte. Jag har personligen ringa intresse för detta förutom hur mina elever kan tänkas använda det. Så i min profession är jag intresserad men för mitt eget lärande - nej. \\ \hdashline
9 & Beror på vilken slags kurs det är och hur gammal chatboten är. Den kan vara användbar om man ställer frågor kring ex marknadsföring men jag ställde en enkel mattefråga till den nyss och den svarade tyvärr fel, liksom ChatGPT 3,5 gjorde. \\ \hdashline
10 & Det vet jag inte i dagsläget. \\ \hdashline
11 & Ja, det kan ju eventuellt förtydliga begrepp som man tycker är svåra att förstå, samt evt underlätta informationssökning. \\ \hdashline
12 & Vet inte, har inte så mycket erfarenhet. \\ \hdashline
13 & Den kan förtydliga saker och ting för mig vid arbetet med case och quiz. \\
\bottomrule
\end{tabularx}
}
\vspace{2mm}
\caption{Answers to the question: Tror du användningen av Chatboten kan ändra hur du tar till dig information i en kurs? Hur då?}
\label{tab:appendix_typeform_table_question_changed_access}
\end{table}

\begin{table}[h]
\centering
{\tiny
\begin{tabularx}{\textwidth}{@{}lX@{}}
\toprule
\textbf{ID} & \textbf{Answer} \\ \midrule
1 & Jag ser inte speciellt positivt på AI-verktygen just när det kommer till matematik, för i slutändan behöver man "plöja" problem som man själv behöver förstå, applicera teori och tänka kring. Men det är möjligt att det är användbart i andra ämnen och områden. Därför ser jag personligen ingen användning av det idag. \\ \hdashline
2 & Jag gick på en kurs för en månad sedan där jag fick veta om AI-verktyg, det låter spännande och skrämmande samtidigt :) \\ \hdashline
3 & Jag är inte ett stort fan av chatbots. Jag märkte en gång att några av mina elever använde det för att lösa ordproblem; det gjorde inte ett bra jobb så det fortsatte inte med det. \\ \hdashline
4 & Min fundering är hur man som lärare ska kunna se eller kontroller om en text är författad av en student eller av en AI? Jag testade att skriva in samma fråga två gånger och fick olika svar. Som jag ser det är det inte en fråga om användandet av AI eller inte. Det kommer att användas även i studiesammanhang vare sig vi vill eller inte. Jag tänker att via AI-verktyget kan vara en hjälp för att få en snabb koll kring en fråga. Den kan också hjälpa till att förstärka det man läser och lär sig, men den kan inte, i alla fall för mig, ersätta den lärprocess som sker när man tar till sig kunskap jag läser (helst analog) i kurslitteratur eller studier samt det som sker i kunskapsutbyte med andra personer. Jag är ännu för okunnig om AI-användning för att tänka kring hur det kan användas i undervisning inom matematik. \\ \hdashline
5 & Jag undervisar i en kurs på gymnasiet för våra åk 3 elever som heter Ledarskap och organisation. Där är AI inte med i det centrala innehållet, men något som jag tycker bör lyftas i och med att det tar större och större del i vår vardag men även arbetsliv. De får diskutera och intervjua (i samband man annat område med) hur ett företag/organisation använder sig av AI - men även nu på slutseminariet får de diskutera utifrån vad de själva tycker och om de t.ex. vill byta ut chefen mot en AI (utifrån en artikel de läst). Även viktigt att förklara för dem hur t.ex. chat GPT fungerar (för alla har använt den till skolarbete på olika sätt), att lära dem hur de kan använda den på ett smart sätt som en "studybudy", men att vara lite mer vaksam när det gäller information m.m. Men en jättebra källa om man söker inspiration till något, skapa quizar m.m. \\ \hdashline
6 & Jag provade att tillverka läsförståelsefrågor åt eleverna med hjälp av ChatGPT. Det såg väldigt bra ut till en början men när man läste mer noggrant fanns det ingen som helst substans i texten. Förmodligen för att jag ställt frågan på helt fel sätt. Jag är inte negativ till AI-verktyg men känner att jag måste lära mig mer innan jag kan/vågar använda det i undervisningen. \\ \hdashline
7 & Ja, jag har använt AI för att skapa flera varianter av enklare matteprov. Men man måste alltid modifiera. Jag ställer ibland frågor. Eleverna använder det för att svara på frågor ibland, men man kan se att det inte är deras egna ord ganska ofta. Det kommer nog att bli stort i framtiden. Men just nu är vi inte riktigt där, är känslan. \\ \hdashline
8 & Jag vet att många elever, särskilt på gymnasiet, använder det. De använder det mest för att få jobbet gjort och lämnar in. Här är det viktigt att vi lärare lär oss om hur det fungerar. Här blir det liksom en kollaps i lärandet som jag ser det. Vi pratas inte ens om prestation kontra lärande utan en AI som levererar åt dig. Problematiskt skulle jag säga. I matematiken anser jag att det är ett mindre problem men i andra ämnen med tex skriftliga inlämningar så behöver man tänka över sin bedömning som lärare. I matematiken skulle det kunna vara ett stöd för elever i det att de får exempel på lösningar till matematiska problem som de sedan kan överföra till andra uppgifter. Eleverna behöver dock få tillgång till studiestrategier som ger dem faktisk nytta av den information AI kan ge dem. \\ \hdashline
9 & Med mina elever använder jag inte AI-verktyg och då jag mest håller på med matte så tror jag inte att jag kommer göra det heller inom en snar framtid. Däremot brukar jag själv använda mig av ChatGPT 3,5 (tycker USD \$20/mån är lite väl dyrt för 4,0, men ska nog inom en snar framtid börja använda den istället). Håller på att starta upp ett företag och då tycker jag ChatGPT är användbar vid frågor kring hur man kan attrahera kunder, marknadsföring, mm. \\ \hdashline
10 & Idag använder inte mina elever några sådana verktyg och det beror nog på att jag är ovan vid och okunnig om det själv. Men jag ska lära mig mer! Först då kan jag svara bättre på den frågan. \\ \hdashline
11 & Jag undervisar just nu i en årskurs 1 så AI används inte av mina elever. Inte heller av mig i min yrkesroll som lärare. Jag skulle dock absolut kunna tänka mig att använda mig av AI-verktyg om jag fick utbildning i hur det ska användas och vilka fallgropar man bör undvika. \\ \hdashline
12 & Jag använder i nuläge väldigt sällan AI-verktyg, men jag har testat att göra en planering till en schemabrytande dag med fokus på motivation. Det gick ganska bra. Jag fick bra tips på aktiviteter, och det gick snabbare än att googla själv. Eleverna får inte använda AI-verktyg just nu, men vi har haft diskussion om det. Jag är öppen för att testa nya AI-verktyg. \\ \hdashline
13 & Jag använder det ibland till hjälp med att formulera uppgifter. Jag vet att också elever använder det bland annat för att söka svar på uppgifter som de sedan kan öva på. \\ \hdashline
14 & Jag missade att använda denna chattbott men brukar uppskatta när det finns. Finns det frågor du hellre ställer till chatboten än till lärare eller lärarassistent? Vilka? Är det frågor men relativ enkla svar så skulle jag gärna ställa dem till en chatbott. Känslan av att fråga något som redan sagts av lärare minskar om man har en chatt att fråga. Jag brukar uppskatta att ta reda på saker själv och genom en chattbott så kan frågorna /svaren utveckla sig jämfört med bara färdiga fråga och svar. I detta läget tror jag inte att man kommer att ta åt sig information på annat sätt. Vi använder i nuläget inget AI-verktyg. Bland mina elever så tror jag att det är viktigt att de förstår hur AI -verktyg fungerar och har ett källkritiskt tänkande. Vidare så tror jag att arbetsmarknaden kommer att ändras och att det tex blir viktigt att kunna /förstå att skriva tex promptar. \\
\bottomrule
\end{tabularx}
}
\vspace{2mm}
\caption{Answers to the question: Hur tänker du kring användandet av AI-verktyg både av dig själv och dina elever? Använder du och/eller dina elever några verktyg idag? Vilka, till vad?}
\label{tab:appendix_typeform_table_question_AI_tools}
\end{table}
