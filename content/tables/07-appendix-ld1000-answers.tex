\begin{table}[h]
\centering
{\small
\begin{tabularx}{\textwidth}{@{}lX@{}}
\toprule
\textbf{ID} & \textbf{Answer} \\ \midrule
1 & Den svarar bra på frågor som handlar om kursen och den hjälper en att hitta i canvas om det är nån specifik sak man söker. Det gör den bra tycker jag. \\ \hdashline
2 & Upplevelsen var bra, den svarade korrekt på alla kurs relaterade frågor som jag ställde. \\ \hdashline
3 & Min upplevelse av att använda chattboten för kursrelaterade frågor var mycket positiv och över min förväntan. Den var skalad och konkret. Gissar att just dens informationsutbud här matchade men utmärkte sig i saklighet och korrekt svar när man kontrollerade mot fakta. \\ \hdashline
4 & Den svarade lite långsamt, men verkade ge korrekt svar. \\ \hdashline
5 & Bra fast alltför grunt. \\ \hdashline
6 & Fick känslan av att kapaciteten finns att besvara frågor om kursen på ett relevant sätt, men något ytligt. \\ \hdashline
7 & Min upplevelse var positiv, det känns kul när man skriver en fråga och man får svar 😊 \\ \hdashline
8 & Jag tycker att chattbotten kan svara på frågor om hur många quiz man måste göra osv. \\ \hdashline
9 & Den svarar informativt och detaljerat och avslutar med information om var jag kan läsa mer. Användbart eftersom man får svar så snabbt. \\
\bottomrule
\end{tabularx}
}
\vspace{2mm}
\caption{Answers to the question: Vad var din upplevelse av att använda chattboten för kursrelaterade frågor?}
\label{tab:appendix_typeform_table_question_experience}
\end{table}


\begin{table}[h]
\centering
{\small
\begin{tabularx}{\textwidth}{@{}lX@{}}
\toprule
\textbf{ID} & \textbf{Answer} \\ \midrule
1 & Jag testade att fråga hur många quiz som ingick i kursen och den beskrev utförligt hur många den bestod av, hur många rätt man måste få för att bli godkänd samt hur många försök man har på sig. Jag tyckte den var extra hjälpsam genom att den gav en länk till "Examinerande moment" så att man själv kan läsa på mer om de olika momenten i kursen! \\ \hdashline
2 & Jag prövade att ställa "större" frågor som hur bestämt bestämma struktureringen av inlärningen av kursen. Där jag tyckte att svaret var väldigt utförligt och kan klassas som extra hjälpsamt. Svaret var en 6 stegs plan om hur hanteringen och bearbetningen av alla moment och moduler skulle gå till. \\ \hdashline
3 & Chatboten var särskilt hjälpsam och den levde upp till dina förväntningar mer än väntat då den var väldigt konkret i sitt svar och snabb på att hitta det exakta svaret jag sökte. \\ \hdashline
4 & Det som kändes extra hjälpsamt var att man fick en länk till mer detaljerad information (gissningsvis källan den inhämtade information från). \\ \hdashline
5 & Nej. \\ \hdashline
6 & Testade att ställa ett par frågor rent allmänt om examinerande moment och upplevde svaren var tillfredsställande. \\ \hdashline
7 & Jag ställde en bredd fråga bland annat och då svarade den med ett par påstående som jag fick välja bland för att få bättre anpassat svar vilket jag uppfattade som hjälpsamt. \\ \hdashline
8 & Den svarade på vad man behövde göra för att bli godkänd i kursen. \\ \hdashline
9 & Den är bra på att ta mig vidare när jag kört fast. Då kan jag fråga boten vad jag ska skriva nu, vilket den förstår inte svarar på men den förtydligar ämnet och säger "tänk till exempel såhär", och så är jag igång igen. \\
\bottomrule
\end{tabularx}
}
\vspace{2mm}
\caption{Answers to the question: Uppstod det situationer där chatboten var särskilt hjälpsam eller där den inte levde upp till dina förväntningar?}
\label{tab:appendix_typeform_table_question_helpful_or_fell_short}
\end{table}

\begin{table}[h]
\centering
{\small
\begin{tabularx}{\textwidth}{@{}lX@{}}
\toprule
\textbf{ID} & \textbf{Answer} \\ \midrule
1 & Det finns egentligen inte någon speciell fråga jag hellre ställer till boten än en lärare. För mig handlar det kanske mer om att jag hellre skriver till boten för att få ett svar på direkten och slipper vänta! \\ \hdashline
2 & Till viss del, särskilt om det är relativt simpla frågor som jag bara vill ha ett snabbt svar på. \\ \hdashline
3 & Jag uppskattar generellt mer att ställa frågor till lärare men tycker att en chattbot är som fungerar som denna skulle vara användbar för att studera på tider när läraren ej är aktiv. Då gällande alla frågor. \\ \hdashline
4 & Man vill ju inte störa kursassistenter eller lärare i onödan för att fråga frågor om kursupplägg, deadlines, etc., så då kändes chattboten som rätt ställe att söka på först. \\ \hdashline
5 & Frågan, varför då? \\ \hdashline
6 & Nej! I alla situationer föredrar jag levande människor. Många onlinekurser har så pass få fysiska onlineträffar, så dessa är oerhört värdefulla. Intressant med mänskliga möten i sig, att få bekantskap med ansiktena bakom en kurs, och att vid behov kunna ställa frågor direkt till dem. \\ \hdashline
7 & Nej, det tror jag inte, frågor kan jag ställa till både lärare, lärarassistent och chatboten? Det som är viktigt är att man kan lita på att det blir rätt svar och att inte chatboten svarar på vad som gäller för andra kurser tex om man frågar vilka examinerande moment som finns. \\ \hdashline
8 & Nej, det tror jag inte. Kanske så kallade "dumma" frågor som borde vara självklara kanske man hellre frågar chatbotten. \\ \hdashline
9 & Är den här diskussionen obligatorisk? \\
\bottomrule
\end{tabularx}
}
\vspace{2mm}
\caption{Answers to the question: Finns det frågor du hellre ställer till chatboten än till lärare eller lärarassistent? Vilka?}
\label{tab:appendix_typeform_table_question_chatbot_vs_teacher}
\end{table}

\begin{table}[h]
\centering
{\small
\begin{tabularx}{\textwidth}{@{}lX@{}}
\toprule
\textbf{ID} & \textbf{Answer} \\ \midrule
1 & Det tror jag. Jag hade använt mig av den för att snabbt få fram den informationen jag vill ha. Vilket kan vara störande i kursen som är stora och har många moduler att navigera igenom. Då hade jag definitivt använt mig av boten. \\ \hdashline
2 & Den kan nog hjälpa till med effektiviteten en del med ställandet av simpla frågor och kolla upp vad som examineras på och så vidare. \\ \hdashline
3 & Om jag visste att det var tillåtet skulle jag använda chattboten 100\% till att ta in information då den verkar kunna ta ut det viktiga och formulera text som ger bilden av vad materialet gör. \\ \hdashline
4 & För min del föredrar jag nog att ha all relevant information koncist beskriven och samlad på en sida på Canvas, men om informationen är väldigt spridd kan jag tänka mig att chattboten skulle kunna vara till hjälp. \\ \hdashline
5 & Ja. Om den kan bli en gruppkompis som aldrig är sur och har oändligt tålamod. \\ \hdashline
6 & En genomarbetad studiehandledning, vilket ingår i de flesta onlinekurser är fullt tillräckligt, då behövs ingen chatbot. Tilltalas personligen av att ha all information samlad om en kurs i en studiehandledning. \\ \hdashline
7 & Nej det tror jag inte, men det är egentligen svårt att svara få jag använt det för lite för att ha någon egentlig uppfattning. \\ \hdashline
8 & Nej, inte väsentligt. Men det kan göra att man snabbare får svar på sina frågor än om man frågar en lärare. \\ \hdashline
9 & Om det är något jag inte förstår kan jag fråga och få ett sammanfattat svar. Då kan jag välja om jag vill läsa mer utförligt eller inte beroende på ämnets relevans. \\
\bottomrule
\end{tabularx}
}
\vspace{2mm}
\caption{Answers to the question: Tror du användningen av Chatboten kan ändra hur du tar till dig information i en kurs? Hur då?}
\label{tab:appendix_typeform_table_question_changed_access}
\end{table}

\begin{table}[h]
\centering
{\small
\begin{tabularx}{\textwidth}{@{}lX@{}}
\toprule
\textbf{ID} & \textbf{Answer} \\ \midrule
1 & Jag tycker att AI-verktyg är en bra grej och använder det i mina studier. Främst använder jag det för att få saker förklarade för mig på andra sätt som kanske är lättare för mig att förstå. Vissa lärare kanske undervisar på ett sätt som inte passar mig. Då är det smidigt att kunna lägga in deras föreläsningar i ett sådant verktyg och be verktyget att formulera om det eller förtydliga otydliga delar. \\ \hdashline
2 & Jag kan se det positiva i att utnyttja AI-verktyg i mina studier men använder det för tillfället inte i mina studier idag. \\ \hdashline
3 & Jag använder endast AI ifall jag vill hitta något jag inte direkt gör i min litteratur men använder den mera privat för att ställa frågor om diverse saker. \\ \hdashline
4 & Jag använder inga AI-verktyg i mina studier och känner nog inte riktigt att det är där de gör mest nytta heller (och då arbetar jag ändå med AI, och har en generellt positiv inställning till det). \\ \hdashline
5 & Chat gtp för programmering. \\ \hdashline
6 & Jag använder inga AI-verktyg i mina studier. Tycker om att vara kreativ själv. \\ \hdashline
7 & Jag använder inte några verktyg idag men jag tycker att det är intressant och testar gärna olika. \\ \hdashline
8 & Jag tycker det är bra, jag använder det ganska ofta för att få en förklaring på något som jag tyckte behövdes förtydligas. Jag använder det också när jag skriver rapporter som ett avancerat rättstavningsprogram som kan hjälpa så att det blir en bra meningsuppbyggnad och låter bra. \\ \hdashline
9 & Jag har inte haft för vana att använda AI-verktyg. Jag vet att de används för att formulera text ibland, vilket jag tycker är lite fusk, men samtidigt en ganska bra idé eftersom så många har svårt för just det. \\
\bottomrule
\end{tabularx}
}
\vspace{2mm}
\caption{Answers to the question: Hur tänker du kring användandet av AI-verktyg i dina studier? Använder du några verktyg idag? Vilka, till vad?}
\label{tab:appendix_typeform_table_question_AI_tools}
\end{table}
