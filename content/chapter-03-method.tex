\chapter{Method or Methods}
\label{ch:methods}


% \sweExpl{Metod eller Metodval}
% \generalExpl{This chapter is about Engineering-related
%   content, Methodologies and Methods.  Use a self-explaining title.\\The
%   contents and structure of this chapter will change with your choice of
%   methodology and methods.}




% \generalExpl{Describe the engineering-related contents (preferably with models) and the research methodology and methods that are used in the degree project.\\
% Give a theoretical description of the scientific or engineering methodology  you are going to use and why have you chosen this method. What other methods did you consider and why did you reject them?\\
% In this chapter, you describe what engineering-related and scientific skills you are going to apply, such as modeling, analyzing, developing, and evaluating engineering-related and scientific content. The choice of these methods should be appropriate for the problem. Additionally, you should be conscious of aspects relating to society and ethics (if applicable). The choices should also reflect your goals and what you (or someone else) should be able to do as a result of your solution - which could not be done well before you started.}


% The purpose of this chapter is to provide an overview of the research method
% used in this thesis. Section~\ref{sec:researchProcess} describes the research
% process. Section~\ref{sec:researchParadigm} details the research
% paradigm. Section~\ref{sec:dataCollection} focuses on the data collection
% techniques used for this research. Section~\ref{sec:experimentalDesign}
% describes the experimental design. Section~\ref{sec:assessingReliability}
% explains the techniques used to evaluate the reliability and validity of the
% data collected. Section~\ref{sec:plannedDataAnalysis} describes the method
% used for the data analysis. Finally, Section~\ref{sec:evaluationFramework}
% describes the framework selected to evaluate xxx.




This chapter outlines the methodologies and procedures used for conducting the research described in this thesis. The focus is on presenting the chosen methods for data collection, analysis, and evaluation of the deployment and effectiveness of AI-assistants within Canvas course rooms at KTH.
% \sweExpl{Vilka vetenskaplig eller ingenjörs-metodik ska du använda och varför har du valt den här metoden. Vilka andra metoder gjorde du övervägde du och varför du avvisar dem.
% Vad är dina mål? (Vad ska du kunna göra som ett resultat av din lösning - vilken inte kan göras i god tid innan du började)
% Vad du ska göra? Hur? Varför? Till exempel, om du har implementerat en artefakt vad gjorde du och varför? Hur kommer du utvärdera den.
% Syftet med detta kapitel är att ge en översikt över forsknings metod som
% används i denna avhandling. Avsnitt~\ref{sec:researchProcess} beskriver forskningsprocessen. Avsnitt~\ref{sec:researchParadigm} beskriver forskningsparadigmen detaljerat. Avsnitt~\ref{sec:dataCollection} fokuserar på datainsamlingstekniker som används för denna forskning. Avsnitt~\ref{sec:experimentalDesign} beskriver experimentell
% design. Avsnitt~\ref{sec:assessingReliability} förklarar de tekniker som används för att utvärdera
% tillförlitligheten och giltigheten av de insamlade uppgifterna. Avsnitt~\ref{sec:plannedDataAnalysis}
% beskriver den metod som används för dataanalysen. Slutligen, Avsnitt~\ref{sec:evaluationFramework}
% beskriver ramverket som valts för att utvärdera xxx.\\
% Ofta kan man koppla ett antal följdfrågor till undersökningsfrågan och problemlösningen t ex\\
% (1) Vilken process skall användas för konstruktion av lösningen och vilken process skall kopplas till denna för att svara på undersökningsfrågan?\\
% (2) Hur och vilket resultat (storheter) skall presenteras både för att redovisa svar på undersökningsfrågan (resultatkapitlet i denna rapport) och redovisa resultat av problemlösningen (prototypen, ofta dokument som bilagor men vilka dokument och varför?).\\
% (3) Vilken teori/teknik skall väljas och användas både för undersökningen (taxonomi, matematik, grafer, storheter mm)  och  problemlösning (UML, UseCases, Java mm) och varför?\\
% (4) Vad behöver du som student leverera för att uppnå hög kvaliet (minimikrav) eller mycket hög kvalitet på examensarbetet?\\
% (5) Frågorna kopplar till de följande underkapitlen.\\
% (6) Resonemanget bygger på att studenter på hing-programmet ofta skall konstruera något åt problemägaren och att man till detta måste koppla en intressant ingenjörsfråga. Det finns hela tiden en dualism mellan dessa aspekter i exjobbet.
% }


\section{Research Process}
\label{sec:research_process}


The research process within this thesis consisted of three main phases. This section will outline each phase and what the purpose was.


\subsection{Prof of concept}


The study will aim to achieve an assistant using open source and permissively licensed \gls{LLM}s and \gls{RAG} techniques, that is comparable to the current best in class models provided under less permissive licesnses and are only available through properitery APIs. Considering that this is a fairly complex task, the research started with a long phase of constructing various proof of concepts.


There would be two main proof of concepts, one using proprietary models and \gls{RAG} techniques, and one strictly using software that is under open source licenses and can be self-hosted.


\subsection{Implementation of study software}


The second phase consists of constructing the software that will be used during the study. This software is the actual AI assistant and all it’s components, such as the course room crawler and indexer, the infrastructure to run \gls{LLM}s at scale and a \gls{GUI} to interact with the assistant. This software also needs to be able to exchange various different components that are the subject for the study, such as the \gls{LLM} being used in a chat. Furthermore, the software needs to be able to record user interactions and responses to questions.


\subsection{Conduct the study at KTH}


This phase consists of deploying and monitoring the assistant in the course rooms that have enrolled in the study. This might involve modifying the software such that it works in the new course room, or ensuring that there is enough capacity on the platform to sustain the new users.


\subsection{Analyse results}


TODO.


% \sweExpl{Undersökningsrocess och utvecklingsprocess}


% \sweExpl{Figur~\ref{fig:researchprocess} visar de steg som utförs för att genomföra\\
% Beskriv, gärna med ett aktivitetsdiagram (UML?), din undersökningsprocess och utvecklingsprocess.  Du måste koppla ihop det akademiska intresset (undersökningsprocess) med ursprungsproblemet (utvecklingsprocess)
% denna forskning.\\
% Aktivitetsdiagram från t ex UML-standard}






% \generalExpl{Example of using customized item labels.}
% Some steps in the process:
% \begin{enumerate}[leftmargin=*, label=\textbf{Step \arabic*}, ref=Step \arabic*] %labelindent=1em for indent
%         \itemsep0em
%         \item \label{x:s1} plan experiment,
%         \item \label{x:s2} conduct experiment,
%         \item \label{x:s3} analyze data from the experiment, and
%         \item \label{x:s4} discuss the results of the analysis.
% \end{enumerate}


\section{Research Paradigm}
\label{sec:researchParadigm}


% \sweExpl{Undersökningsparadigm\\
% Exempelvis\\
% Positivistisk (vad/hur fungerar det?) kvalitativ fallstudie med en deduktivt (förbestämd) vald ansats och ett induktivt(efterhand uppstår dataområden och data) insamlade av data och erfarenheter.}




\section{Data Collection}
\label{sec:dataCollection}


% \sweExpl{Datainsamling\\
% (Detta bör också visa att du är medveten om de sociala och etiska frågor som
% kan vara relevanta för dina data insamlingsmetod.)}
% \generalExpl{This should also show that you are aware of the social and ethical concerns that might be relevant to your data collection method.}




\subsection{Sampling}


\subsection{Sample Size}


\subsection{Target Population}


\section[Experimental design/Planned Measurements]{Experimental design and\\Planned Measurements}
\label{sec:experimentalDesign}


\subsection{Test environment/test bed/model}


% \engExpl{Describe everything that someone else would need to reproduce your test environment/test bed/model/… .}
% \sweExpl{Testmiljö/testbädd/modell\\
% Beskriv allt att någon annan skulle behöva återskapa din testmiljö / testbädd / modell / …}


\subsection{Hardware/Software to be used}




\section{Assessing reliability and validity of the data collected}
\label{sec:assessingReliability}


% \sweExpl{Bedömning av validitet och reliabilitet hos använda metoder och insamlade data }




\subsection{Validity of method}
\label{sec:validtyOfMethod}


% \sweExpl{Giltigheten av metoder\\
%   Har dina metoder gett dig de rätta svaren och lösningarna? Var metoderna korrekta?}


% \engExpl{How will you know if your results are valid?}
% \engExpl{Remember that validity is about the \textit{accuracy} of a measurement while reliability is about the \textit{consistency} of the measurement values under the same conditions (\ie repeatability).}


\subsection{Reliability of method}
\label{sec:reliabilityOfMethod}


% \sweExpl{Tillförlitlighet av för metoder\\
% Hur bra är dina metoder, finns det bättre metoder? Hur kan du förbättra dem?}
% \engExpl{How will you know if your results are reliable?}


\subsection{Data validity}
\label{sec:dataValidity}


% \sweExpl{Giltigheten av uppgifter\\
% Hur vet du om dina resultat är giltiga? Är ditt resultat rättvisande?}


\subsection{Reliability of data}
\label{sec:reliabilityOfData}




% \sweExpl{Tillförlitlighet av data\\
% Hur vet du om dina resultat är tillförlitliga? Hur bra är dina resultat?}


\section{Planned Data Analysis}
\label{sec:plannedDataAnalysis}


% \sweExpl{Metod för analys av data}


\subsection{Data Analysis Technique}
\label{sec:dataAnalysisTechnique}


\subsection{Software Tools}
\label{sec:softwareTools}




\section{Evaluation framework}
\label{sec:evaluationFramework}


% \sweExpl{Utvärdering och ramverk\\
% Metod för utvärdering, jämförelse mm. Kopplar till kapitel~\ref{ch:resultsAndAnalysis}.}


\section{System documentation}
\label{sec:systemDocumentation}




% \sweExpl{Systemdokumentation\\
% Med vilka dokument och hur skall en konstruerad prototyp dokumenteras? Detta blir ofta bilagor till rapporten och det som problemägaren till det ursprungliga problemet (industrin) ofta vill ha.\\
% Bland dessa bilagor återfinns ofta, och enligt någon angiven standard, kravdokument, arkitekturdokument, designdokumnet, implementationsdokument, driftsdokument, testprotokoll mm.}
% \generalExpl{If this is going to be a complete document consider putting it in as an appendix, then just put the highlights here.}




\cleardoublepage