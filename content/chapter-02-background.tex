\chapter{Background}
\label{ch:background}


% \sweExpl{Bakgrund}
% \generalExpl{When you do your literature study, you should have a nearly complete Chapters 1 and 2.\\
% You may also find it convenient to introduce the future work section into your report early – so that you can put things that you think about but decide not to do now into this section.\\
% Note that later you can move things between this future work section and what you have done as you may change your mind about what to do now versus what to put off to future work.
% }
% \generalExpl{What does a reader (another x student -- where x is your study line) need to know to understand your report?
% What have others already done? (This is the “related work”.) Explain what and
% how prior work/prior research will be applied on or used in the degree
% project/work (described in this thesis). Explain why and what is not used in
% the degree project and give valid reasons for rejecting the work/research.}


This chapter provides the necessary background for understanding the research conducted within this thesis. This chapter also showcase the related work for this thesis and how the research relates to it.


% \sweExpl{Vilken viktig litteratur och
% (forsknings-)artiklar har du studerat inom området (litteraturstudie)? }


\section{Neural Networks}




%        "Learning representations by back-propagating errors" by Rumelhart, Hinton, and Williams (1986): This is a pivotal paper that introduced the backpropagation algorithm, crucial for training neural networks.
%         "Gradient-based learning applied to document recognition" by Yann LeCun et al. (1998): This paper by LeCun introduces Convolutional Neural Networks (CNNs), significantly impacting image processing tasks.


\subsection{Large Language Models}


Mention transformers. Mention OpenAI and Mistral. How they can be used.


\section{Prompt engineering}


Explain it.


\section{Crawling}


\section{Information Retrieval}


Some stuff here


\subsection{Embedding functions}


Explain it. How it relates to LLMs.


\subsection{TFIDF}


\section{RAG}


Explain it.


\section{AI Agents}


Explain it.


\section{User research methods}


Explain it. The relevant ones that will be in this project.


\section{Related work area}


% \sweExpl{Relaterade arbeten}


...




\subsection{Major related work 1}


Carrier clouds have been suggested as a way to reduce the delay between the users and the cloud server that is providing them with content. However, there is a question of how to find the available resources in such a carrier cloud. One approach has been to disseminate resource information using an extension to OSPF-TE, see Roozbeh, Sefidcon, and Maguire \cite{roozbeh_resource_2013}.


\subsection{Major related work n}


\subsection{Minor related work 1}


...


\subsection{Minor related work n}


\section{Summary}


% \sweExpl{Det är trevligt om detta kapitel
%   avslutas med en sammanfattning. Till exempel kan du inkludera en tabell som
%   sammanfattar andras idéer och fördelar och nackdelar med varje - så som
%   senare kan du jämföra din lösning till var och en av dessa. Detta kommer
%   också att hjälpa dig att definiera de variabler som du kommer att använda
%   för din utvärdering.}


% \engExpl{It is nice to have this chapter conclude with a summary. For
%   example, you can include a table that summarizes other people's ideas and
%   benefits and drawbacks with each - so as later you can compare your solution
%   to each of them. This will also help you define the variables that you will
%   use for your evaluation.}


\cleardoublepage