\chapter{What you did}
\label{ch:whatYouDid}


% \engExpl{Choose your own chapter title to describe this}
% \sweExpl{[Vad gjorde du? Hur gick det till? – Välj lämplig rubrik (“Genomförande”, “Konstruktion”, ”Utveckling”  eller annat]}


% \engExpl{What have you done? How did you do it? What design decisions did you make? How did what you did help you to meet your goals?}
% \sweExpl{Vad du har gjort? Hur gjorde du det? Vilka designval gjorde du?\\
% Hur kom det du hjälpte dig att uppnå dina mål?}


% the following sets the TOC entry to break after the & - note you have to include the first letter of the following word as it get swolled by the \texorpdfstring{}{} processing


% \section[Hardware/Software design …/Model/Simulation model \&\texorpdfstring{\\}{ p} parameters/…]{Hardware/Software design …/Model/Simulation model \& parameters/…}


\section{Proof of Concepts}


The following section will outline the various \gls{POC} applications that were built before the actual software that was written to conduct the research outlined in this thesis. Each \gls{POC} will outline what it was trying to accomplish and what the outcome was.


\subsection{Langchain based applications}


Langchain is a company \footnote{\href{https://langchain.com}{langchain.com}} and framework \footnote{\href{https://python.langchain.com}{python.langchain.com}} for building context aware reasoning applications. The framework allows for easy composition of language models and \gls{RAG} techniques and tools that makes it easy to build chatbots with a connected knowledge base. This section outlines some \gls{POC}s that were made with the langchain framework.


\subsubsection{GPT-4 and text-embedding-3-large}
\label{sec:poc_gpt_langchain}


To build a chat application with an AI-assistant that has access to an external knowledge base, one of the most popular approaches is to use langchain to connect the following four parts.


\begin{enumerate}
        \item A \gls{LLM}, such as GPT-4 to run the chat.
        \item A \gls{LLM}, such as GPT-4 to run the query construction.
        \item An embedding function, such as OpenAI’s text-embedding-3-large used to index and query documents.
        \item A vector store, such as ChromaDB, that stores the vector embeddings and associated documents \footnote{\href{https://www.trychroma.com/}{trychroma.com/}}
\end{enumerate}.


In this configuration, Lanchain acts as the glue connecting these components and handling tasks like chunking larger documents. The goal of this \gls{POC} was to test a common approach for building AI assistants and evaluate its potential for use in the full study. \href{https://www.youtube.com/watch?v=bKjxi-NKRHo}{A video can be seen here} that showcases this \gls{POC}.


\subsubsection{Mistral 7B v0.2 and e5-large-v2}


There was a \gls{POC} constructed that had the same approach as the one outlined in \ref{sec:poc_gpt_langchain} with the notable requirement that all tools had to be under an open source licence. This meant the GPT-4 model and text-embedding-3-large models couldn’t be used. A similar version of the same \gls{POC} was made that used the Mistral 7B v0.2 model and the embedding function e5-large-v2 \cite{wang_text_2024}. These are both under an open source licence and are freely available on Huggingface \footnote{\href{https://huggingface.co/mistralai/Mistral-7B-Instruct-v0.2}{huggingface.co/mistralai/Mistral-7B-Instruct-v0.2} \href{https://huggingface.co/intfloat/e5-large-v2}{huggingface.co/intfloat/e5-large-v2}}. This \gls{POC} did however suffer from poor performance in initial tests for retrieval and performance. It was difficult to tune the prompts to get decent performance. This \gls{POC} showed it was difficult for the researcher to get good performance out of certain models using the langchain framework.


\subsection{Custom applications}


\subsubsection{Simple Python API for models on Huggingface}


Langchain and similar tools support running language models locally. However, working with the prompt templates in less advanced models than GPT-4 and achieving good retrieval and chat performance was challenging. Therefore, a simple \gls{POC} was developed to create higher-level Python abstraction APIs on top of the Hugging Face Transformers library that could be integrated into completely custom solutions. These APIs include examples like those shown in listings ~\ref{fig:python-apis-for-llm} and ~\ref{fig:python-apis-for-embeddings}.


\begin{listing}[H]
\centering
\renewcommand{\theFancyVerbLine}{\scriptsize\arabic{FancyVerbLine}}
\scriptsize
\begin{minted}[
frame=lines,
framesep=2mm,
baselinestretch=1.2,
fontsize=\scriptsize,
linenos
]{python}
def load_hf_model(
    model_path: str,
    device: str
) -> (transformers.AutoModelForCausalLM, transformers.AutoTokenizer):
    """
    Loads a Hugging Face causal language model and its tokenizer for a given
    model path and device.
    """

def generate_text(
    model: transformers.AutoModelForCausalLM,
    tokenizer: transformers.AutoTokenizer,
    device: str,
    params: Params,
    prompt: str
) -> str:
    """
    Generates text from a prompt using the specified model, tokenizer, and
    generation parameters.
    """

async def generate_text_streaming(
    model: transformers.AutoModelForCausalLM,
    tokenizer: transformers.AutoTokenizer,
    device: str,
    params: Params,
    prompt: str
) -> AsyncGenerator[str, None]:
    """
    Asynchronously generates text from a prompt, yielding tokens incrementally.
    Useful for streaming responses.
    """

def _tokenise_inputs(
    tokeniser: transformers.AutoTokenizer,
    input_texts: list[str],
    max_length: int = 8192
) -> dict:
    """
    Tokenizes the input texts with padding and truncation.
    """

def should_stop_generating(
    output_token_ids: list,
    tokenizer: transformers.AutoTokenizer,
    params: Params,
    token_id: int
) -> bool:
    """
    Determines whether to stop generating text based on stop conditions.
    """
\end{minted}
\caption{High level API on-top of Huggingface's tranformers library that can be used for generating text using models available on Huggingface.}
\label{fig:python-apis-for-llm}
\end{listing}

\begin{listing}[H]
\centering
\renewcommand{\theFancyVerbLine}{\scriptsize\arabic{FancyVerbLine}}
\scriptsize
\begin{minted}[
frame=lines,
framesep=2mm,
baselinestretch=1.2,
fontsize=\scriptsize,
linenos
]{python}
def load_hf_embedding_model(
    model_path: str,
    device: str
) -> (torch.nn.Module, transformers.AutoTokenizer):
    """
    Loads a Hugging Face embedding model and its tokenizer for a given model
    path and device.
    """

async def compute_embedding(
    model: torch.nn.Module,
    tokeniser: transformers.AutoTokenizer,
    text: str
) -> List[float]:
    """
    Computes and returns the normalized embedding for a given text using the
    specified model and tokenizer.
    """

def _compute_model_embeddings(
    model: torch.nn.Module,
    tokenised_inputs: dict
) -> torch.Tensor:
    """
    Computes the model embeddings from the tokenized inputs.
    """
\end{minted}
\caption{High level API on-top of Huggingface's tranformers library that can be used for generating vector embeddings using models available on Huggingface.}
\label{fig:python-apis-for-embeddings}
\end{listing}



\subsubsection{Mistral 7B v0.2 and Opensearch}


\subsubsection{Mistral 7B v0.2 with Opensearch and Postprocessing}


\subsubsection{Mistral 7B v0.2 with Opensearch and Heavy Postprocessing}


\section{The architecture of the software}


% https://lucid.app/lucidchart/f5dc4477-fa25-44e8-93d9-946def9cd4a9/edit

\begin{figure}[H]
    \centering
    \includegraphics[width=\textwidth]{content/figures/assets/06-system-architecture-diagram.pdf}
    \caption{Diagram that shows the system architecture of the software constructed to run the study}
    \label{fig:system_architecture_diagram}
\end{figure}



\subsection{Courseroom Crawler}


\subsection{Running large language models at scale}


\subsection{Datastore and Index}


\subsection{User interface}


\section{How the software is deployed}


% https://lucid.app/lucidchart/4fca3d7e-5905-40a3-8926-917ce384926b/edit

\begin{figure}[H]
    \centering
    \includegraphics[width=\textwidth]{content/figures/assets/07-aws-diagram.pdf}
    \caption{Diagram that shows how the software was deployed on Amazon AWS}
    \label{fig:aws_diagram}
\end{figure}



% \sweExpl{Hårdvara / Mjukvarudesign ... / modell / Simuleringsmodell och parametrar / …}




% \sweExpl{Figur~\ref{fig:homepageicon}  visar en enkel ikon för en hemsida. Tiden för att få tillgång till den här sidan när den laddas kommer att kvantifieras i en serie experiment. De konfigurationer som har testats i provbänk listas ini tabell~\ref{tab:configstested}.\\
% Vad du har gjort? Hur gjorde du det? Vilka designval gjorde du?}


\section{Implementation …/Modeling/Simulation/…}
\label{sec:implementationDetails}


\subsection{Some examples of coding}


% \engExpl{This section is simply to show some example of how you can include code in your thesis - this is not a section you would have in your thesis.}
% \sweExpl{Det här avsnittet är helt enkelt för att visa ett exempel på hur du kan inkludera kod i ditt examensarbete - det här är inte ett avsnitt du skulle ha i ditt examensarbete.}


\subsection{Some examples of figures in tikz}


% \engExpl{This section is simply to show some example of how you can draw your own figures for in your thesis - this is not a section you would have in your thesis.}
% \sweExpl{Det här avsnittet är helt enkelt för att visa ett exempel på hur du kan rita dina egna figurer i ditt examensarbete – det här är inte ett avsnitt du skulle ha i ditt examensarbete.}
% These figures are just some examples to show that you can draw your own figures for in your thesis. This has two advantages: \first you do not have to worry about copyrights -- as these are your own figures and \Second the text is now readable and not simply a picture of text -- so screen readers can read the figure's contents to someone who is listening to the contents of your thesis.


\subsubsection{Azure's Form Recognizer}


\cleardoublepage