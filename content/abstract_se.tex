% \warningExpl{Inside the following scontents environment, you cannot use a \textbackslash include{filename} as it will not end up in the for diva information. Additionally, you should not use a straight double quote character in the abstracts or keywords, use two single quote characters instead.}

\begin{scontents}[store-env=abstracts,print-env=true]
% \generalExpl{Enter your Swedish abstract or summary here!}
% \sweExpl{Alla avhandlingar vid KTH \textbf{måste ha} ett abstrakt på både \textit{engelska} och \textit{svenska}.\\
% Om du skriver din avhandling på svenska ska detta göras först (och placera det som det första abstraktet) - och du bör revidera det vid behov.}

% \engExpl{If you are writing your thesis in English, you can leave this until the draft version that goes to your opponent for the written opposition. In this way, you can provide the English and Swedish abstract/summary information that can be used in the announcement for your oral presentation.\\If you are writing your thesis in English, then this section can be a summary targeted at a more general reader. However, if you are writing your thesis in Swedish, then the reverse is true – your abstract should be for your target audience, while an English summary can be written targeted at a more general audience.\\This means that the English abstract and Swedish sammnfattning
% or Swedish abstract and English summary need not be literal translations of each other.}

% \warningExpl{Do not use the \textbackslash glspl\{\} command in an abstract that is not in English, as my programs do not know how to generate plurals in other languages. Instead, you will need to spell these terms out or give the proper plural form. In fact, it is a good idea not to use the glossary commands at all in an abstract/summary in a language other than the language used in the \texttt{acronyms.tex file} - since the glossary package does \textbf{not} support use of more than one language.}

% \engExpl{The abstract in the language used for the thesis should be the first abstract, while the Summary/Sammanfattning in the other language can follow}

\end{scontents}

Foobar

\subsection*{Nyckelord}

\begin{scontents}[store-env=keywords,print-env=true]
% SwedishKeywords were set earlier, hence we can use alternative 2
\InsertKeywords{swedish}
\end{scontents}
% \sweExpl{Nyckelord som beskriver innehållet i uppsatsen eller rapporten}
