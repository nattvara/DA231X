%%% Local Variables:
%%% mode: latex
%%% TeX-master: t
%%% End:
% The following command is used with glossaries-extra
\setabbreviationstyle[acronym]{long-short}
% The form of the entries in this file is \newacronym{label}{acronym}{phrase}
%                                      or \newacronym[options]{label}{acronym}{phrase}
% see "User Manual for glossaries.sty" for the  details about the options, one example is shown below
% note the specification of the long form plural in the line below
\newacronym[longplural={Debugging Information Entities}]{DIE}{DIE}{Debugging Information Entity}
%
% The following example also uses options
\newacronym[shortplural={OSes}, firstplural={operating systems (OSes)}]{OS}{OS}{operating system}

% note the use of a non-breaking dash in long text for the following acronym

\newacronym{KTH}{KTH}{KTH Royal Institute of Technology}

\newacronym{LMS}{LMS}{Learning Manegement System}
\newacronym{RAG}{RAG}{Retrieval Augmented Generation}
\newacronym{LLM}{LLM}{Large Language Models}
\newacronym{RNN}{RNN}{Recurrent Neural Network}
\newacronym{NLP}{NLP}{Natural Language Processing}
\newacronym{GPT}{GPT}{Generative Pre-trained Transformers}
\newacronym{GQA}{GQA}{Grouped-Query Attention}
\newacronym{SWA}{SWA}{Sliding window attention}
\newacronym{SMoE}{SMoE}{Sparse Mixture of Experts}
\newacronym{IR}{IR}{Information Retrieval}
\newacronym{TF-IDF}{TF-IDF}{Term Frequency-Inverse Document Frequency}
