
%%
%% forked from https://gits-15.sys.kth.se/giampi/kthlatex kthlatex-0.2rc4 on 2020-02-13
%% expanded upon by Gerald Q. Maguire Jr.
%% This template has been adapted by Anders Sjögren to the University
%% Engineering Program in Computer Science at KTH ICT. This adaptation was to
%% translation of English headings into Swedish as the addition of Swedish.
%% Many thanks to others who have provided constructive input regarding the template.

% Make it possible to conditionally depend on the TeX engine used
\RequirePackage{ifxetex}
\RequirePackage{ifluatex}
\newif\ifxeorlua
\ifxetex\xeorluatrue\fi
\ifluatex\xeorluatrue\fi

\ifxeorlua
% The following is to ensure that the PDF uses a recent version rather than the typical PDF 1-5
%  This same version of PDF should be set as an option for hyperef

\RequirePackage{expl3}
\ExplSyntaxOn
%pdf_version_gset:n{2.0}
%\pdf_version_gset:n{1.5}

%% Alternatively, if you have a LaTeX newer than June 2022, you can use the following. However, then you have to remove the pdfversion from hyperef. It also breaks hyperxmp. So perhaps it is too early to try using it!
%\DocumentMetadata
%{
%% testphase = phase-I, % tagging without paragraph tagging
% testphase = phase-II % tagging with paragraph tagging and other new stuff.
%pdfversion = 2.0 % pdfversion must be set here.
%}

% Optionally, you can set the uncompress flag to make it easier to examine the PDF
%\pdf_uncompress: % to check the pdf
\ExplSyntaxOff
\else
\RequirePackage{expl3}
\ExplSyntaxOn
%\pdf_version_gset:n{2.0}
\pdf_version_gset:n{1.5}
\ExplSyntaxOff
\fi


%% The template is designed to handle a thesis in English or Swedish
% set the default language to english or swedish by passing an option to the documentclass - this handles the inside title page
% To optimize for digital output (this changes the color palette add the option: digitaloutput
% To use \ifnomenclature add the option nomenclature
% To use bibtex or biblatex - include one of these as an option
\documentclass[nomenclature, english, bibtex]{kththesis}
%\documentclass[swedish, biblatex]{kththesis}
% if pdflatex \usepackage[utf8]{inputenc}

%% Conventions for todo notes:
% Informational
%% \generalExpl{Comments/directions/... in English}
\newcommand*{\generalExpl}[1]{\todo[inline]{#1}}

% Language-specific information (currently in English or Swedish)
\newcommand*{\engExpl}[1]{\todo[inline, backgroundcolor=kth-lightgreen40]{#1}} %% \engExpl{English descriptions about formatting}
\newcommand*{\sweExpl}[1]{\todo[inline, backgroundcolor=kth-lightblue40]{#1}}  %% % \sweExpl{Text på svenska}

% warnings
\newcommand*{\warningExpl}[1]{\todo[inline, backgroundcolor=kth-lightred40]{#1}} %% \warningExpl{warnings}

% Uncomment to hide specific comments, to hide **all** ToDos add `final` to
% document class
% \renewcommand\warningExpl[1]{}
% \renewcommand\generalExpl[1]{}
% \renewcommand\engExpl[1]{}
% For example uncommenting the following line hides the Swedish language explanations
% \renewcommand\sweExpl[1]{}


% \usepackage[style=numeric,sorting=none,backend=biber]{biblatex}
\ifbiblatex
    %\usepackage[language=english,bibstyle=authoryear,citestyle=authoryear, maxbibnames=99]{biblatex}
    % alternatively you might use another style, such as IEEE and use citestyle=numeric-comp  to put multiple citations in a single pair of square brackets
    \usepackage[style=ieee,citestyle=numeric-comp]{biblatex}
    \addbibresource{references.bib}
    \bibliographystyle{bibstyle/myIEEEtran}
    %\DeclareLanguageMapping{norsk}{norwegian}
\else
    % The line(s) below are for BibTeX
    \bibliographystyle{bibstyle/myIEEEtran}
    %\bibliographystyle{apalike}
\fi


% include a variety of packages that are useful
\input{lib/includes}
\input{lib/kthcolors}

%\glsdisablehyper
%\makeglossaries
%\makenoidxglossaries
%%%% Local Variables:
%%% mode: latex
%%% TeX-master: t
%%% End:
% The following command is used with glossaries-extra
\setabbreviationstyle[acronym]{long-short}
% The form of the entries in this file is \newacronym{label}{acronym}{phrase}
%                                      or \newacronym[options]{label}{acronym}{phrase}
% see "User Manual for glossaries.sty" for the  details about the options, one example is shown below
% note the specification of the long form plural in the line below
\newacronym[longplural={Debugging Information Entities}]{DIE}{DIE}{Debugging Information Entity}
%
% The following example also uses options
\newacronym[shortplural={OSes}, firstplural={operating systems (OSes)}]{OS}{OS}{operating system}

% note the use of a non-breaking dash in long text for the following acronym
\newacronym{IQL}{IQL}{Independent Q‑Learning}

% example of putting in a trademark on first expansion
\newacronym[first={NVIDIA OpenSHMEM Library (NVSHMEM\texttrademark)}]{NVSHMEM}{NVSHMEM}{NVIDIA OpenSHMEM Library}

\newacronym{KTH}{KTH}{KTH Royal Institute of Technology}

\newacronym{LMS}{LMS}{Learning Manegement System}
\newacronym{RAG}{RAG}{Retrieval Augmented Generation}
\newacronym{LLM}{LLM}{Large Language Models}
\newacronym{RNN}{RNN}{Recurrent Neural Network}
\newacronym{CNN}{CNN}{Convolutional Neural Networks}
\newacronym{LSTM}{LSTM}{Long Short-Term Memory}
\newacronym{GRU}{GRU}{Gated Recurrent Units}
\newacronym{BERT}{BERT}{Bidirectional Encoder Representations from Transformers}
\newacronym{GAN}{GAN}{Generative Adversarial Network}
\newacronym{NLP}{NLP}{Natural Language Processing}
\newacronym{GPT}{GPT}{Generative Pre-trained Transformers}
\newacronym{GQA}{GQA}{Grouped-Query Attention}
\newacronym{SWA}{SWA}{Sliding window attention}
\newacronym{SMoE}{SMoE}{Sparse Mixture of Experts}
\newacronym{IR}{IR}{Information Retrieval}
\newacronym{TF-IDF}{TF-IDF}{Term Frequency-Inverse Document Frequency}
\newacronym{CBOW}{CBOW}{Continuous Bag-of-Words}
\newacronym{MTEB}{MTEB}{Massive Text Embedding Benchmark}
\newacronym{seq2seq}{seq2seq}{Sequence-to-sequence}
\newacronym{TAM}{TAM}{Technology Acceptance Model}
\newacronym{ECM}{ECM}{Expectation-Confirmation Model}
\newacronym{RLHF}{RLHF}{Reinforcement learning from human feedback}
\newacronym{MMLU}{MMLU}{Massive Multitask Language Understanding}
\newacronym{GUI}{GUI}{Graphical user interface}
\newacronym{POC}{POC}{Proof-of-concept}
\newacronym{ORM}{ORM}{Object-relational mapping}
\newacronym{ECS}{ECS}{Amazon Elastic Container Service}
\newacronym{EC2}{EC2}{Amazon Elastic Compute Cloud}
\newacronym{ECR}{ECR}{Amazon Elastic Container Registry}
\newacronym{S3}{S3}{Amazon Simple Storage Service}
\newacronym{RDS}{RDS}{Amazon Relational Database Service}
\newacronym{SNS}{SNS}{Amazon Simple Notification Service}
\newacronym{SDG}{SDG}{Sustainable Development Goal}
\newacronym{PIQA}{PIQA}{Physical Interaction: Question Answering}
\newacronym{GSM8K}{GSM8K}{Grade School Math 8K}
\newacronym{MBPP}{MBPP}{Mostly Basic Programming Problems}
                %load the acronyms file

\input{lib/defines}  % load some additional definitions to make writing more consistent

% The following is needed in conjunction with generating the DiVA data with abstracts and keywords using the scontents package and a modified listings environment
%\usepackage{listings}   %  already included
\ExplSyntaxOn
\newcommand\typestoredx[2]{\expandafter\__scontents_typestored_internal:nn\expandafter{#1} {#2}}
\ExplSyntaxOff
\makeatletter
\let\verbatimsc\@undefined
\let\endverbatimsc\@undefined
\lst@AddToHook{Init}{\hyphenpenalty=50\relax}
\makeatother


\lstnewenvironment{verbatimsc}
    {
    \lstset{%
        basicstyle=\ttfamily\tiny,
        backgroundcolor=\color{white},
        %basicstyle=\tiny,
        %columns=fullflexible,
        columns=[l]fixed,
        language=[LaTeX]TeX,
        %numbers=left,
        %numberstyle=\tiny\color{gray},
        keywordstyle=\color{red},
        breaklines=true,                 % sets automatic line breaking
        breakatwhitespace=true,          % sets if automatic breaks should only happen at whitespace
        %keepspaces=false,
        breakindent=0em,
        %fancyvrb=true,
        frame=none,                     % turn off any box
        postbreak={}                    % turn off any hook arrow for continuation lines
    }
}{}

%% Add some more keywords to bring out the structure more
\lstdefinestyle{[LaTeX]TeX}{
morekeywords={begin, todo, textbf, textit, texttt}
}

%% definition of new command for bytefield package
\newcommand{\colorbitbox}[3]{%
	\rlap{\bitbox{#2}{\color{#1}\rule{\width}{\height}}}%
	\bitbox{#2}{#3}}




% define a left aligned table cell that is ragged right
\newcolumntype{L}[1]{>{\raggedright\let\newline\\\arraybackslash\hspace{0pt}}p{#1}}

% Because backref is not compatible with biblatex
\ifbiblatex
    \usepackage[plainpages=false]{hyperref}
\else
    \usepackage[
    backref=page,
    pagebackref=false,
    plainpages=false,
                            % PDF related options
    unicode=true,           % Unicode encoded PDF strings
    bookmarks=true,         % generate bookmarks in PDF files
    bookmarksopen=false,    % Do not automatically open the bookmarks in the PDF reading program
    pdfpagemode=UseNone,    % None, UseOutlines, UseThumbs, or FullScreen
    destlabel,              % better naming of destinations
    pdfencoding=auto,       % for unicode in
    ]{hyperref}
    \makeatletter
    \ltx@ifpackageloaded{attachfile2}{
    % cannot use backref if one is using attachfile
    }
    {\usepackage{backref}
    %
    % Customize list of backreferences.
    % From https://tex.stackexchange.com/a/183735/1340
    \renewcommand*{\backref}[1]{}
    \renewcommand*{\backrefalt}[4]{%
    \ifcase #1%
          \or [Page~#2.]%
          \else [Pages~#2.]%
    \fi%
    }
    }
    \makeatother

\fi
\usepackage[all]{hypcap}	%% prevents an issue related to hyperref and caption linking

%% Acronyms
% note that nonumberlist - removes the cross references to the pages where the acronym appears
% note that super will set the descriptions text aligned
% note that nomain - does not produce a main glossary, thus only acronyms will be in the glossary
% note that nopostdot - will prevent there being a period at the end of each entry
\usepackage[acronym, style=super, section=section, nonumberlist, nomain,
nopostdot]{glossaries}
\setlength{\glsdescwidth}{0.75\textwidth}
\usepackage[]{glossaries-extra}
\ifinswedish
    %\usepackage{glossaries-swedish}
\fi

%% For use with the README_notes
% Define a new type of glossary so that the acronyms defined in the README_notes document can be distinct from those in the thesis template
% the tlg, tld, and dn will be the file extensions used for this glossary
\newglossary[tlg]{readme}{tld}{tdn}{README acronyms}


\input{lib/includes-after-hyperref}

%\glsdisablehyper
\makeglossaries
%\makenoidxglossaries

% The following bit of ugliness is because of the problems PDFLaTeX has handling a non-breaking hyphen
% unless it is converted to UTF-8 encoding.
% If you do not use such characters in your acronyms, this could be simplified to just include the acronyms file.
\ifxeorlua
%%% Local Variables:
%%% mode: latex
%%% TeX-master: t
%%% End:
% The following command is used with glossaries-extra
\setabbreviationstyle[acronym]{long-short}
% The form of the entries in this file is \newacronym{label}{acronym}{phrase}
%                                      or \newacronym[options]{label}{acronym}{phrase}
% see "User Manual for glossaries.sty" for the  details about the options, one example is shown below
% note the specification of the long form plural in the line below
\newacronym[longplural={Debugging Information Entities}]{DIE}{DIE}{Debugging Information Entity}
%
% The following example also uses options
\newacronym[shortplural={OSes}, firstplural={operating systems (OSes)}]{OS}{OS}{operating system}

% note the use of a non-breaking dash in long text for the following acronym
\newacronym{IQL}{IQL}{Independent Q‑Learning}

% example of putting in a trademark on first expansion
\newacronym[first={NVIDIA OpenSHMEM Library (NVSHMEM\texttrademark)}]{NVSHMEM}{NVSHMEM}{NVIDIA OpenSHMEM Library}

\newacronym{KTH}{KTH}{KTH Royal Institute of Technology}

\newacronym{LMS}{LMS}{Learning Manegement System}
\newacronym{RAG}{RAG}{Retrieval Augmented Generation}
\newacronym{LLM}{LLM}{Large Language Models}
\newacronym{RNN}{RNN}{Recurrent Neural Network}
\newacronym{CNN}{CNN}{Convolutional Neural Networks}
\newacronym{LSTM}{LSTM}{Long Short-Term Memory}
\newacronym{GRU}{GRU}{Gated Recurrent Units}
\newacronym{BERT}{BERT}{Bidirectional Encoder Representations from Transformers}
\newacronym{GAN}{GAN}{Generative Adversarial Network}
\newacronym{NLP}{NLP}{Natural Language Processing}
\newacronym{GPT}{GPT}{Generative Pre-trained Transformers}
\newacronym{GQA}{GQA}{Grouped-Query Attention}
\newacronym{SWA}{SWA}{Sliding window attention}
\newacronym{SMoE}{SMoE}{Sparse Mixture of Experts}
\newacronym{IR}{IR}{Information Retrieval}
\newacronym{TF-IDF}{TF-IDF}{Term Frequency-Inverse Document Frequency}
\newacronym{CBOW}{CBOW}{Continuous Bag-of-Words}
\newacronym{MTEB}{MTEB}{Massive Text Embedding Benchmark}
\newacronym{seq2seq}{seq2seq}{Sequence-to-sequence}
\newacronym{TAM}{TAM}{Technology Acceptance Model}
\newacronym{ECM}{ECM}{Expectation-Confirmation Model}
\newacronym{RLHF}{RLHF}{Reinforcement learning from human feedback}
\newacronym{MMLU}{MMLU}{Massive Multitask Language Understanding}
\newacronym{GUI}{GUI}{Graphical user interface}
\newacronym{POC}{POC}{Proof-of-concept}
\newacronym{ORM}{ORM}{Object-relational mapping}
\newacronym{ECS}{ECS}{Amazon Elastic Container Service}
\newacronym{EC2}{EC2}{Amazon Elastic Compute Cloud}
\newacronym{ECR}{ECR}{Amazon Elastic Container Registry}
\newacronym{S3}{S3}{Amazon Simple Storage Service}
\newacronym{RDS}{RDS}{Amazon Relational Database Service}
\newacronym{SNS}{SNS}{Amazon Simple Notification Service}
\newacronym{SDG}{SDG}{Sustainable Development Goal}
\newacronym{PIQA}{PIQA}{Physical Interaction: Question Answering}
\newacronym{GSM8K}{GSM8K}{Grade School Math 8K}
\newacronym{MBPP}{MBPP}{Mostly Basic Programming Problems}
                %load the acronyms file
\else
%%% Local Variables:
%%% mode: latex
%%% TeX-master: t
%%% End:
% The following command is used with glossaries-extra
\setabbreviationstyle[acronym]{long-short}
% The form of the entries in this file is \newacronym{label}{acronym}{phrase}
%                                      or \newacronym[options]{label}{acronym}{phrase}
% see "User Manual for glossaries.sty" for the  details about the options, one example is shown below
% note the specification of the long form plural in the line below
\newacronym[longplural={Debugging Information Entities}]{DIE}{DIE}{Debugging Information Entity}
%
% The following example also uses options
\newacronym[shortplural={OSes}, firstplural={operating systems (OSes)}]{OS}{OS}{operating system}

% note the use of a non-breaking dash in long text for the following acronym

\newacronym{KTH}{KTH}{KTH Royal Institute of Technology}

\newacronym{LMS}{LMS}{Learning Manegement System}
\newacronym{RAG}{RAG}{Retrieval Augmented Generation}
\newacronym{LLM}{LLM}{Large Language Models}
\newacronym{RNN}{RNN}{Recurrent Neural Network}
\newacronym{NLP}{NLP}{Natural Language Processing}
\newacronym{GPT}{GPT}{Generative Pre-trained Transformers}
\newacronym{GQA}{GQA}{Grouped-Query Attention}
\newacronym{SWA}{SWA}{Sliding window attention}
\newacronym{SMoE}{SMoE}{Sparse Mixture of Experts}
\newacronym{IR}{IR}{Information Retrieval}
\newacronym{TF-IDF}{TF-IDF}{Term Frequency-Inverse Document Frequency}
\newacronym{CBOW}{CBOW}{Continuous Bag-of-Words}
\newacronym{MTEB}{MTEB}{Massive Text Embedding Benchmark}

\fi


% insert the configuration information with author(s), examiner, supervisor(s), ...
\input{custom_configuration}

\title{Evaluating retrieval and summarisation performance of AI-Assistants built with Large Language Models and RAG-techniques (Retrieval Augmented Generation) in the domain of a LMS (Learning Management System)}
\subtitle{A subtitle in the language of the thesis}

% give the alternative title - i.e., if the thesis is in English, then give a Swedish title

\alttitle{Detta är den svenska översättningen av titeln}
\altsubtitle{Detta är den svenska översättningen av undertiteln}

% alternative, if the thesis is in Swedish, then give an English title
%\alttitle{This is the English translation of the title}
%\altsubtitle{This is the English translation of the subtitle}

% Enter the English and Swedish keywords here for use in the PDF meta data _and_ for later use
% following the respective abstract.
% Try to put the words in the same order in both languages to facilitate matching. For example:

\EnglishKeywords{Canvas Learning Management System, Docker containers, Performance tuning}
\SwedishKeywords{Canvas Lärplattform, Dockerbehållare, Prestandajustering}



%%%%% For the oral presentation
%% Add this information once your examiner has scheduled your oral presentation
\presentationDateAndTimeISO{2022-03-15 13:00}
\presentationLanguage{eng}
\presentationRoom{via Zoom https://kth-se.zoom.us/j/ddddddddddd}
\presentationAddress{Isafjordsgatan 22 (Kistagången 16)}
\presentationCity{Stockholm}

% When there are multiple opponents, separate their names with '\&'
% Opponent's information
\opponentsNames{A. B. Normal \& A. X. E. Normalè}

% Once a thesis is approved by the examiner, add the TRITA number
% The TRITA number for a thesis consists of two parts a series (unique to each school)
% and the number in the series which is formatted as the year followed by a colon and
% then a unique series number for the thesis - starting with 1 each year.
\trita{TRITA-EECS-EX}{2023:0000}

% Put the title, author, and keyword information into the PDF meta information
\input{lib/pdf_related_includes}


% the custom colors and the commands are defined in defines.tex
\hypersetup{
	colorlinks  = true,
	breaklinks  = true,
	linkcolor   = \linkscolor,
	urlcolor    = \urlscolor,
	citecolor   = \refscolor,
	anchorcolor = black
}

\ifnomenclature
% The following lines make the page numbers and equations hyperlinks in the Nomenclature list
\renewcommand*{\pagedeclaration}[1]{\unskip, \dotfill\hyperlink{page.#1}{page\nobreakspace#1}}
% The following does not work correctly, as the name of the cross-reference is incorrect
%\renewcommand*{\eqdeclaration}[1]{, see equation\nobreakspace(\hyperlink{equation.#1}{#1})}

% You can also change the page heading for the nomenclature
\renewcommand{\nomname}{List of Symbols Used}

% You can even add customization text before the list
\renewcommand{\nompreamble}{The following symbols will be later used within the body of the thesis.}
\makenomenclature
\fi

%
% The commands below are to configure JSON listings
%
% format for JSON listings
\colorlet{punct}{red!60!black}
\definecolor{delim}{RGB}{20,105,176}
\definecolor{numb}{RGB}{106, 109, 32}
\definecolor{string}{RGB}{0, 0, 0}

\lstdefinelanguage{json}{
    numbers=none,
    numberstyle=\small,
    frame=none,
    rulecolor=\color{black},
    showspaces=false,
    showtabs=false,
    breaklines=true,
    postbreak=\raisebox{0ex}[0ex][0ex]{\ensuremath{\color{gray}\hookrightarrow\space}},
    breakatwhitespace=true,
    basicstyle=\ttfamily\small,
    extendedchars=false,
    upquote=true,
    morestring=[b]",
    stringstyle=\color{string},
    literate=
     *{0}{{{\color{numb}0}}}{1}
      {1}{{{\color{numb}1}}}{1}
      {2}{{{\color{numb}2}}}{1}
      {3}{{{\color{numb}3}}}{1}
      {4}{{{\color{numb}4}}}{1}
      {5}{{{\color{numb}5}}}{1}
      {6}{{{\color{numb}6}}}{1}
      {7}{{{\color{numb}7}}}{1}
      {8}{{{\color{numb}8}}}{1}
      {9}{{{\color{numb}9}}}{1}
      {:}{{{\color{punct}{:}}}}{1}
      {,}{{{\color{punct}{,}}}}{1}
      {\{}{{{\color{delim}{\{}}}}{1}
      {\}}{{{\color{delim}{\}}}}}{1}
      {[}{{{\color{delim}{[}}}}{1}
      {]}{{{\color{delim}{]}}}}{1}
      {’}{{\char13}}1,
}

\lstdefinelanguage{XML}
{
  basicstyle=\ttfamily\color{blue}\bfseries\small,
  morestring=[b]",
  morestring=[s]{>}{<},
  morecomment=[s]{<?}{?>},
  stringstyle=\color{black},
  identifierstyle=\color{blue},
  keywordstyle=\color{cyan},
  breaklines=true,
  postbreak=\raisebox{0ex}[0ex][0ex]{\ensuremath{\color{gray}\hookrightarrow\space}},
  breakatwhitespace=true,
  morekeywords={xmlns,version,type}% list your attributes here
}

% In case you use both listings and lstlistings - this makes them both use the same counter
\makeatletter
\AtBeginDocument{\let\c@listing\c@lstlisting}
\makeatother
\usepackage{subfiles}

% To have Creative Commons (CC) license and logos use the doclicense package
% Note that the lowercase version of the license has to be used in the modifier
% i.e., one of by, by-nc, by-nd, by-nc-nd, by-sa, by-nc-sa, zero.
% For background see:
% https://www.kb.se/samverkan-och-utveckling/oppen-tillgang-och-bibsamkonsortiet/open-access-and-bibsam-consortium/open-access/creative-commons-faq-for-researchers.html
% https://kib.ki.se/en/publish-analyse/publish-your-article-open-access/open-licence-your-publication-cc
\begin{comment}
\usepackage[
    type={CC},
    %modifier={by-nc-nd},
    %version={4.0},
    modifier={by-nc},
    imagemodifier={-eu-88x31},  % to get Euro symbol rather than Dollar sign
    hyphenation={RaggedRight},
    version={4.0},
    %modifier={zero},
    %version={1.0},
]{doclicense}
\end{comment}

\begin{document}
%\selectlanguage{swedish}
%
\selectlanguage{english}

%%% Set the numbering for the title page to a numbering series not in the preface or body
\pagenumbering{alph}
\kthcover
\clearpage\thispagestyle{empty}\mbox{} % empty back of front cover
\titlepage

% If you do not want to have a bookinfo page, comment out the line saying \bookinfopage and add a \cleardoublepage
% If you want a bookinfo page: you will get a copyright notice, unless you have used the doclicense package in which case you will get a Creative Commons license. To include the doclicense package, uncomment the configuration of this package above and configure it with your choice of license.
\bookinfopage

% Frontmatter includes the abstracts and table-of-contents
\frontmatter
\setcounter{page}{1}
\begin{abstract}
% The first abstract should be in the language of the thesis.
% Abstract fungerar på svenska också.
  \markboth{\abstractname}{}
\begin{scontents}[store-env=lang]
eng
\end{scontents}
%%% The contents of the abstract (between the begin and end of scontents) will be saved in LaTeX format
%%% and output on the page(s) at the end of the thesis with information for DiVA facilitating the correct
%%% entry of the meta data for your thesis.
%%% These page(s) will be removed before the thesis is inserted into DiVA.
% \engExpl{All theses at KTH are \textbf{required} to have an abstract in both \textit{English} and \textit{Swedish}.}


% \engExpl{Exchange students may want to include one or more abstracts in the language(s) used in their home institutions to avoid the need to write another thesis when returning to their home institution.}


% \generalExpl{Keep in mind that most of your potential readers are only going to read your \texttt{title} and \texttt{abstract}. This is why the abstract must give them enough information so that they can decide if this document is relevant to them or not. Otherwise, the likely default choice is to ignore the rest of your document.\\
% An abstract should stand on its own, i.e., no citations, cross-references to the body of the document, acronyms must be spelled out, \ldots .\\Write this early and revise as necessary. This will help keep you focused on what you are trying to do.}


\begin{scontents}[store-env=abstracts,print-env=true]


% \generalExpl{Enter your abstract here!}
% Write an abstract that is about 250 and 350 words (1/2 A4-page)  with the following components:
% key parts of the abstract
% \begin{itemize}
%   \item What is the topic area? (optional) Introduces the subject area for the project.
%   \item Short problem statement
%   \item Why was this problem worth a Bachelor's/Master’s thesis project? (\ie, why is the problem both significant and of a suitable degree of difficulty for a Bachelor's/Master’s thesis project? Why has no one else solved it yet?)
%   \item How did you solve the problem? What was your method/insight?
%   \item Results/Conclusions/Consequences/Impact: What are your key results/\linebreak[4]conclusions? What will others do based on your results? What can be done now that you have finished - that could not be done before your thesis project was completed?
% \end{itemize}


\end{scontents}


% \engExpl{The following are some notes about what can be included (in terms of LaTeX) in your abstract.}
% Choice of typeface with \textbackslash textit, \textbackslash textbf, and \textbackslash texttt:  \textit{x}, \textbf{x}, and \texttt{x}.


% Text superscripts and subscripts with \textbackslash textsubscript and \textbackslash textsuperscript: A\textsubscript{x} and A\textsuperscript{x}.


% Some symbols that you might find useful are available, such as: \textbackslash textregistered, \textbackslash texttrademark, and \textbackslash textcopyright. For example,
% the copyright symbol: \textbackslash textcopyright Maguire 2022 results in \textcopyright Maguire 2022. Additionally, here are some examples of text superscripts (which can be combined with some symbols): \textbackslash textsuperscript\{99m\}Tc, A\textbackslash textsuperscript\{*\}, A\textbackslash textsuperscript\{\textbackslash textregistered\}, and A\textbackslash texttrademark resulting in \textsuperscript{99m}Tc, A\textsuperscript{*}, A\textsuperscript{\textregistered}, and A\texttrademark. Two examples of subscripts are: H\textbackslash textsubscript\{2\}O and CO\textbackslash textsubscript\{2\} which produce  H\textsubscript{2}O and CO\textsubscript{2}.


% You can use simple environments with begin and end: itemize and enumerate and within these use instances of \textbackslash item.


% The following commands can be used: \textbackslash eg, \textbackslash Eg, \textbackslash ie, \textbackslash Ie, \textbackslash etc, and \textbackslash etal: \eg, \Eg, \ie, \Ie, \etc, and \etal.


% The following commands for numbering with lowercase Roman numerals: \textbackslash first, \textbackslash Second, \textbackslash third, \textbackslash fourth, \textbackslash fifth, \textbackslash sixth, \textbackslash seventh, and \textbackslash eighth: \first, \Second, \third, \fourth, \fifth, \sixth, \seventh, and \eighth. Note that the second case is set with a capital 'S' to avoid conflicts with the use of second of as a unit in the \texttt{siunitx} package.


% Equations using \textbackslash( xxxx \textbackslash) or \textbackslash[ xxxx \textbackslash] can be used in the abstract. For example: \( (C_5O_2H_8)_n \)
% or \[ \int_{a}^{b} x^2 \,dx \]
% Note that you \textbf{cannot} use an equation between dollar signs.


% Even LaTeX comments can be handled, for example: \% comment.
% Note that one can include percentages, such as: 51\% or \SI{51}{\percent}.


Foobar


\subsection*{Keywords}
\begin{scontents}[store-env=keywords,print-env=true]


% % If you set the EnglishKeywords earlier, you can retrieve them with:


\InsertKeywords{english}


% % If you did not set the EnglishKeywords earlier then simply enter the keywords here:


% % comma separate keywords, such as: Canvas Learning Management System, Docker containers, Performance tuning


\end{scontents}


% \engExpl{\textbf{Choosing good keywords can help others to locate your paper, thesis, dissertation, \ldots and related work.}}
% Choose the most specific keyword from those used in your domain, see for example: the ACM Computing Classification System ({\small \url{https://www.acm.org/publications/computing-classification-system/how-to-use})},
% the IEEE Taxonomy ({\small \url{https://www.ieee.org/publications/services/thesaurus-thank-you.html}}), PhySH (Physics Subject Headings)\linebreak[4] ({\small \url{https://physh.aps.org/}}), \ldots or keyword selection tools such as the  National Library of Medicine's Medical Subject Headings (MeSH)  ({\small \url{https://www.nlm.nih.gov/mesh/authors.html}}) or Google's Keyword Tool ({\small \url{https://keywordtool.io/}})\\


% \textbf{Formatting the keywords}:
% \begin{itemize}
%   \item The first letter of a keyword should be set with a capital letter and proper names should be capitalized as usual.
%   \item Spell out acronyms and abbreviations.
%   \item Avoid "stop words" - as they generally carry little or no information.
%   \item List your keywords separated by commas (",").
% \end{itemize}
% Since you should have both English and Swedish keywords - you might think of ordering them in corresponding order (\ie, so that the n\textsuperscript{th} word in each list correspond) - this makes it easier to mechanically find matching keywords.
\end{abstract}
\cleardoublepage
\babelpolyLangStart{swedish}
\begin{abstract}
    \markboth{\abstractname}{}
\begin{scontents}[store-env=lang]
swe
\end{scontents}

% \warningExpl{Inside the following scontents environment, you cannot use a \textbackslash include{filename} as it will not end up in the for diva information. Additionally, you should not use a straight double quote character in the abstracts or keywords, use two single quote characters instead.}


\begin{scontents}[store-env=abstracts,print-env=true]


% \generalExpl{Enter your Swedish abstract or summary here!}
% \sweExpl{Alla avhandlingar vid KTH \textbf{måste ha} ett abstrakt på både \textit{engelska} och \textit{svenska}.\\
% Om du skriver din avhandling på svenska ska detta göras först (och placera det som det första abstraktet) - och du bör revidera det vid behov.}


% \engExpl{If you are writing your thesis in English, you can leave this until the draft version that goes to your opponent for the written opposition. In this way, you can provide the English and Swedish abstract/summary information that can be used in the announcement for your oral presentation.\\If you are writing your thesis in English, then this section can be a summary targeted at a more general reader. However, if you are writing your thesis in Swedish, then the reverse is true – your abstract should be for your target audience, while an English summary can be written targeted at a more general audience.\\This means that the English abstract and Swedish sammnfattning
% or Swedish abstract and English summary need not be literal translations of each other.}


% \warningExpl{Do not use the \textbackslash glspl\{\} command in an abstract that is not in English, as my programs do not know how to generate plurals in other languages. Instead, you will need to spell these terms out or give the proper plural form. In fact, it is a good idea not to use the glossary commands at all in an abstract/summary in a language other than the language used in the \texttt{acronyms.tex file} - since the glossary package does \textbf{not} support use of more than one language.}


% \engExpl{The abstract in the language used for the thesis should be the first abstract, while the Summary/Sammanfattning in the other language can follow}


\end{scontents}


Språkmodeller (LLMs) har blivit mycket populära under de senaste åren. Deras intelligensnivå och användbarhet är tydlig. Men att integrera dem i riktiga produkter och tjänster kräver arbete. Det här examensarbetet utvärderar verktyg och teknologier som ofta används när man bygger applikationer med LLMs. Detta inkluderar Retrieval Augmented Generation (RAG), embedding funktioner med mera.


Syftet med forskningen i det här examensarbetet är att svara på vilka verktyg, modeller och tekniker som ger de bästa resultaten. Dessutom undersöker examensarbetet om AI-drivna applikationer kan byggas med verktyg som uteslutande använder öppna källkod. Detta görs genom att bygga en AI-assistent med tillgång till riktiga kursrum i Canvas på KTH. AI-assistenten används av riktiga studenter som slumpmässigt tilldelas en av de teknologier som studien omfattar. Kvantitativ och kvalitativ feedback samlas in från studenterna och analyseras i detta examensarbete.


Resultaten visar att vissa modeller föredras av studenterna. Studenter som fick använda GPT-4 av OpenAI rapporterade mer positiva svar gällande hastighet, tillförlitlighet och användbarhet än studenter som använde en mindre och öppen modell utvecklad av Mistral AI.


Examensarbetet drar slutsatsen att medan det är möjligt att bygga effektiva AI-drivna applikationer med öppna modeller och tekniker, är det för närvarande enklare att skapa smartare system med proprietära modeller. Deltagarna i studien var generellt positiva till AI-assistenten som utvecklades för studien, även om vissa uttryckte oro över integritetsfrågor. Ytterligare forskning behövs för att fullt ut förstå effektiviteten hos olika modeller och tekniker inom specialiserade domäner, som den som utforskas i detta arbete.


\subsection*{Nyckelord}


\begin{scontents}[store-env=keywords,print-env=true]
% SwedishKeywords were set earlier, hence we can use alternative 2
\InsertKeywords{swedish}
\end{scontents}
% \sweExpl{Nyckelord som beskriver innehållet i uppsatsen eller rapporten}

\end{abstract}
\babelpolyLangStop{swedish}

\cleardoublepage

\section*{Acknowledgments}
\markboth{Acknowledgments}{}
% \sweExpl{Författarnas tack}


% \engExpl{It is nice to acknowledge the people that have helped you. It is
%   also necessary to acknowledge any special permissions that you have gotten –
%   for example, getting permission from the copyright owner to reproduce a
%   figure. In this case, you should acknowledge them and this permission here
%   and in the figure’s caption. \\
%   Note: If you do \textbf{not} have the copyright owner’s permission, then you \textbf{cannot} use any copyrighted figures/tables/\ldots . Unless stated otherwise all figures/tables/\ldots are generally copyrighted.
% }
% \sweExpl{I detta kapitel kan du ev nämna något om
%   din bakgrund om det påverkar rapporten på något sätt. Har du t ex inte
%   möjlighet att skriva perfekt svenska för att du är nyanländ till landet kan
%   det vara på sin plats att nämna detta här. OBS, detta får dock inte vara en
%   ursäkt för att lämna in en rapport med undermåligt språk, undermålig grammatik och
%   stavning (t ex får fel som en automatisk stavningskontroll och
%   grammatikkontroll kan upptäcka inte förekomma)\\
% En dualism som måste hanteras i hela rapporten och projektet
% }


I would first like to thank my two supervisors on this thesis. First Fredrik Enoksson from the Unit of Digital Learning at KTH, for helping me conceptualise and build an assistant that could actually work in the digital environment at KTH. Additionally, helping me with getting in touch with several examiners who participated in the study, helping me with the logistics of getting access to the course room, and much more. I will miss our weekly catch-up that we’ve had throughout the project.


Secondly, I would like to thank my supervisor on the research in the thesis, Michael Welle from the Division of Robotics, Perception and Learning. Michael has been a great support helping figure out what research I should do within my project. I appreciate that you’ve kept me on track throughout the project, without your guidance I’m sure I would’ve still been building on those early prototypes.


I would like to thank all the participating teachers and TAs. Without the courses who voluntarily allowed the assistant to be deployed in their course rooms this research could not have been completed. These include Marcus Lithander, Malin Jansson, Dena Hussain, Christian Smith, Richard James Glassey,  Patric Jensfelt. An extra big thanks to Antonio Maffei and Fabio Marco Monetti who not only participated but also helped in constructing the study.


Without the funding provided from KTH Innovation the research in this thesis would not have been possible. The grants they provided me last year for kthGPT helped me with that and this thesis. Thanks to Hannes Eder Öhrström for all advice and setting me up with the funding last year.


I would also like to thank Sebastian Ware for inspiring me to pursue this subject for my Master thesis. I’ve learned a lot about LLMs and RAG. I’m not sure I would’ve pursued this thesis unless you advised me to do so, thanks!


The support I’ve had from my friends through all of my five years at KTH have been paramount. I’m confident that I wouldn't have made it without them. Even though we’ve been on our own with the master thesis, and many courses before that, the daily support and camaraderie have made these years some of the best years in my life.


Lastly, I would like to thank my employer, Stockholm Code Group, which has supported me in pursuing my degree. They hired me 6 months before enrolling at KTH, knowing that I would become a fulltime student, not many firms would do so. I’ve spent countless hours studying in our offices, and that office, along with the people in it, have been just as significant in helping me finish this degree. Thanks for supporting my side-hustle as a full time student. A special thanks to Patrick Zeits, who convinced me to pursue an education in the first place. I’m grateful you showed me \textit{why} higher education was useful.


\acknowlegmentssignature

\fancypagestyle{plain}{}
\renewcommand{\chaptermark}[1]{ \markboth{#1}{}}
\tableofcontents
  \markboth{\contentsname}{}

\cleardoublepage
\listoffigures

\cleardoublepage

\listoftables
\cleardoublepage
% \lstlistoflistings\engExpl{If you have listings in your thesis. If not, then remove this preface page.}
\cleardoublepage
% Align the text expansion of the glossary entries
\newglossarystyle{mylong}{%
  \setglossarystyle{long}%
  \renewenvironment{theglossary}%
     {\begin{longtable}[l]{@{}p{\dimexpr 2cm-\tabcolsep}p{0.8\hsize}}}% <-- change the value here
     {\end{longtable}}%
 }
%\glsaddall
%\printglossaries[type=\acronymtype, title={List of acronyms}]

\printglossary[style=mylong, type=\acronymtype, title={List of acronyms and abbreviations}]
%\printglossary[type=\acronymtype, title={List of acronyms and abbreviations}]

%\printnoidxglossary[style=mylong, title={List of acronyms and abbreviations}]
% \engExpl{The list of acronyms and abbreviations should be in alphabetical order based on the spelling of the acronym or abbreviation.}

% if the nomenclature option was specified, then include the nomenclature page(s)
\ifnomenclature
    \cleardoublepage
    % Output the nomenclature list
    \printnomenclature
\fi

%% The following label is essential to know the page number of the last page of the preface
%% It is used to compute the data for the "For DIVA" pages
\label{pg:lastPageofPreface}
% Mainmatter is where the actual contents of the thesis goes
\mainmatter
\glsresetall
\renewcommand{\chaptermark}[1]{\markboth{#1}{}}
\selectlanguage{english}

\chapter{Introduction}
\label{ch:introduction}


% \sweExpl{svensk: Introduktion}
% \sweExpl{Ofta kommer problemet och problemägaren från industrin där man önskar en specifik lösning på ett specifikt problem. Detta är ofta ”för smalt” definierat och ger ofta en ”för smal” lösning för att resultatet skall vara intressant ur ett mer allmänt ingenjörsperspektiv och med ”nya” erfarenheter som resultat. Fundera tillsammans med projektets intressenter (student, problemägare och akademi) hur man skulle kunna använda det aktuella problemet/förslaget för att undersöka någon ingenjörsaspekt och vars resultat kan ge ny eller kompletterande erfarenhet till ingenjörssamfundet och vetenskapen.\\slöser man en del eller hela delen av det ursprungliga problemet.\\Erfarenheten kommer ur en frågeställning som man i examensarbetet försöker besvara med tidigare och andras erfarenhet, egna eller modifierade metoder som ger ett resultat vilket kan användas för att diskutera ett svar på undersökningsfrågan.\\Detta stycke skall alltså, förutom det ursprungliga ”smala” problemet, innehålla  vad som skall undersökas för att skapa ny ingenjörserfarenhet och/eller vetenskap.}


% \engExpl{The first paragraph after a heading is not indented, all of the
%   subsequent paragraphs have their first line indented.}


% This chapter describes the specific problem that this thesis addresses, the context of the problem, the
% goals of this thesis project, and outlines the structure of the thesis.\\


% \generalExpl{Give a general introduction to the area. (Remember to use appropriate references in this and all other sections.)}


% One can use either biblatex or bibtex - set as the option for the document at the top of this file
% \ifbiblatex
% \engExpl{We use the \emph{biblatex} package to handle our references.  We
% use the command \texttt{parencite} to get a reference in parenthesis, like
% this \textbackslash parencite\{heisenberg2015\} resulting in \parencite{heisenberg2015}.  It is also possible to include the author as part of the sentence using \texttt{textcite}, like talking about the work of \textbackslash textcite\{einstein2016\} resulting in \textcite{einstein2016}.\\
% This also means that you have to change the include files to include biblatex and change the way that the \texttt{reference.bib} file is included.}
% \else
% \engExpl{We use the \emph{bibtex} package to handle our references.  We, therefore,
% use the command \textbackslash cite\{farshin\_make\_2019\}. For example, Farshin, \etal described how to improve LLC
% cache performance in \cite{farshin_make_2019} in the context of links running
% at \qty{200}{Gbps}.}
% \fi


% \engExpl{Use the glossaries package to help yourself and your readers.
% Add the acronyms and abbreviations to lib/acronyms.tex. Some examples are shown below:}
% In this thesis, we will examine the use of \glspl{LAN}. In this thesis, we will
% assume that \glspl{LAN} include \glspl{WLAN}, such as \gls{WiFi}.




\section{Background}
\label{sec:background}


This degree project will investigate \gls{LLM} and \gls{RAG} systems in the form of deploying an AI-assistant in Canvas online course rooms at KTH. The degree project will investigate how to evaluate these systems in very specialised domains and benchmark various models, approaches and techniques.

This research is important because \gls{LLM}s have gained widespread attention and we are likely to see large-scale adoption of these models into various applications. Understanding how to benchmark and evaluate these systems in specialised domains will be crucial to understand how to build these systems, which techniques to use, and which models work well. One such technique is \gls{RAG}, which is a way of providing specialised knowledge to a model. Understanding how to do that well is crucial for understanding how to build an \gls{LLM}-based application well.


Many organisations need to, due to commercial and regulatory compliance, host all AI-models themselves. This aspect is also interesting to evaluate, i.e. how well open source and commercially licensed models compare against the closed source models, such as GPT-4 by OpenAI.


% \sweExpl{svensk: Bakgrund}
% \generalExpl{Present the background for the area. Set the context for your project – so that your reader can understand both your project and this thesis. (Give detailed background information in Chapter 2 - together with related work.)
% Sometimes it is useful to insert a system diagram here so that the reader
% knows what are the different elements and their relationship to each
% other. This also introduces the names/terms/… that you are going to use
% throughout your thesis (be consistent). This figure will also help you later
% delimit what you are going to do and what others have done or will do.}


The research will be carried out within the e-learning management object at KTH, who are responsible for the digital learning environment at KTH. The object consists of two teams at the KTH IT department and one team at the digital learning unit at the ITM-school. The university hosts thousands of courses with domain specific information, such as 
assignments, lectures and schedules, that aren’t part of the public domain and therefore not part of the training set of LLMs.


All the work done by KTH IT aims to improve the operations at the university. Among this is reducing the administrative burden undertaken by teachers and teaching assistants (TAs). KTH IT wants to investigate if AI-assistants can be deployed into the canvas course rooms to reduce the workload of teachers and TAs which would help them focus on teaching, helping students and improve the quality of the education. KTH IT wants to see if it’s feasible to deploy an AI assistant into the canvas course rooms.


\section{Problem}
\label{sec:problem}


% \sweExpl{svensk: Problemdefinition eller Frågeställning\\
% Lyft fram det ursprungliga problemet om det finns något och definiera därefter
% den ingenjörsmässiga erfarenheten eller/och vetenskapen som kan komma ur
% projektet. }


\gls{LLM}s have gained widespread use since its popularisation by ChatGPT. Their abilities to summarise large bodies of text and follow user instructions have proven very useful in many contexts. However, considering their limited context window (and drawbacks of models with larger context window \cite{liu_lost_2024}) deploying useful applications with a chat based interface still rely upon integrating a \gls{RAG} system, introduced by Lewis et al. \cite{lewis_retrieval-augmented_2020}. These can retrieve relevant information needed to answer a user's query from outside data sources and inject them into the conversation.


Some announced but currently unreleased models, such as the gemini family of models \cite{gemini_team_gemini_2024}, have been reported to show great recall performance and reasoning abilities over millions of tokens. This could significantly reduce the importance of \gls{RAG} systems in applications which utilise \gls{LLM}s and external datasets to create intelligent systems with domain specific knowledge. However, even though no exact figures are presented by the Gemini team, inference speed (the time taken to produce a response to a prompt) seems to be significantly slower than with shorter contexts. This would again highlight the importance of efficient \gls{RAG} systems. Still, other approaches than traditional GPUs have been shown recently \cite{abts_software-defined_2022} by the Groq team to greatly increase inference speed.


Evaluating large language models is notoriously difficult. There are objective and automated metrics that can be used for tasks such as evaluating a model's summarisation capabilities, as shown by Basyal and Sanghvi \cite{basyal_text_2023}. However, for more complicated evaluations it gets trickier. In their seminal instructGPT paper Ouyang et al. \cite{ouyang_training_2022-3} at OpenAI try to evaluate \textit{“how well a model can follow instructions”} which is a very subjective question. They essentially relied upon human labellers to judge the overall quality of each response generated by the model.


In their Gemini-paper the Gemini team discuss the benchmarks used for their largest model. The team states that benchmarks are often designed to test shorter prompts whereas their longer prompts challenge tests used in traditional evaluation methods that rely heavily on manual evaluation. This highlights the relevance of good evaluation metrics. Regardless of context size or inference speed, evaluation of models tends to be very general. Which makes sense, when considering their general application.


When releasing their Mixtral model \cite{jiang_mixtral_2024} the Mistral AI team used a range of benchmark tests, such as \gls{MMLU}, \gls{PIQA}, \gls{GSM8K} etc. \gls{MMLU} \cite{hendrycks_measuring_2020} benchmarks a \gls{LLM}s proficiency in understanding and reasoning across various subjects such as humanities, STEM, and professional and everyday knowledge, by evaluating its performance on 57 tasks, to test its ability to generalise and apply knowledge. \gls{PIQA} \cite{bisk_piqa_2020}, evaluates a language model's understanding of physical commonsense by asking them to predict the outcome of physical interactions in various scenarios through multiple-choice questions. \gls{GSM8K} \cite{cobbe_training_2021} tests the ability to solve elementary-level mathematics word problems.


Evaluation of how well \gls{LLM}s perform is an open research question. As shown above \gls{LLM} developers often utilise multiple testsuites. These are oftentimes, very general tests. When implementing \gls{LLM}s in practical applications good performance often relies upon very good raw summarisation performance and reasoning abilities. Since the domain specific knowledge is provided to the model, raw built-in knowledge isn’t crucial. It is more important for the model to learn the task at hand using very few examples and within the given domain understand the question being asked by a user.
Further, as argued by by Siriwardhana et al., the training data of \gls{LLM}s include the knowledge of datasets such as Wikipedia \cite{siriwardhana_improving_2023} which means that evaluation methods in very specialised domains hold higher value than generalised domains. These brand new domains, that with certainty haven't been seen during training, tests the models zero-shot, and depending on the implementation, few-shot learning abilities.


\subsection{Research question}
\label{sec:researchQuestion}


The research question for this project is \textit{Which language model and which retrieval techniques do students prefer using?} and \textit{Is it possible to deploy an AI-assistant using a completely open source toolchain?}.


I believe the answer to the first question is that the closed source alternatives will be preferred by the students, however, I think the results will show it is possible to deploy an open source based AI assistant too.


\subsection{Original problem and definition}


% \sweExpl{Ursprungligt problem och definition}
% Some text


% \subsection{Scientific and engineering issues}\sweExpl{Vetenskaplig och ingenjörsmässig frågeställning}
% some text


The core challenge addressed in this thesis is the effective deployment and evaluation of AI-assistants powered by \gls{LLM} and \gls{RAG} techniques in a specialised domain, specifically within the \gls{LMS} of Canvas course rooms at KTH. This involves assessing the practicality and efficiency of integrating AI-Assistants built upon LLMs and RAG techniques into the educational settings to aid in reducing administrative burdens on educators and enhancing student interaction with course materials.


The original problem stems from the need to understand whether AI-assistants can effectively handle the domain-specific data intrinsic to educational platforms that are not included in their initial training datasets. Furthermore, the project aims to compare the efficacy and acceptability of open-source versus proprietary AI models in real-world educational applications.


\section{Purpose}


% \sweExpl{Syfte}
% \sweExpl{Skilj på syfte och mål! Syfte är att förändra något till det bättre. I examensarbetet finns ofta två aspekter på detta. Dels vill problemägaren (företaget) få sitt problem löst till det bättre men akademin och ingenjörssamfundet vill också få nya erfarenheter och vetskap. Beskriv ett syfte som tillfredställer båda dessa aspekter.\\
% Det finns även ett syfte till som kan vara värt att beakta och det är att du som student skall ta examen och att du måste bevisa, i ditt examensarbete, att du uppfyller examensmålen. Dessa mål sammanfaller med kursmålen för examensarbetskursen.
% }
% \generalExpl{State the purpose of your thesis and the purpose of your degree project.\\
% Describe who benefits and how they benefit if you achieve your goals. Include anticipated ethical, sustainability, social issues, etc. related to your project. (Return to these in your reflections in Section~\ref{sec:reflections}.)}


The purpose of this thesis is two-fold: firstly, to innovate within the educational technology space by integrating AI-assistants to potentially reduce workload and improve informational access within Canvas course rooms. Secondly, the thesis aims to contribute to academic knowledge by providing empirical data on the performance of these AI systems in a controlled educational setting. The dual purpose of this thesis ensures it not only investigates the immediate needs of KTH's digital learning environment but also enriches the scientific community’s understanding of applied AI within a specialised domain, such as education.


This research is intended to benefit educational institutions by potentially offering a tool that improves operational efficiency and students by providing an alternative, possibly more effective way of interacting with course content. In addition the research will bring benefits for researchers within AI and education. Ethically, the study focuses on the sustainable development of AI technologies by emphasising open-source solutions, aiming to democratise advanced technological developments and reduce reliance on proprietary models.




\section{Goals}
\label{sec:goals}


% \sweExpl{Mål}
% \sweExpl{Skilj på syfte och mål. Syftet är att åstakomma en förändring i något. Målen är vad som konkret skall göras för att om möjligt uppnå den önskade förändringen (syfte). }


% \generalExpl{State the goal/goals of this degree project.}


%The goal of this project is XXX. This has been divided into the following three sub-goals:
%\begin{enumerate}
%\item Subgoal 1 % \sweExpl{för att tillfredsställa problemägaren – industrin?}
%\item Subgoal 2 % \sweExpl{för att tillfredsställa ingenjörssamfundet och vetenskapen – akademin) }
%\item Subgoal 3 % \sweExpl{eventuellt, för att uppfylla kursmålen – du som student}
%\end{enumerate}


\begin{enumerate}
        \item \textbf{Technological Efficacy:} To evaluate the accuracy, speed, and reliability of responses by AI-assistants utilising both proprietary and open-source models in handling domain-specific content, such as the course rooms in canvas.


        \item \textbf{User Preference:} To understand the preferences of students regarding the usability, information quality, and overall experience of interacting with an AI-assistant built upon different models and retrieval techniques.


        \item \textbf{Operational Feasibility:} To assess the feasibility of integrating an AI-assistant built on fully open-source technologies within an academic setting, considering logistical, technical, and regulatory constraints.


        \item \textbf{Educational Impact:} To explore the potential of AI-assistants to reduce administrative burdens on educators and improve information accessibility for students.


        \item \textbf{Comparative Analysis:} To perform a comparative study between various \gls{LLM} models and \gls{RAG}-techniques.
\end{enumerate}




% \generalExpl{In addition to presenting the goal(s), you might also state what the deliverables and results of the project are.}


\section{Research Methodology}
\label{sec:research_methodology}


% \sweExpl{Undersökningsmetod}
% \sweExpl{Här anger du vilken vilken övergripande undersökningsstrategi eller metod du skall använda för att försöka besvara den akademiska frågeställning och samtidigt lösa det e v ursprungliga problemet. Ofta kan man använda ”lösandet av ursprungsproblemet” som en fallstudie kring en akademisk frågeställning. Du undersöker någon intressant fråga i ”skarpt” läge och samlar resultat och erfarenhet ur detta.\\
% Tänk på att företaget ibland måste stå tillbaka i sin önskan och förväntan på projektets resultat till förmån för ny eller kompletterande ingenjörserfarenhet och vetenskap (ditt examensarbete). Det är du som student som bestämmer och löser fördelningen mellan dessa två intressen men se till att alla är informerade. }
% \generalExpl{Introduce your choice of methodology/methodologies and method/methods – and the reason why you chose them. Contrast them with and explain why you did not choose other methodologies or methods. (The details of the actual methodology and method you have chosen will be given in Chapter~\ref{ch:methods}. Note that in Chapter~\ref{ch:methods}, the focus could be research strategies, data collection, data analysis, and quality assurance.)\\
% In this section you should present your philosophical assumption(s), research method(s), and research approach(es).}


This project primarily employs an empirical study based on data collection and quantitative analysis of responses collected from students using the software designed for this study. There are some forms distributed that will be used to collect insights using a qualitative approach. Both of these will be used to evaluate the implementation of AI-assistants in the educational domain.


\subsection{Experimental Design and Implementation}


The following section will outline the considerations the experiments in this thesis must consider. This includes both requirements for the software design and architecture, and the questions to distribute in the course.


\begin{description}
        \item[Model selection] Different models, including proprietary and open-source, with different sizes (number of active parameters), will be tested. The relevant models will be included in the study for testing.


        \item[RAG technique selection] Various configurations of \gls{RAG} systems will be tested to identify the most effective method for enhancing the AI's responses with respect to the layout of the data in Canvas course rooms. The relevant techniques will be included in the study for testing.


\item[Implement AI Assistant] Design and implement the system that will be used in the study. This includes application logic, user interface design, etc.


\item[Construct study questions] Craft the questions that will be asked to students and implement them in the AI assistant.


\item[Courses to include in the study] Find willing course administrators that want to participate in the study with their students.


\end{description}


\subsection{Evaluation Design}


The evaluation of the experiments is key to ensure the study is methodologically sound. The experimental setup is described in detail in \autoref{ch:methods}. On a high-level the experiments must consider the following to ensure its setup is robust and meaningful.


\begin{description}
        \item[Study Participants] The study will involve students using the AI-assistant and ask them to provide feedback on their experiences. This is a good selection of participants, considering the goals of the research.


\item[Experimental Setup] Controlled experiments will be conducted where participants use different configurations of the AI-assistant for typical student questions. These configurations are randomly assigned.


\item[Data Collection Methods] Data will be collected through integrated survey questions within the chat interface, capturing real-time feedback on the AI-assistant’s performance and student satisfaction.
\end{description}


\subsection{Analysis Techniques}


The analysis techniques are similarly described in greater detail in \autoref{ch:methods}. However, briefly its method can be broken down into two categories. These are;


\begin{description}
        \item[Quantitative Analysis] The data collected from the systems performance, and the responses collected through the multiple choice questions, asked to the participating students, will be collated into tables and charts. Using statistical methods this thesis will present results on which models, tools and techniques lead to the highest student satisfaction.
        \item[Qualitative Analysis] Feedback and open-ended responses will be analysed textually to understand user perceptions and contextual effectiveness of the AI-assistant.
\end{description}


This methodology was chosen for its ability to provide a comprehensive evaluation of both the technical capabilities and the practical usability of AI-assistants, offering insights into their potential benefits and limitations in the specific context for this study.




\section{Delimitations}
\label{sec:delimitations}


% \generalExpl{Describe the boundary/limits of your thesis project and what you are explicitly not going to do. This will help you bound your efforts – as you have clearly defined what is out of the scope of this thesis project. Explain the delimitations. These are all the things that could affect the study if they were examined and included in the degree project.}


This project has several delimitations that define the scope and boundaries of the research to ensure a focused and manageable study. The key delimitations are;


\begin{itemize}
        \item \textbf{Model Scope:} The project will not involve the development of new models or the fine-tuning of existing models. This includes \gls{LLM} and embedding functions. The study will utilise pre-trained models offered by bigger vendors or the open source community.
        \item \textbf{Data Limitations:} Only existing courses within KTH's Canvas \gls{LMS} will be utilised for the study. No new course content will be created, and no modifications will be made to existing course materials beyond what is necessary for the integration and testing of the AI-assistants.
        \item \textbf{Course Data Access:} The project will not use Canvas APIs for data integration. All interactions with the Canvas platform will be through existing interfaces, or data will be scraped and used from the Canvas web interface.
        \item \textbf{Geographic and Cultural Constraints:} The study is limited to the KTH environment, which may not represent other educational settings in different cultural or geographic contexts or languages. The findings might not be directly transferable to other institutions or countries without additional localisation and adaptation.
\end{itemize}


% \section{Structure of the thesis}


% \sweExpl{ Rapportens disposition}


% Chapter~\ref{ch:background} presents relevant background information about xxx.  % Chapter~\ref{ch:methods} presents the methodology and method used to solve the problem.  % …


\cleardoublepage

\chapter{Background}
\label{ch:background}


% \sweExpl{Bakgrund}
% \generalExpl{When you do your literature study, you should have a nearly complete Chapters 1 and 2.\\
% You may also find it convenient to introduce the future work section into your report early – so that you can put things that you think about but decide not to do now into this section.\\
% Note that later you can move things between this future work section and what you have done as you may change your mind about what to do now versus what to put off to future work.
% }
% \generalExpl{What does a reader (another x student -- where x is your study line) need to know to understand your report?
% What have others already done? (This is the “related work”.) Explain what and
% how prior work/prior research will be applied on or used in the degree
% project/work (described in this thesis). Explain why and what is not used in
% the degree project and give valid reasons for rejecting the work/research.}


This chapter provides the necessary background for understanding the research conducted within this thesis. This chapter also showcase the related work for this thesis and how the research relates to it.


% \sweExpl{Vilken viktig litteratur och
% (forsknings-)artiklar har du studerat inom området (litteraturstudie)? }


\section{Neural Networks}


Neural network models are a type of models within the broader field of machine learning whose design have been inspired by human brains. These models allow computers to recognise patterns and solve complex problems. The backpropagation algorithm was popularised by Rumelhart, Hinton, and Williams \cite{rumelhart_learning_1986}. This algorithm efficiently computes the gradient of the loss function with respect to the weights of the network by propagating the error back from the output layer to the input layer. This method is critical to understand all machine learning pipelines because it enables the network to adjust its weights in a way that minimises the error, thereby improving the model's predictions over time.


Building on backpropagation, Yann LeCun et al. \cite{lecun_gradient-based_1998} introduced Convolutional Neural Networks (CNNs) in 1998. These are a specialised kind of neural network for processing data, such as images, which can be converted to a matrix. CNNs utilise layers with convolving filters that apply the learned weights across subsections of the input data. This reduces the amount of parameters in the network and improves its efficiency.


These are two steps in the evolution of neural network models, particularly the developments in CNNs and other deep learning technologies, are central for setting the stage for even more complex architectures aimed at processing not just visual data, but sequential data such as text. This will eventually lead to Large Language Models (\gls{LLM}), which leverage deep learning techniques to understand and \textit{generate} human language. LLMs are built upon the principles of neural networks. Understanding the models we commonly refer to as LLMs involves understanding models such as Transformer models, BERT, and other encoder-decoder networks.


\subsection{Recurrent Neural Networks (RNNs)}


A Recurrent Neural Network (\gls{RNN}) is a type of neural network that is good for modelling sequential data. They are significantly different from other neural networks in their ability to maintain memory of previous inputs using an internal state. This state which is maintained inside the network while it’s running, will influence the network’s output. RNNs proved to be fundamental in tasks where context was crucial, such as language modelling and generation of text.


In an RNN, each neuron, its most basic building block, processes a part of the sequence, receiving both the current input \(x_t\) and the output from the previous step \(h_{t-1}\), this is known as the "hidden state". The core of an RNN operation involves updating this hidden state using:
\[
h_t = \text{tanh}(W_{hh} h_{t-1} + W_{xh} x_t + b)
\]
where \(W_{hh}\) and \(W_{xh}\) are the weights for the hidden state and input, respectively, and \(b\) is a bias. The updated state \(h_t\) is used in the next step to generate the output \(y_t\) via:
\[
y_t = W_{hy} h_t + b_y
\]




However, RNNs often struggle with maintaining a longer context due to problems like vanishing and exploding gradients, as written by Hochreiter and Schmidhuber \cite{hochreiter_long_1997}. This was a problem other RNN models tried to mitigate as it significantly reduce their usefulness in various tasks. The vanishing gradient problem makes it difficult for the RNN to learn connections between events that occur at longer distances in the input sequence because the gradient of the loss function decays exponentially with the length of the input sequence.


This led to the development of more sophisticated variants like Long Short-Term Memory (LSTM) networks and Gated Recurrent Units (GRUs) were developed. LSTMs \cite{hochreiter_long_1997}, use input, output, and "forget gates" to manage information flow, which allows them to maintain stable gradients. GRUs, which was proposed by Cho et al. \cite{cho_learning_2014}, simplifies this by merging the gates and states, reducing complexity while preserving performance across various tasks.


\subsection{Sequence-to-Sequence Models}


Sequence-to-sequence (seq2seq) models are designed to process sequences of data, such as text or speech, and generate corresponding output sequences. Sutskever et al. \cite{sutskever_sequence_2014} were the first to introduce these models which typically consist of two main components: an encoder and a decoder. The encoder will process the input and convert it into a dense vector. This vector encodes the entire input sequence which is then passed to the decoder, which generates the output. This architecture proved very useful in certain tasks such as translating text between languages. Bahdanau, Cho, and Bengio built upon this concept with attention mechanisms \cite{bahdanau_neural_2016} which would allow the decoder to focus on a specific piece of the input for small parts of the output, which improved the models ability to focus on longer sequences.


\subsection{Transformer Models}


The Transformer model, introduced by Vaswani et al. \cite{vaswani_attention_2023}, was a new approach for sequence-to-sequence networks, with a self-attention mechanism which was different from the recurrent design of RNNs. The new transformer architecture introduced by Vaswani et al. allowed the network to weigh the importance of different tokens in the input data irrespective of their sequential position. Where a token is a sequence of characters that can be treated as a single logical entity in the input and output sequence.


The key innovation of the Transformer is its ability to handle dependencies between single tokens or sequences of tokens at long distances from each other. This makes the transformer architecture especially good at understanding context in text data.


The introduction of the transformer model was foundational in the field, and today most models use this architecture, see section~\ref{sec:openai_models} and ~\ref{sec:mistral_models}.


\subsection{BERT and Advances in Encoder-Decoder Models}
\label{sec:bert_and_encoder_decoder}


Bidirectional Encoder Representations from Transformers also known as \textit{BERT} was introduced by Devlin et al. \cite{devlin_bert_2019} in 2018 and was a major improvement within natural language processing. The BERT model optimised token representations bidirectionally which means that it was refining the understanding of each token by looking at the tokens before and after each token. BERT was built on the transformer model’s encoder which allowed for pre-training on large text corpora, followed by fine-tuning for various tasks such as sentiment analysis and question answering.


Encoder-decoder models are important in machine learning for tasks that involve converting one sequence into another, such as machine translation or speech-to-text. In this type of model the encoder processes the input sequence and compresses information into what’s known as a context vector, this is a condensed representation of the input data. The decoder takes this context vector and generates an output sequence token by token. Each of these two components may be built using recurrent networks, convolutional networks, or more commonly nowadays, transformer architectures.


In contrast to traditional encoder-decoder models, encoder-only models, such as BERT, focus on generating an output based on an input without the need for a decoder. These models are typically used for tasks that require deep understanding of language context like sentence classification.


Decoder-only models, like the Generative Pre-trained Transformer (see section~\ref{sec:openai_models}), focus on generating sequences from a given context or starting point. These models are very good in situations where the model needs to exhibit "creative" properties, such as when generating text completions.


Parallel to BERT, other encoder-decoder models like the Transformer \cite{vaswani_attention_2023} and sequence-to-sequence networks with attention mechanisms \cite{bahdanau_neural_2016} have shown great results when translating sequences in tasks like machine translation, exemplified by Google's Neural Machine Translation system \cite{wu_googles_2016}, and speech recognition, as seen in Apple's Siri voice assistant \cite{hinton_deep_2012}.


\subsection{Generative AI}


Generative AI is a term used to describe a subset of artificial intelligence technologies that are designed to create new content. This can be images such as with DALL-E \cite{ramesh_zero-shot_2021}, text with models like GPT-3 \cite{brown_language_2020} or movies \cite{openai_video_2024}. These models are capable of generating realistic and arguably novel outputs by understanding and simulating the underlying structure of the training data. One of the most popular frameworks in Generative AI includes Generative Adversarial Networks (GANs), introduced by Goodfellow et al. \cite{goodfellow_generative_2014}, which consist of two neural networks, the generator and the discriminator. These two networks will compete against each other. The generator creates items that are as realistic as possible, and the discriminator evaluates them. This process runs until the discriminator can no longer accurately separate generated items from the training data.


\subsection{State-of-the-Art Large Language Models}


\gls{LLM} represent a significant breakthrough in \gls{NLP}. They are capable of understanding and generating text similar to that written by humans. In recent years, several cutting-edge LLMs have been developed by prominent companies and research institutions that have gained wide-spread use. This section gives an overview of some notable examples of these advanced LLMs.


\label{sec:openai_models}
\subsubsection{OpenAI's GPT Series}


OpenAI's \gls{GPT} series of language models have over the past few years featured some of the most widely used language models. GPT-1 was first released in 2017 followed by GPT-2, GPT-3, and GPT-4 (with various variants of these models). GPT-3, in particular, with its 175 billion parameters, has demonstrated strong capabilities in tasks such as text completion, question answering, and even code generation \cite{brown_language_2020}. These models are some of the most widely used models, primarily due to their popularisation by the product from the same company, ChatGPT \footnote{\href{https://chat.openai.com}{chat.openai.com}}.


\subsubsection{Mistral}
\label{sec:mistral_models}


Mistral is a french firm that has released a few models that has gained widespread adoption in the open source community. As of writing, \textit{Mistral-7B-Instruct-v0.2} had 2,297,845 million downloads on huggingface last month \footnote{\href{https://huggingface.co/mistralai/Mistral-7B-Instruct-v0.2}{The huggingface page for Mistral-7B-Instruct-v0.2}}, and Mixtral-8x7B-Instruct-v0.1 had 628,927 \footnote{\href{https://huggingface.co/mistralai/Mixtral-8x7B-Instruct-v0.1}{The huggingface page for Mixtral-8x7B-Instruct-v0.1}}.


\textit{Mistral 7B v0.1} \cite{jiang_mistral_2023} was their first major model to get widespread notoriety. The model is a 7-billion-parameter language model which was small enough to run on consumer-grade GPUs. The model utilised \gls{GQA}\cite{ainslie_gqa_2023} and \gls{SWA} \cite{roy_efficient_2020} techniques to achieve impressive results across various benchmarks, including reasoning, mathematics, and code generation tasks. \textit{Mistral 7B v0.1 instruct} is a related fine-tuned model.


The "instruct" version of generative AI models, such as the Mistral 7B, has been fine-tuned to follow prompted instructions. In contrast, the base model simply generates output based on the provided prompt. This process was first published by the team at OpenAI \cite{ouyang_training_2022}, however it’s also employed by mistral and other model vendors. This approach is commonly used for models deployed in AI assistants or chat applications.


The \textit{Mixtral of Experts} model \cite{jiang_mixtral_2024}, is a variant of the Mistral model that introduces a \gls{SMoE} architecture, as described by Jiang et al. \textit{Mixtral-8x7B-Instruct-v0.1} employs 8 feedforward blocks (experts) in each layer, with a router network selecting two experts for processing and combining their outputs at each timestep. The model has access to 47 billion parameters, but effectively only utilise 13 billion parameters during inference, which makes the model easier to deploy on GPUs with less amounts of memory.


\subsubsection{Google's Language Models}


Google has two major families of model, the first being the Gemini family, as introduced in a series of papers by Google's team \cite{gemini_team_gemini_2024-1}, consists of models like Gemini Ultra, Pro, and Nano, each of these models are designed for specific applications and more importantly size of GPU. Where the larger models require enterprise-grade GPUs that are expensive to operate. Gemini 1.5 extended on these models with an even larger context window by effectively processing and recalling information across millions of tokens in a multi-modal context (tokens include both text, audio and image tokens) \cite{gemini_team_gemini_2024}. This is the first model to demonstrate resilience to the problem first described by Nelson et al. where the model would be biassed towards instructions or data in the beginning and end of larger prompts \cite{liu_lost_2023}.


Goggles Gemma family of models \cite{gemma_team_gemma_2024} represents Google's effort to provide state-of-the-art, lightweight models to the open source community. These models, available in sizes of 2 billion and 7 billion parameters. The models demonstrate worse performance against their Gemini class of models across all tasks such language understanding and reasoning. However, the Gemma models’ size make them easier to deploy on smaller consumer-grade GPUs.


\subsubsection{The LLama family of models}


In February 2023, Meta AI released LLaMA \cite{touvron_llama_2023-1} in four distinct sizes: 7, 13, 33, and 65 billion parameters. The model utilised features such as SwiGLU activation functions, rotary positional embeddings, and root-mean-squared layer-normalisation to achieve comparable results to OpenAIs GPT-3 model. Despite being initially released under a noncommercial licence, the weights of LLaMA were leaked, prompting widespread unauthorised use. This accelerated its adoption across various applications.


Later in July of 2023, Meta released LLaMA-2 \cite{touvron_llama_2023-2} which was built upon the foundational models of its predecessor with enhanced data sets of 2 trillion tokens, fine-tuning capabilities, and improved dialogue system performance through specialised LLaMA-2 Chat models, these are similar to the instruct models mentioned in section ~\ref{sec:mistral_models}. LLaMA-2 had a 40\% larger training corpus and extended the context length to 4,000 tokens. The release included model sizes from 7 to 70 billion parameters. These models were released under a similar licence to the first LLaMA models.


Recently, in April 2024, Meta AI released LLaMA-3, this time with two models, one 8 billion parameter model and one 70 billion parameter model. These were open source and available online \footnote{\href{https://github.com/meta-llama/llama3}{The GitHub repository for LLaMA-3}} from day one under a commercial licence. The model was pre-trained on approximately 15 trillion tokens. Meta announced an, as of writing, future release of a 400 billion parameter model.


\subsubsection{Notable other vendors}


Besides the major players such as OpenAI, Google, and Meta, there exists a vast array of players, of varying size, that also develops language models. These include, but are not limited to, Anthropic, IBM and DeepMind (which is also a part of Google).


\section{Prompt engineering}
\label{sec:prompt_engineering}


Prompt engineering is the name given to the technique that evolved from the use of language models. This is the task of optimising the performance of a \gls{LLM} such as GPT-4, LLaMA, and others. This involves crafting the input text, or \textit{"prompt"} to these models in a way that guides them to produce desired outputs \cite{kathiriya_power_2023, chen_unleashing_2023}.


Prompt engineering is defined as the practice of designing input prompts that maximise the efficacy and accuracy of LLM outputs. It is a key factor in the success of deploying LLM-based applications. The process of prompt engineering involves several key techniques. A prompt should, according to Chen et al. include clear instructions and enough contextual details to guide the model towards providing the expected answer in the expected format. There are numerous advanced techniques such as "role-prompting", zero-shot, one-shot, and few-shot prompting that can improve the performance of LLM.


For instance, Kathiriya et al. \cite{kathiriya_power_2023} demonstrates that role-prompting produces responses with heightened professional relevance. Similarly, Chen et al. highlight how few-shot prompting can refine the model's ability to perform complex analytical tasks by providing some targeted examples. Both of these studies show how prompt engineering techniques can improve performance.


Figure \ref{fig:role_prompt}, taken from the paper published by Chen et al. \cite{chen_unleashing_2023} illustrates an example of role-prompting. In this example the LLM is instructed to assume the role of an expert in artificial intelligence, which aligns its responses with specific professional knowledge.


\begin{figure}[H]
\centering
\begin{tikzpicture}
    \newlength{\boxwidth}
    \setlength{\boxwidth}{0.4\textwidth}

    \node[draw, rectangle, rounded corners=2pt, inner sep=5pt, text width=\boxwidth, align=center] (a) {
        \smaller{You are an expert in artificial intelligence specialising in large language models...}
    };
    \node[draw, rectangle, rounded corners=2pt, inner sep=5pt, text width=\boxwidth, align=center, right=of a, xshift=0.5cm] (b) {
        \smaller{Here are five basic methods to quickly get started with large language models: \\\\ 1. Experiment with web-based interfaces \\\\ ...}
    };

    \draw[-latex, line width=1mm] (a) -- (b);

    \node[align=center, below=0.1cm of a] (c) {\footnotesize A role prompt};
    \node[align=center, below=0.1cm of b] (d) {\footnotesize Model output};
\end{tikzpicture}
\caption{Role prompting example.}
\label{fig:role_prompt}
\end{figure}



Another technique known as few-shot prompting, is shown in figure \ref{fig:fewshot_prompt}, taken from the paper written by Kathiriya et al. \cite{kathiriya_power_2023}. With this technique the model is provided with multiple examples to better understand the task. If only one example is given, this is referred to as "one-shot” prompting. Similarly, if no example is given, then the prompt is referred to as a "zero-shot" prompt.


\begin{figure}[H]
    \centering
    \begin{tikzpicture}
        % Set the width of the boxes
        \newlength{\boxwidth}
        \setlength{\boxwidth}{0.4\textwidth}

        \node[draw, rectangle, rounded corners=2pt, inner sep=5pt, text width=\boxwidth, align=left] (input) {
            \small{
                \\
                Example 1: ``Bug Report: Application crashes on startup. Category: Critical. Priority: High.''\\\\
                \vspace{2mm}
                Example 2: ``Bug Report: Minor typo in the user interface. Category: Trivial. Priority: Low.''\\\\
                \vspace{2mm}
                Task: ``Classify and prioritize the following bug report. Bug Report: User unable to login with valid credentials.''
                \\
                \vspace{2mm}
            }
        };

        \node[draw, rectangle, rounded corners=2pt, inner sep=5pt, text width=\boxwidth, align=left, right=of input, xshift=2cm] (output) {
            \small{
                \\
                Category: Major. Priority: High
                \\
                \vspace{2mm}
            }
        };

        \draw[-latex, line width=1mm] (input) -- (output);

        \node[align=center, below=0.1cm of input] (label1) {\footnotesize Input};
        \node[align=center, below=0.1cm of output] (label2) {\footnotesize Output};

    \end{tikzpicture}
    \caption{Few-shot prompting example.}
    \label{fig:fewshot_prompt}
\end{figure}



\section{Crawling}


\section{Information Retrieval}


Some stuff here


\subsection{Embedding functions}


Explain it. How it relates to LLMs.


\subsection{TFIDF}


\section{RAG}


Explain it.


\section{AI Agents}


Explain it.


\section{User research methods}


Explain it. The relevant ones that will be in this project.


\section{Related work area}


% \sweExpl{Relaterade arbeten}


...




\subsection{Major related work 1}


Carrier clouds have been suggested as a way to reduce the delay between the users and the cloud server that is providing them with content. However, there is a question of how to find the available resources in such a carrier cloud. One approach has been to disseminate resource information using an extension to OSPF-TE, see Roozbeh, Sefidcon, and Maguire \cite{roozbeh_resource_2013}.


\subsection{Major related work n}


\subsection{Minor related work 1}


...


\subsection{Minor related work n}


\section{Summary}


% \sweExpl{Det är trevligt om detta kapitel
%   avslutas med en sammanfattning. Till exempel kan du inkludera en tabell som
%   sammanfattar andras idéer och fördelar och nackdelar med varje - så som
%   senare kan du jämföra din lösning till var och en av dessa. Detta kommer
%   också att hjälpa dig att definiera de variabler som du kommer att använda
%   för din utvärdering.}


% \engExpl{It is nice to have this chapter conclude with a summary. For
%   example, you can include a table that summarizes other people's ideas and
%   benefits and drawbacks with each - so as later you can compare your solution
%   to each of them. This will also help you define the variables that you will
%   use for your evaluation.}


\cleardoublepage

\chapter{Method or Methods}
\label{ch:methods}


% \sweExpl{Metod eller Metodval}
% \generalExpl{This chapter is about Engineering-related
%   content, Methodologies and Methods.  Use a self-explaining title.\\The
%   contents and structure of this chapter will change with your choice of
%   methodology and methods.}




% \generalExpl{Describe the engineering-related contents (preferably with models) and the research methodology and methods that are used in the degree project.\\
% Give a theoretical description of the scientific or engineering methodology  you are going to use and why have you chosen this method. What other methods did you consider and why did you reject them?\\
% In this chapter, you describe what engineering-related and scientific skills you are going to apply, such as modeling, analyzing, developing, and evaluating engineering-related and scientific content. The choice of these methods should be appropriate for the problem. Additionally, you should be conscious of aspects relating to society and ethics (if applicable). The choices should also reflect your goals and what you (or someone else) should be able to do as a result of your solution - which could not be done well before you started.}


% The purpose of this chapter is to provide an overview of the research method
% used in this thesis. Section~\ref{sec:researchProcess} describes the research
% process. Section~\ref{sec:researchParadigm} details the research
% paradigm. Section~\ref{sec:dataCollection} focuses on the data collection
% techniques used for this research. Section~\ref{sec:experimentalDesign}
% describes the experimental design. Section~\ref{sec:assessingReliability}
% explains the techniques used to evaluate the reliability and validity of the
% data collected. Section~\ref{sec:plannedDataAnalysis} describes the method
% used for the data analysis. Finally, Section~\ref{sec:evaluationFramework}
% describes the framework selected to evaluate xxx.




This chapter outlines the methodologies and procedures used for conducting the research described in this thesis. The focus is on presenting the chosen methods for data collection, analysis, and evaluation of the deployment and effectiveness of AI-assistants within Canvas course rooms at KTH.
% \sweExpl{Vilka vetenskaplig eller ingenjörs-metodik ska du använda och varför har du valt den här metoden. Vilka andra metoder gjorde du övervägde du och varför du avvisar dem.
% Vad är dina mål? (Vad ska du kunna göra som ett resultat av din lösning - vilken inte kan göras i god tid innan du började)
% Vad du ska göra? Hur? Varför? Till exempel, om du har implementerat en artefakt vad gjorde du och varför? Hur kommer du utvärdera den.
% Syftet med detta kapitel är att ge en översikt över forsknings metod som
% används i denna avhandling. Avsnitt~\ref{sec:researchProcess} beskriver forskningsprocessen. Avsnitt~\ref{sec:researchParadigm} beskriver forskningsparadigmen detaljerat. Avsnitt~\ref{sec:dataCollection} fokuserar på datainsamlingstekniker som används för denna forskning. Avsnitt~\ref{sec:experimentalDesign} beskriver experimentell
% design. Avsnitt~\ref{sec:assessingReliability} förklarar de tekniker som används för att utvärdera
% tillförlitligheten och giltigheten av de insamlade uppgifterna. Avsnitt~\ref{sec:plannedDataAnalysis}
% beskriver den metod som används för dataanalysen. Slutligen, Avsnitt~\ref{sec:evaluationFramework}
% beskriver ramverket som valts för att utvärdera xxx.\\
% Ofta kan man koppla ett antal följdfrågor till undersökningsfrågan och problemlösningen t ex\\
% (1) Vilken process skall användas för konstruktion av lösningen och vilken process skall kopplas till denna för att svara på undersökningsfrågan?\\
% (2) Hur och vilket resultat (storheter) skall presenteras både för att redovisa svar på undersökningsfrågan (resultatkapitlet i denna rapport) och redovisa resultat av problemlösningen (prototypen, ofta dokument som bilagor men vilka dokument och varför?).\\
% (3) Vilken teori/teknik skall väljas och användas både för undersökningen (taxonomi, matematik, grafer, storheter mm)  och  problemlösning (UML, UseCases, Java mm) och varför?\\
% (4) Vad behöver du som student leverera för att uppnå hög kvaliet (minimikrav) eller mycket hög kvalitet på examensarbetet?\\
% (5) Frågorna kopplar till de följande underkapitlen.\\
% (6) Resonemanget bygger på att studenter på hing-programmet ofta skall konstruera något åt problemägaren och att man till detta måste koppla en intressant ingenjörsfråga. Det finns hela tiden en dualism mellan dessa aspekter i exjobbet.
% }


\section{Research Process}
\label{sec:research_process}


The research process within this thesis consisted of three main phases. This section will outline each phase and what the purpose was.


\subsection{Prof of concept}


The study will aim to achieve an assistant using open source and permissively licensed \gls{LLM}s and \gls{RAG} techniques, that is comparable to the current best in class models provided under less permissive licesnses and are only available through properitery APIs. Considering that this is a fairly complex task, the research started with a long phase of constructing various proof of concepts.


There would be two main proof of concepts, one using proprietary models and \gls{RAG} techniques, and one strictly using software that is under open source licenses and can be self-hosted.


\subsection{Implementation of study software}


The second phase consists of constructing the software that will be used during the study. This software is the actual AI assistant and all its components, such as the course room crawler and indexer, the infrastructure to run \gls{LLM}s at scale and a \gls{GUI} to interact with the assistant. This software also needs to be able to exchange various different components that are the subject for the study, such as the \gls{LLM} being used in a chat. Furthermore, the software needs to be able to record user interactions and responses to questions.


\subsection{Conduct the study at KTH}


This phase consists of deploying and monitoring the assistant in the course rooms that have enrolled in the study. This might involve modifying the software such that it works in the new course room, or ensuring that there is enough capacity on the platform to sustain the new users.


\subsection{Analyse results}


Once the study has concluded the data analysis will be conducted, the final results and analysis can be found in \autoref{ch:resultsAndAnalysis}. The various different data points collected in the experiments were all stored in a database. Once the study concluded, the analysis was done using different python jupyter notebooks. The notebooks can be found in the github repository for this thesis \footnote{\href{https://github.com/nattvara/DA231X/tree/main/results/notebooks}{github.com/nattvara/DA231X/results/notebooks}}. There are three notebooks, one for each theme.


The first is the \textit{"usage"} notebook. This contains the charts and tables constructed to analyse the general usage of the assistant in all courses. These cover metrics like, how many messages were sent, how many chats were held, users registered in each course, etc. The second notebook is the \textit{"feedback"} notebook. This notebook contains an analysis of all user submitted responses to the feedback questions. The third and last notebook is the \textit{"performance"} notebook. This notebook contains charts and tables that analyse all performance metrics collected by the system, such as how quickly models produced response to prompts, how long indexing took, etc.




% \sweExpl{Undersökningsrocess och utvecklingsprocess}


% \sweExpl{Figur~\ref{fig:researchprocess} visar de steg som utförs för att genomföra\\
% Beskriv, gärna med ett aktivitetsdiagram (UML?), din undersökningsprocess och utvecklingsprocess.  Du måste koppla ihop det akademiska intresset (undersökningsprocess) med ursprungsproblemet (utvecklingsprocess)
% denna forskning.\\
% Aktivitetsdiagram från t ex UML-standard}






% \generalExpl{Example of using customized item labels.}
% Some steps in the process:
% \begin{enumerate}[leftmargin=*, label=\textbf{Step \arabic*}, ref=Step \arabic*] %labelindent=1em for indent
%         \itemsep0em
%         \item \label{x:s1} plan experiment,
%         \item \label{x:s2} conduct experiment,
%         \item \label{x:s3} analyze data from the experiment, and
%         \item \label{x:s4} discuss the results of the analysis.
% \end{enumerate}


\section{Research Paradigm}
\label{sec:researchParadigm}


% \sweExpl{Undersökningsparadigm\\
% Exempelvis\\
% Positivistisk (vad/hur fungerar det?) kvalitativ fallstudie med en deduktivt (förbestämd) vald ansats och ett induktivt(efterhand uppstår dataområden och data) insamlade av data och erfarenheter.}


This research in this thesis follows a pragmatic approach that blends aspects of the study from positivist paradigms to investigate the practical application and user reception of AI assistants generally within the educational setting. Additionally, it seeks to determine which technologies perform best by analysing segmented responses to user queries based on the specific technologies used to generate the answers.
The pragmatic approach supports using mixed methods to answer the research questions effectively, focusing on 'what works' as the basis for knowledge claims. The positivist elements of the study will quantify which technologies deliver the fastest responses and yield the most usage among users. Participants will be exposed to one technology from a predefined set, selected through random sampling. This methodological approach allows for a positivist analysis to determine which technology is the fastest and most preferred by its users.


The collection and analysis of the feedback from the participants in the study is an interpretivisit approach to answer the questions of more subjective nature. These are questions like which technologies or models generate the most accurate results, or answers that are more preferred by users.


To summarise, initially the chatbot is deployed (positivist approach), followed by the collection and analysis of user feedback (interpretivist approach) to understand the broader implications of AI-assistant technology in specialised domains such as education. This method aims to understanding the functional capabilities of the AI-assistant in addition to its practical utility and acceptance by end-users, students and teachers.




\section{Data Collection}
\label{sec:dataCollection}


% \sweExpl{Datainsamling\\
% (Detta bör också visa att du är medveten om de sociala och etiska frågor som
% kan vara relevanta för dina data insamlingsmetod.)}
% \generalExpl{This should also show that you are aware of the social and ethical concerns that might be relevant to your data collection method.}


The data collection in this study comes from two sources;


\begin{itemize}
        \item The AI-assistant software built to conduct this research. This includes metrics from the usage of the system, in addition to integrated survey components of the system such as responses to questions from users of the system.
        \item Responses to coursework questions from students in selected courses that are part of the study, submitted as part of their course requirements.
\end{itemize}


\subsection{Data collected by the software constructed for this study}


The AI-assistant software was constructed to record various pieces of data, these can be grouped into four categories, operational data, usage data, performance data, feedback data. The data collected by the AI-assistant is anonymous. No personal information was recorded by the system, aside from any information that may have been submitted as part of a question by the user to the assistant.


\subsubsection{Operational data}


The operational data refers to data collected by the system to function. This includes information in or about the course rooms such as the various pieces of content found in a course room and their relations. This can be lecture slides, lab assignments, links to external sites etc.


\subsubsection{Usage data}


Usage data in system refers to data that is generated when users use the system. This includes session information and its metadata, chat information such as which course room a chat is associated with and which messages were sent by the user and the ai-assistant in each chat.


\subsubsection{Performance data}


Various metrics regarding the performance of the system are also computed and kept. These include metrics such as how quickly a model generates a response to a given prompt and how long the response. Also, how quickly a vector embedding was computed and how quickly the index returns documents.


\subsubsection{Feedback data}
\label{sec:method_feedback_data}




The feedback data is the most intentional data tracked by the system. This includes questions injected into the chat at specified intervals. These questions and intervals are the same for all users. The questions have a predefined set of answers and the system tracks which of the answers a user selects, or if they don’t select any answer at all. In addition binary thumbs up/down questions are also asked about certain responses from the system.


\subsection{Data collected by questions in the coursework}


In the courses that agreed, a form was distributed to all their students. The form was designed to gather insights as to why students replied like they did in the feedback questions. The form was slightly different in each course depending on the demographic of the participating students. For instance, one course was designed for maths teachers, therefore their form included questions meant to gather insights from the teaching viewpoint. The forms, their questions and responses, can be viewed in their entirety in \autoref{appendix:mg2040_form}, \autoref{appendix:ld1000_form} and \autoref{appendix:ld1006_form}. The qualitative analysis was very high-level, as it was not the primary form of gathering data for this study. The qualitative analysis can be found in section~\ref{sec:qualitative_analysis_of_user_responses}.


% \subsection{Sampling}
% not sure how i’d use these
% \subsection{Sample Size}


\subsection{The participants in the study}


The study sources its participants from courses from course administrators that have volunteered for their courses to participate in the study. The courses were found by emailing course responsibles at the EECS school at KTH in addition to connections of the supervisors of this thesis. \autoref{tab:courses_credits_students} shows the courses the participating students were taking.


\begin{table}[H]
\centering
{\scriptsize
\begin{tabularx}{\textwidth}{@{}llllc@{}}
\toprule
\textbf{Course Code} & \textbf{Course Name} & \textbf{Credits (hp)} & \textbf{Number of Students} \\ \midrule
LD1000 & Lär dig lära online & 2.0 & 83 \\
DD1380 & Javaprogrammering för Pythonprogrammerare & 1.5 & \textbf{393} \\
MG2040 & Assembly Technology & 6.0 & 32 \\
LD1006 & Kognitiv psykologi för lärare: Matematikundervisning & 3.0 & 51 \\
DD1349 & Projektuppgift i introduktion till datalogi & 3.0 & 213 \\
DD2419 & Project Course in Robotics and Autonomous Systems & \textbf{9.0} & 42 \\
DD1367 & Software Engineering in Project Form & \textbf{9.0} & 231 \\ \midrule
    & \textbf{Total} &  & \textbf{1045} \\
\bottomrule
\end{tabularx}
}
\vspace{2mm}
\caption{Courses that participated in the study, number of credits per course, and the number of students enrolled}
\label{tab:courses_credits_students}
\end{table}



\section[Experimental design/Planned Measurements]{Experimental design and\\Planned Measurements}
\label{sec:experimentalDesign}


Participants in the study will be randomly assigned to groups, each of which will use a specific set of technologies and techniques. This random assignment will be managed by software written for this study, and it will apply to every chat session started with the assistant by the participating student. Each group will utilise a unique configuration based on one of these predefined parameters;


\begin{itemize}
        \item The language model used to run the chat. This is the \gls{LLM} that is used to generate chat replies within the chat. It’s also used for the internal logic of the assistant. This internal logic consists of tasks such as identifying whether the user asked the assistant a question that needs data from the knowledgebase.
        \item The \gls{RAG} technique and technology used to access the indexed data. This could be \gls{TF-IDF} or an embedding function. The latter group also has a defined embedding model assigned.
        \item Post processing of the retrieved documents. This is a boolean flag that configures the assistant to post process documents before inserting them into the context window for the configured \gls{LLM} to generate an answer with.
\end{itemize}


The software allows for questions to be inserted into a chat the participant is having using the following two triggers.


\begin{enumerate}
        \item After a participant has had $n$ chats and is sending the $m$:th message in that chat
        \item After a student clicks one of the frequently asked questions
\end{enumerate}


Each of these triggers allows for inserting any given number of questions into the chat. Any question has to follow one of the following templates;


\begin{enumerate}[label=\alph*)]
        \item A question, such as \textit{"Was this a good answer?"} accompanied with a "thumbs up" or a "thumbs down" button to answer the question with.
        \item A question, such as \textit{"How accurate did you find this answer?"} with a set of answers to select from such as \textit{Very accurate}, \textit{Somewhat accurate}, \textit{Neither accurate nor inaccurate}, \textit{Somewhat inaccurate}, \textit{Very inaccurate}.
\end{enumerate}


With these configurations, the experiment aims to measure the impact of the predefined parameters on users' responses to the questions posed during the chats at the configured triggers.


\section{Test environment}


% \engExpl{Describe everything that someone else would need to reproduce your test environment/test bed/model/… .}
% \sweExpl{Testmiljö/testbädd/modell\\
% Beskriv allt att någon annan skulle behöva återskapa din testmiljö / testbädd / modell / …}


\subsection{Software}


To reproduce the results of this study the software that was written to crawl course rooms, index their content and host the chat with the assistant is available in its entirety on github, see \autoref{appendix:source_code}.


\subsection{Configuration}


There is quite a bit of configuration needed to get the software operational. The \textit{README} in the source code extensively covers how to run the software in most common environments, see \autoref{appendix:source_code}.


\subsection[Data/Access to Canvas]{Data and access to canvas}


To run the AI assistant with data from an actual course room the bot needs to have access to a canvas through a KTH registered user. Due to time constraints the option to use the official Canvas API was abandoned early in the planning of this study. The softwares’ crawler therefore needs the cookies of an authenticated user with access to the course rooms included in the study.


\textbf{note: maybe explore publishing the course content of some course rooms? As like a downloadable from the github repo or something?}


\subsection{Models}


The software constructed for the study utilise the transformers python library \footnote{The GitHub page for the transformers library \href{https://github.com/huggingface/transformers}{/github.com/huggingface/transformers} (accessed on \today)} maintained by HuggingFace. The library manages the download and loading of the open source \gls{LLM}s and embedding models used in the thesis. This also means that if the models are no longer available on the huggingface registry, or the registry is nonoperational, the models have to be obtained by other means. The open source models that are supported by the software are the following models


\begin{itemize}
        \item The Mistral-7B-Instruct, provided by MistralAI, \footnote{\href{https://huggingface.co/mistralai/Mistral-7B-Instruct-v0.2}{huggingface.co/mistralai/Mistral-7B-Instruct-v0.2} (accessed on \today)}
        \item The Gemma-7B, provided by Google, \footnote{\href{https://huggingface.co/google/gemma-7b}{huggingface.co/google/gemma-7b} (accessed on \today)}
        \item The Falcon-7B, provided by TII UAE, \footnote{\href{https://huggingface.co/tiiuae/falcon-7b}{huggingface.co/tiiuae/falcon-7b} (accessed on \today)}
        \item The SFR-Embedding-Mistral, provided by Salesforce, \footnote{\href{https://huggingface.co/Salesforce/SFR-Embedding-Mistral}{huggingface.co/Salesforce/SFR-Embedding-Mistral} (accessed on \today)}
        \item The Meta-Llama-3-8B-Instruct, provided by Meta, \footnote{\href{https://huggingface.co/meta-llama/Meta-Llama-3-8B-Instruct}{huggingface.co/meta-llama/Meta-Llama-3-8B-Instruct} (accessed on \today)}
\end{itemize}


In addition to the open source models the software also supports experiments using some proprietary models by OpenAI. The two models that are supported are the \textit{GPT-4} model and the \textit{text-embedding-3-large} embedding model \footnote{\href{https://platform.openai.com/docs/guides/embeddings}{platform.openai.com/docs/guides/embeddings/} (accessed on \today)}. Both of these are accessible using the python API \footnote{\href{https://github.com/openai/openai-python}{github.com/openai/openai-python} (accessed on \today)} which require an OpenAI subscription and API key.


\subsection{Hardware}


The study software should be able to execute on most common hardware and most parts of the application are not particularly compute intensive. The notable exception to this is the worker processes in the \text{LLM Service} part of the software that, depending on the model, run quite intensive compute loads. That is unless the agent runs either of the proprietary cloud hosted models provided by OpenAI.


If the agent is running one of the \gls{LLM}s it can run, such as the \textit{Mistral 7B instruct} model, the agent needs access to a quite capable GPU. For the supported models this needs to be a GPU with at least 24 GB of video memory. Example of such graphics cards are \textit{NVIDIA GeForce RTX 3090}\footnote{\href{https://www.nvidia.com/sv-se/geforce/graphics-cards/30-series/rtx-3090-3090ti/}{nvidia.com/sv-se/geforce/graphics-cards/30-series/rtx-3090-3090ti/} (accessed on \today)}, \textit{NVIDIA TITAN RTX}\footnote{\href{https://www.nvidia.com/en-eu/deep-learning-ai/products/titan-rtx/}{nvidia.com/en-eu/deep-learning-ai/products/titan-rtx/} (accessed on \today)} or \textit{NVIDIA A10 Tensor Core}\footnote{\href{https://www.nvidia.com/en-us/data-center/products/a10-gpu/}{nvidia.com/en-us/data-center/products/a10-gpu/} (accessed on \today)}. This used servers on AWS, specifically the G5 instances (g5.4xlarge) equipped with \textit{NVIDIA A10 Tensor Core} GPUs \footnote{\href{https://aws.amazon.com/ec2/instance-types/g5/}{aws.amazon.com/ec2/instance-types/g5/} (accessed on \today)}. The open source embedding models such as \textit{SFR-Embedding-Mistral} can be run on CPUs, which only require the same amount of RAM available to run the models.


\section{Assessing reliability and validity of the data collected}
\label{sec:assessingReliability}


% \sweExpl{Bedömning av validitet och reliabilitet hos använda metoder och insamlade data }




\subsection{Validity of method}
\label{sec:validtyOfMethod}


% \sweExpl{Giltigheten av metoder\\
%   Har dina metoder gett dig de rätta svaren och lösningarna? Var metoderna korrekta?}


% \engExpl{How will you know if your results are valid?}
% \engExpl{Remember that validity is about the \textit{accuracy} of a measurement while reliability is about the \textit{consistency} of the measurement values under the same conditions (\ie repeatability).}


To research how well AI-assistants work in a specialised domain, other methods could’ve been used. For instance, instead of building an assistant and deploying it, the thesis could’ve explored similar domains and drawn conclusions from the success of similar systems in similar domains. However, actually implementing an assistant and evaluating its efficacy using standard methods such as \gls{TAM} and \gls{ECM} is a more accurate way of measuring the research questions laid out.


\subsection{Reliability of method}
\label{sec:reliabilityOfMethod}


% \sweExpl{Tillförlitlighet av för metoder\\
% Hur bra är dina metoder, finns det bättre metoder? Hur kan du förbättra dem?}
% \engExpl{How will you know if your results are reliable?}


The methods used in this study, including those for measuring user acceptance of the tool, are considered reliable. However, the results of the study will be heavily impacted by the type of users using the tool. Given the research is taking place at the division of robotics at a technical university, the participating courses are mostly technical courses with tech savvy students. To reproduce the results of this study it would have to closely reproduce the student population participating in the study. To increase the reliability of the method in this study the research could have been performed at a wider array of universities with a more diverse student population.


\subsection{Data validity}
\label{sec:dataValidity}


% \sweExpl{Giltigheten av uppgifter\\
% Hur vet du om dina resultat är giltiga? Är ditt resultat rättvisande?}

% TODO: after the results section has been written.


% \subsection{Reliability of data}
% \label{sec:reliabilityOfData}




% \sweExpl{Tillförlitlighet av data\\
% Hur vet du om dina resultat är tillförlitliga? Hur bra är dina resultat?}


% TODO: after the results section has been written.


% \section{Planned Data Analysis}
% \label{sec:plannedDataAnalysis}


% \sweExpl{Metod för analys av data}


% TODO: after the results section has been written.


% \subsection{Data Analysis Technique}
% \label{sec:dataAnalysisTechnique}


% TODO: after the results section has been written.


% \subsection{Software Tools}
% \label{sec:softwareTools}


% TODO: after the results section has been written.


% \section{Evaluation framework}
% \label{sec:evaluationFramework}


% TODO: after the results section has been written.


% \sweExpl{Utvärdering och ramverk\\
% Metod för utvärdering, jämförelse mm. Kopplar till kapitel~\ref{ch:resultsAndAnalysis}.}


\section{System documentation}
\label{sec:systemDocumentation}


% \sweExpl{Systemdokumentation\\
% Med vilka dokument och hur skall en konstruerad prototyp dokumenteras? Detta blir ofta bilagor till rapporten och det som problemägaren till det ursprungliga problemet (industrin) ofta vill ha.\\
% Bland dessa bilagor återfinns ofta, och enligt någon angiven standard, kravdokument, arkitekturdokument, designdokumnet, implementationsdokument, driftsdokument, testprotokoll mm.}
% \generalExpl{If this is going to be a complete document consider putting it in as an appendix, then just put the highlights here.}


\autoref{appendix:source_code} includes links to the source code developed for the research in this thesis. The source code includes a \textit{README} with comprehensive instructions for how to build, run and deploy the software.




\cleardoublepage

\chapter{What you did}
\label{ch:whatYouDid}


% \engExpl{Choose your own chapter title to describe this}
% \sweExpl{[Vad gjorde du? Hur gick det till? – Välj lämplig rubrik (“Genomförande”, “Konstruktion”, ”Utveckling”  eller annat]}


% \engExpl{What have you done? How did you do it? What design decisions did you make? How did what you did help you to meet your goals?}
% \sweExpl{Vad du har gjort? Hur gjorde du det? Vilka designval gjorde du?\\
% Hur kom det du hjälpte dig att uppnå dina mål?}


% the following sets the TOC entry to break after the & - note you have to include the first letter of the following word as it get swolled by the \texorpdfstring{}{} processing


% \section[Hardware/Software design …/Model/Simulation model \&\texorpdfstring{\\}{ p} parameters/…]{Hardware/Software design …/Model/Simulation model \& parameters/…}


\section{Proof of Concepts}


The following section will outline the various \gls{POC} applications that were built before the actual software that was written to conduct the research outlined in this thesis. Each \gls{POC} will outline what it was trying to accomplish and what the outcome was.


\subsection{Langchain based applications}


Langchain is a company \footnote{\href{https://langchain.com}{langchain.com}} and framework \footnote{\href{https://python.langchain.com}{python.langchain.com}} for building context aware reasoning applications. The framework allows for easy composition of language models and \gls{RAG} techniques and tools that makes it easy to build chatbots with a connected knowledge base. This section outlines some \gls{POC}s that were made with the langchain framework.


\subsubsection{GPT-4 and text-embedding-3-large}
\label{sec:poc_gpt_langchain}


To build a chat application with an AI-assistant that has access to an external knowledge base, one of the most popular approaches is to use langchain to connect the following four parts.


\begin{enumerate}
        \item A \gls{LLM}, such as GPT-4 to run the chat.
        \item A \gls{LLM}, such as GPT-4 to run the query construction.
        \item An embedding function, such as OpenAI’s text-embedding-3-large used to index and query documents.
        \item A vector store, such as ChromaDB, that stores the vector embeddings and associated documents \footnote{\href{https://www.trychroma.com/}{trychroma.com/}}
\end{enumerate}.


In this configuration, Lanchain acts as the glue connecting these components and handling tasks like chunking larger documents. The goal of this \gls{POC} was to test a common approach for building AI assistants and evaluate its potential for use in the full study. \href{https://www.youtube.com/watch?v=bKjxi-NKRHo}{A video can be seen here} that showcases this \gls{POC}.


\subsubsection{Mistral 7B v0.2 and e5-large-v2}


There was a \gls{POC} constructed that had the same approach as the one outlined in \ref{sec:poc_gpt_langchain} with the notable requirement that all tools had to be under an open source licence. This meant the GPT-4 model and text-embedding-3-large models couldn’t be used. A similar version of the same \gls{POC} was made that used the Mistral 7B v0.2 model and the embedding function e5-large-v2 \cite{wang_text_2024}. These are both under an open source licence and are freely available on Huggingface \footnote{\href{https://huggingface.co/mistralai/Mistral-7B-Instruct-v0.2}{huggingface.co/mistralai/Mistral-7B-Instruct-v0.2} \href{https://huggingface.co/intfloat/e5-large-v2}{huggingface.co/intfloat/e5-large-v2}}. This \gls{POC} did however suffer from poor performance in initial tests for retrieval and performance. It was difficult to tune the prompts to get decent performance. This \gls{POC} showed it was difficult for the researcher to get good performance out of certain models using the langchain framework.


\subsection{Custom applications}


\subsubsection{Simple Python API for models on Huggingface}


Langchain and similar tools support running language models locally. However, working with the prompt templates in less advanced models than GPT-4 and achieving good retrieval and chat performance was challenging. Therefore, a simple \gls{POC} was developed to create higher-level Python abstraction APIs on top of the Hugging Face Transformers library that could be integrated into completely custom solutions. These APIs include examples like those shown in listings ~\ref{fig:python-apis-for-llm} and ~\ref{fig:python-apis-for-embeddings}.


\begin{listing}[H]
\centering
\renewcommand{\theFancyVerbLine}{\scriptsize\arabic{FancyVerbLine}}
\scriptsize
\begin{minted}[
frame=lines,
framesep=2mm,
baselinestretch=1.2,
fontsize=\scriptsize,
linenos
]{python}
def load_hf_model(
    model_path: str,
    device: str
) -> (transformers.AutoModelForCausalLM, transformers.AutoTokenizer):
    """
    Loads a Hugging Face causal language model and its tokenizer for a given
    model path and device.
    """

def generate_text(
    model: transformers.AutoModelForCausalLM,
    tokenizer: transformers.AutoTokenizer,
    device: str,
    params: Params,
    prompt: str
) -> str:
    """
    Generates text from a prompt using the specified model, tokenizer, and
    generation parameters.
    """

async def generate_text_streaming(
    model: transformers.AutoModelForCausalLM,
    tokenizer: transformers.AutoTokenizer,
    device: str,
    params: Params,
    prompt: str
) -> AsyncGenerator[str, None]:
    """
    Asynchronously generates text from a prompt, yielding tokens incrementally.
    Useful for streaming responses.
    """

def _tokenise_inputs(
    tokeniser: transformers.AutoTokenizer,
    input_texts: list[str],
    max_length: int = 8192
) -> dict:
    """
    Tokenizes the input texts with padding and truncation.
    """

def should_stop_generating(
    output_token_ids: list,
    tokenizer: transformers.AutoTokenizer,
    params: Params,
    token_id: int
) -> bool:
    """
    Determines whether to stop generating text based on stop conditions.
    """
\end{minted}
\caption{High level API on-top of Huggingface's tranformers library that can be used for generating text using models available on Huggingface.}
\label{fig:python-apis-for-llm}
\end{listing}

\begin{listing}[H]
\centering
\renewcommand{\theFancyVerbLine}{\scriptsize\arabic{FancyVerbLine}}
\scriptsize
\begin{minted}[
frame=lines,
framesep=2mm,
baselinestretch=1.2,
fontsize=\scriptsize,
linenos
]{python}
def load_hf_embedding_model(
    model_path: str,
    device: str
) -> (torch.nn.Module, transformers.AutoTokenizer):
    """
    Loads a Hugging Face embedding model and its tokenizer for a given model
    path and device.
    """

async def compute_embedding(
    model: torch.nn.Module,
    tokeniser: transformers.AutoTokenizer,
    text: str
) -> List[float]:
    """
    Computes and returns the normalized embedding for a given text using the
    specified model and tokenizer.
    """

def _compute_model_embeddings(
    model: torch.nn.Module,
    tokenised_inputs: dict
) -> torch.Tensor:
    """
    Computes the model embeddings from the tokenized inputs.
    """
\end{minted}
\caption{High level API on-top of Huggingface's tranformers library that can be used for generating vector embeddings using models available on Huggingface.}
\label{fig:python-apis-for-embeddings}
\end{listing}



\subsubsection{Mistral 7B v0.2 and Opensearch}


\subsubsection{Mistral 7B v0.2 with Opensearch and Postprocessing}


\subsubsection{Mistral 7B v0.2 with Opensearch and Heavy Postprocessing}


\section{The architecture of the software}


% https://lucid.app/lucidchart/f5dc4477-fa25-44e8-93d9-946def9cd4a9/edit

\begin{figure}[H]
    \centering
    \includegraphics[width=\textwidth]{content/figures/assets/06-system-architecture-diagram.pdf}
    \caption{Diagram that shows the system architecture of the software constructed to run the study}
    \label{fig:system_architecture_diagram}
\end{figure}



\subsection{Courseroom Crawler}


\subsection{Running large language models at scale}


\subsection{Datastore and Index}


\subsection{User interface}


\section{How the software is deployed}


% https://lucid.app/lucidchart/4fca3d7e-5905-40a3-8926-917ce384926b/edit

\begin{figure}[H]
    \centering
    \includegraphics[width=\textwidth]{content/figures/assets/07-aws-diagram.pdf}
    \caption{Diagram that shows how the software was deployed on Amazon AWS}
    \label{fig:aws_diagram}
\end{figure}



% \sweExpl{Hårdvara / Mjukvarudesign ... / modell / Simuleringsmodell och parametrar / …}




% \sweExpl{Figur~\ref{fig:homepageicon}  visar en enkel ikon för en hemsida. Tiden för att få tillgång till den här sidan när den laddas kommer att kvantifieras i en serie experiment. De konfigurationer som har testats i provbänk listas ini tabell~\ref{tab:configstested}.\\
% Vad du har gjort? Hur gjorde du det? Vilka designval gjorde du?}


\section{Implementation …/Modeling/Simulation/…}
\label{sec:implementationDetails}


\subsection{Some examples of coding}


% \engExpl{This section is simply to show some example of how you can include code in your thesis - this is not a section you would have in your thesis.}
% \sweExpl{Det här avsnittet är helt enkelt för att visa ett exempel på hur du kan inkludera kod i ditt examensarbete - det här är inte ett avsnitt du skulle ha i ditt examensarbete.}


\subsection{Some examples of figures in tikz}


% \engExpl{This section is simply to show some example of how you can draw your own figures for in your thesis - this is not a section you would have in your thesis.}
% \sweExpl{Det här avsnittet är helt enkelt för att visa ett exempel på hur du kan rita dina egna figurer i ditt examensarbete – det här är inte ett avsnitt du skulle ha i ditt examensarbete.}
% These figures are just some examples to show that you can draw your own figures for in your thesis. This has two advantages: \first you do not have to worry about copyrights -- as these are your own figures and \Second the text is now readable and not simply a picture of text -- so screen readers can read the figure's contents to someone who is listening to the contents of your thesis.


\subsubsection{Azure's Form Recognizer}


\cleardoublepage

\chapter{Results and Analysis}
\label{ch:resultsAndAnalysis}


% \engExpl{Sometimes this is split into two chapters.\\Keep in mind: How you are going to evaluate what you have done? What are your metrics?\\Analysis of your data and proposed solution\\Does this meet the goals which you had when you started?}


This chapter will present and analyse the results of the research conducted in this thesis.


% \sweExpl{I detta kapitel presenterar vi resultaten och diskutera dem.\\Ibland delas detta upp i två kapitel.\\Hur du ska utvärdera vad du har gjort? Vad är din statistik?\\Analys av data och föreslagen lösning\\Innebär detta att uppfyllelse av de mål som du hade när du började?}


% \sweExpl{Huvudsakliga resultat}


\section{Feasibility of building an AI assistant on open source technologies}


One of the goals of the research in this thesis was, as outlined in section~\ref{sec:goals}, to assess the feasibility of building an AI-assistant on open-source technologies and deploying the agent in an academic setting. This section will outline the results and showcase the impact open source tooling had on the implementation of the AI assistant.


\subsection{How popular was the system}


The system was developed during the spring of 2024 and gradually deployed to seven real courses at KTH starting on the 18th of April 2024. The students in the courses that participated in the study held a total of \input{results/latex_variables/usage-01-total-number-of-chats}chats and the users of the system sent \input{results/latex_variables/usage-07-total-number-of-messages}messages. As can be seen in \autoref{fig:usage_01_cumulative_number_of_chats} and \autoref{fig:usage_08_number_of_messages_per_day} these steadily increased over the course of the study as students initiated new chats with the assistant.


\begin{figure}[H]
    \centering
    \includegraphics[width=1\textwidth]{results/plots/assets/usage-01-cumulative-number-of-chats.png}
    \caption{Cumulative number of chats started by users participating in the study}
    \label{fig:usage_01_cumulative_number_of_chats}
\end{figure}


\begin{figure}[H]
    \centering
    \includegraphics[width=\textwidth]{results/plots/assets/usage-08-number-of-messages-per-day.png}
    \caption{Cumulative number of messages per day}
    \label{fig:usage_08_number_of_messages_per_day}
\end{figure}


Separating the chats initiated in the separate course rooms we observe that some courses followed a fairly linear increase in the number of chats. One example of this is the course \textit{MG2040 Assembly Technology 6.0 credits}, which can be seen in figure~\ref{fig:usage_02_cumulative_number_of_chats_per_course}.


\begin{figure}[H]
    \centering
    \includegraphics[width=1\textwidth]{results/plots/assets/usage-02-cumulative-number-of-chats-per-course.png}
    \caption{Cumulative number of chats started by users participating in the study in each course}
    \label{fig:usage_02_cumulative_number_of_chats_per_course}
\end{figure}


Looking at other courses in figure~\ref{fig:usage_02_cumulative_number_of_chats_per_course} it is evident that not all courses follow the same pattern as \textit{MG2040}. Some courses initially have very few chats due to the chatbot not being deployed simultaneously across all courses. \autoref{tab:course_start_dates} details the start dates for each course. For instance, \textit{DD1349 Projektuppgift i introduktion till datalogi 3,0 hp} exhibits a steep increase in users when it launched, followed by no further growth. This is because the course officially ended shortly after the chatbot was introduced. In all courses participating in the study, the chatbot was deployed well after the courses had already begun, therefore a similar pattern can be seen in many other courses.


\begin{table}[H]
\centering
{\scriptsize
\begin{tabularx}{\textwidth}{@{}X c@{}}
\toprule
\textbf{Course} & \textbf{Go live date} \\ \midrule
LD1000 Lär dig lära online 2,0 hp & 2024-04-17 \\
LD1006 Kognitiv psykologi för lärare: Matematikundervisning 3,0 hp & 2024-04-17 \\
MG2040 Assembly Technology 6.0 credits & 2024-04-17 \\
DD1349 Projektuppgift i introduktion till datalogi 3,0 hp & 2024-05-01 \\
DD2419 Project Course in Robotics and Autonomous Systems 9.0 credits & 2024-05-03 \\
DD1367 Software Engineering in Project Form 9.0 credits & 2024-05-07 \\
DD1380 Javaprogrammering för Pythonprogrammerare 1,5 hp & 2024-05-13 \\
\bottomrule
\end{tabularx}
}
\vspace{2mm}
\caption{The start dates for each course, when the bot was deployed in each canvas course room.}
\label{tab:course_start_dates}
\end{table}



\autoref{fig:usage_03_number_of_chats_per_course} shows the total number of chats held in each course, and \autoref{fig:usage_03_number_of_chats_per_calendar_week_per_course} shows how these were distributed over each course over time. The course \textit{MG2040} held the most, \input{results/latex_variables/usage-04-chats-mg2040}chats. \textit{DD1349} held the second most and \textit{DD1367} the third most, \input{results/latex_variables/usage-05-chats-dd1349}and \input{results/latex_variables/usage-06-chats-dd1367}chats respectively.


\begin{figure}[H]
    \centering
    \includegraphics[width=\textwidth]{results/plots/assets/usage-05-number-of-chats-per-course.png}
    \caption{Number of chats held by in each course}
    \label{fig:usage_03_number_of_chats_per_course}
\end{figure}


\begin{figure}[H]
    \centering
    \includegraphics[width=1\textwidth]{results/plots/assets/usage-04-number-of-chats-per-calendar-week-per-course.png}
    \caption{Number of chats held each week per course}
    \label{fig:usage_03_number_of_chats_per_calendar_week_per_course}
\end{figure}




Looking at the number of sessions created in \autoref{fig:usage_06_number_of_sessions_per_day} and \autoref{fig:usage_07_number_of_sessions_per_day_and_course}, we can see a similar pattern linear pattern. A session is started whenever a user loads the application without already having loaded it before. A session is not tracked between devices, therefore a user would have two sessions if the same user accessed the chat on two different devices, such as a desktop and a mobile phone. However, the same session is used across courses.


\begin{figure}[H]
    \centering
    \includegraphics[width=\textwidth]{results/plots/assets/usage-06-number-of-sessions-per-day.png}
    \caption{Cumulative number of users as percentage of number of users in all participating courses}
    \label{fig:usage_06_number_of_sessions_per_day}
\end{figure}


\begin{figure}[H]
    \centering
    \includegraphics[width=1\textwidth]{results/plots/assets/usage-07-number-of-sessions-per-day-and-course.png}
    \caption{Cumulative number of users per day in each course, as a percentage of the course's students, that had started a chat}
    \label{fig:usage_07_number_of_sessions_per_day_and_course}
\end{figure}


Looking at the distribution of how many chats and messages is sent per session, as seen in figure \autoref{fig:usage_12_number_of_sessions_with_number_of_chats} and \autoref{fig:usage_13_number_of_sessions_with_number_of_messages} we can see that it was very common for users to only start one or two chats. Most users sent quite a few messages though. The average user held \input{results/latex_variables/usage-02-average-chats-per-session}chats and sent \input{results/latex_variables/usage-03-average-messages-per-session}messages.


\begin{figure}[H]
    \centering
    \includegraphics[width=1\textwidth]{results/plots/assets/usage-12-number-of-sessions-with-number-of-chats.png}
    \caption{Number of sessions that held a given number of chats}
    \label{fig:usage_12_number_of_sessions_with_number_of_chats}
\end{figure}


\begin{figure}[H]
    \centering
    \includegraphics[width=\textwidth]{results/plots/assets/usage-13-number-of-sessions-with-number-of-messages.png}
    \caption{Number of sessions with each number of messages}
    \label{fig:usage_13_number_of_sessions_with_number_of_messages}
\end{figure}


\subsection{Open source v. Proprietary LLMs}


With regards to the feasibility of building AI assistants on open source technologies there are a number of metrics to look at for comparing open source \gls{LLM}s to proprietary language models. \autoref{tab:sessions_chats_and_messages_by_model} shows metrics for both models that were included in the experiment. The experiment was designed to sample between the included models randomly, and as we can see the number of sessions started between the two models are virtually the same. However, there are notable differences in the number of chats started and messages sent between the two. The proprietary model, \textit{GPT-4} by OpenAI, leads the open source model, \textit{Mistral-7B-Instruct-v0.2} by MistralAI. Section~\ref{sec:qualitative_analysis_of_user_responses} and \ref{sec:impact_of_llm_on_user_preferences} will showcase the user preferences with respect to these models, which could explain the discrepancy between two models with regards to simple usage metrics, which is what is shown in \autoref{tab:sessions_chats_and_messages_by_model}.



\begin{table}[H]
\centering
\scriptsize
\begin{tabular}{@{}lcccccc@{}}
\toprule
Model & No. Sessions & \% & No. Chats & \% & No. Messages & \% \\
\midrule
\textit{openai/gpt4} & \textbf{305} & \textbf{50.92} & \textbf{359} & \textbf{60.23} & \textbf{1141} & \textbf{52.29} \\
\textit{mistralai/Mistral-7B-Instruct-v0.2} & 294 & 49.08 & 237 & 39.77 & 1041 & 47.71 \\

\bottomrule
\end{tabular}
\caption{Statistics of Sessions, Chats, and Messages by Model}
\label{tab:sessions_chats_and_messages_by_model}
\end{table}



Looking at the operational performance of both models included in \autoref{tab:sessions_chats_and_messages_by_model}, there are two notable metrics that were measured in the experiment with respect to operating these models, more specifically metrics that doesn’t measure the quality of their responses (these are covered in \label{sec:impact_of_llm_on_user_preferences}). The metrics are; the response time for the model and time taken to generate queries. The latter is measuring what is generated by the system to query the index that was produced when crawling the course room. The \gls{LLM} is obviously used to generate the assistant's next message, but it is also used to generate a search query, based upon the current conversation between the assistant and the user. The time taken to generate this query is obviously important for the overall performance of the system.


\autoref{fig:performance_05_daily_average_response_time_including_pending_time} shows the daily average response for each model. This includes the time taken before a worker node had picked up the workload. This is important because in the event of high traffic to the system, \gls{LLM} tasks could be queued up and response times could increase. The chart shows that the two models are generally very similar in terms of the time it takes them to produce a reply to the user's question. It is notable however, that the open source alternative (\textit{Mistral-7B-Instruct-v0.2}) has higher peaks on certain days.


\begin{figure}[H]
    \centering
    \includegraphics[width=\textwidth]{results/plots/assets/performance-05-daily-average-response-time-per-model-including-pending-time.png}
    \caption{How long each model took to generate a response, including time spent pending.}
    \label{fig:performance_05_daily_average_response_time_including_pending_time}
\end{figure}


\autoref{fig:performance_06_generating_queries} shows the time taken to produce queries. Similar to what could be said about \autoref{fig:performance_05_daily_average_response_time_including_pending_time}, both models perform very similarly. However, in this metric the open source alternative is faster. The reason the open source model outperforms \textit{GPT-4} on this metric is likely due to the higher latency sometimes observed on the OpenAI API. The custom built solution to operate \gls{LLM} for this study exhibits much lower latency.


\begin{figure}[H]
    \centering
    \includegraphics[width=1\textwidth]{results/plots/assets/performance-06-generating-queries.png}
    \caption{How long each model took to generate queries.}
    \label{fig:performance_06_generating_queries}
\end{figure}


\subsection{Open source v. Proprietary Embedding functions}


The intention of the research in this thesis was to compare open source embedding functions with proprietary alternatives commonly used in \gls{RAG}-based applications. In addition to this, the experiment was also designed to be able to measure vector search as a retrieval technique, with traditional full text search. However, due to the number of students that participated in the study, no configuration was used that didn’t use the vector search strategy with the embedding function \textit{text-embedding-3-large} by OpenAI \cite{openai_new_2024}. So no metrics were captured in the retrieval phase for any other strategy or model. The retrieval time taken for the model that was used is shown in \autoref{fig:performance_07_embedding_function_query_performance}. Metrics were, however, captured during the indexing phase on other models.


\begin{figure}[H]
    \centering
    \includegraphics[width=1\textwidth]{results/plots/assets/performance-07-embedding-function-query-performance.png}
    \caption{How long each model took to execute queries.}
    \label{fig:performance_07_embedding_function_query_performance}
\end{figure}


\subsubsection{Understanding the indexing}
\label{sec:understanding_indexing}


The indexing of course rooms was done regularly. It wasn’t done on a schedule, it was instead done on an ad-hoc basis whenever a course was updated with new content. \autoref{fig:performance_01_timeline_of_snapshots_taken} shows a timeline for each course which includes the date for when each snapshot of the course room was taken.


\begin{figure}[H]
    \centering
    \includegraphics[width=\textwidth]{results/plots/assets/performance-01-timeline-of-snapshots-taken.png}
    \caption{Timeline for each snapshot taken of the courses participating in the study.}
    \label{fig:performance_01_timeline_of_snapshots_taken}
\end{figure}


Each course room is different and includes pages and content of varying length. No exact metric for how much content was included in each course room is presented in this section, however \autoref{fig:performance_02_urls_per_course} shows how many urls the crawler found in each course, and how many of them were indexed. This is an imperfect, yet decent proxy for how "large" a course room was.


\begin{figure}[H]
    \centering
    \includegraphics[width=1\textwidth]{results/plots/assets/performance-02-urls-per-course.png}
    \caption{Number of URLs included in the most recent snapshot of each course.}
    \label{fig:performance_02_urls_per_course}
\end{figure}


To understand why the size of the course room is relevant we need to look at \autoref{fig:performance_03_index_time}, that shows how long each course room took to index. We can see that indexing time for the same course room vary a lot. The reason for this is not that the size of the course room varies over time. Looking at \autoref{fig:performance_03_index_time} we can see that the higher values occur when snapshots are taken simultaneously. The way indexing time is measured is by taking the time between the first url in a course room being crawled, and the last time a url was indexed in that snapshot. \autoref{fig:performance_03_index_time} suggests that there is an operation that takes a lot of time, and the indexer gets overloaded when a lot of courses are being indexed at the same time. Section~\ref{sec:measuring_indexing_performance} will show that this is due to the performance of the open source embedding function used in the experiments.


\begin{figure}[H]
    \centering
    \includegraphics[width=1\textwidth]{results/plots/assets/performance-03-index-time.png}
    \caption{How long indexing took per snapshot.}
    \label{fig:performance_03_index_time}
\end{figure}


\subsubsection{Measuring indexing performance}
\label{sec:measuring_indexing_performance}


As shown and explained in section~\ref{sec:understanding_indexing} the most time consuming part of indexing a course room was computing the embeddings for all documents added to the index. \autoref{tab:average_response_time_embedding_functions} shows the open source embedding function used in the experiment, Salesforces’ \textit{SFR-Embedding-Mistral}\cite{meng_sfr-embedding-mistral_2024}, which was chosen because it had the highest score on the \gls{MTEB}-benchmark \footnote{As of \today}, is two orders of magnitude slower than \textit{text-embedding-3-large}, the currently best embedding function developed by OpenAI. The reason for this was likely the execution environment chosen for the open source candidate.



\begin{table}[H]
\centering
\begin{tabular}{@{}lc@{}}
\toprule
Model Name & Average Response Time (seconds) \\
\midrule
\textit{Salesforce/SFR-Embedding-Mistral} & 103.51s \\
\textit{openai/text-embedding-3-large} & \textbf{0.03}s \\

\bottomrule
\end{tabular}
\caption{Average Total Response Time per Embedding Function}
\label{tab:average_response_time_embedding_functions}
\end{table}



During the experiment the embedding functions utilised the same servers that ran the open source \gls{LLM}s. To utilise the hardware rented for this thesis optimally, the open source embedding models were executed on the unutilised CPUs of the servers which ran the \gls{LLM}s on their attached GPU devices. The embedding models are, as opposed to the \gls{LLM}s at least feasible to run using a CPU only. However, as shown in \autoref{tab:average_response_time_embedding_functions}, this had quite a drastic impact on the indexing performance.


Had the open source models been used for retrieval, the performance difference would likely not have been as big. As shown in \autoref{tab:tab:average_response_time_by_length_embedding_functions} the difference in computation time between \textit{GPT-4} and \textit{SFR-Embedding-Mistra} is much smaller for smaller documents. Computing the embedding of a user query, which is what is done during retrieval, equates to computing the embedding for a very small document.



\begin{table}[H]
\centering
\scriptsize
\begin{tabular}{@{}lcccccc@{}}
\toprule
Prompt Length (No. characters) & 0-256 & 257-512 & 513-1024 & 1025-2048 & 2049-4096 & 4097-8192 \\
\midrule
Model Name &  &  &  &  &  &  \\
\textit{Salesforce/SFR-Embedding-Mistral} & 14.24s & 17.51s & 25.9s & 48.6s & 102.56s & 153.31s \\
\textit{openai/text-embedding-3-large} & \textbf{0.0}s & \textbf{0.02}s & \textbf{0.02}s & \textbf{0.06}s & \textbf{0.02}s & \textbf{0.03}s \\

\bottomrule
\end{tabular}
\caption{Average Total Response Time per Embedding Function and Prompt Length}
\label{tab:tab:average_response_time_by_length_embedding_functions}
\end{table}





\section{The impact of different LLM models on the speed, accuracy and reliability of responses}
\label{sec:impact_of_llm_on_user_preferences}


This section will present and analyse the gathered data on user preference and technological efficacy of different tools and technologies such as different \gls{RAG} toolchains and \gls{LLM}, as outlined in section~\ref{sec:goals}.


\subsection{Thumbs up/Thumbs down responses to FAQ questions}


After each response to a question that was selected from the frequently asked questions (FAQs) that were shown before a user had sent any messages, as shown in \autoref{fig:faq_questions}, the user was presented with an optional binary thumbs up/down vote on the quality of the response. Both had a tooltip that said \textit{"This was a good response"} and \textit{"This was a bad response"} respectively. \autoref{fig:feedback_01_frequency_of_answer_for_question_2e09fd} shows how the participants in the study voted. Generally, it can be observed that users had a positive response to the replies produced to the FAQs. Almost twice as many positive answers were recorded as negative responses.


Looking at the breakdown per model, the open-source alternative \textit{Mistral-7B-Instruct-v0.2}, had almost the same amount of positive and negative responses, which can be seen in \autoref{fig:feedback_02_frequency_of_answer_for_question_per_model_2e09fd}. The proprietary model \textit{GPT-4} had a significantly higher proportion of positive responses, which indicates a generally more favourable reception from the participants in the study.


\begin{figure}[H]
    \centering
    \includegraphics[width=\textwidth]{results/plots/assets/feedback-01-frequency-of-answer-for-question-2e09fd.png}
    \caption{The number of answers to each answer for the question \textit{"Was this a good reply?"}}
    \label{fig:feedback_01_frequency_of_answer_for_question_2e09fd}
\end{figure}


\begin{figure}[H]
    \centering
    \includegraphics[width=1\textwidth]{results/plots/assets/feedback-02-frequency-of-answer-for-question-per-model-2e09fd.png}
    \caption{The number of answers to each answer for the question \textit{"Was this a good reply?"}}
    \label{fig:feedback_02_frequency_of_answer_for_question_per_model_2e09fd}
\end{figure}


\subsection{Survey questions injected into the chat}


The software written for this thesis was, as discussed in sections \ref{sec:method_feedback_data} and \ref{sec:what_you_did_gathering_feedback_data}, and shown in \autoref{fig:multiple_choice}, designed to gather user feedback by injecting questions in the chat at certain triggers. The questions inserted after receiving the first response in chat number 2, 4 and 6 of each session can be seen in \autoref{tab:inserted_questions_2_4_6}. The questions inserted after receiving the first response in chat number 8 can be seen in \autoref{tab:inserted_questions_2_4_6}.


Unfortunately, very few of the users who used the system answered these questions. \autoref{tab:number_of_answers_received} shows how many answers the first set of questions got, and how many answers the second set of questions got.


However, for the users that did answer the first set of questions, their answers can be seen in figures \ref{fig:feedback_01_frequency_of_answer_for_question_cbfea1}, \ref{fig:feedback_01_frequency_of_answer_for_question_ead094} and \ref{fig:feedback_01_frequency_of_answer_for_question_d474ac}.


\renewcommand{\arraystretch}{1.5}
\begin{table}[H]
\centering
{\small
\begin{tabularx}{\textwidth}{@{}X X@{}}
\toprule
\textbf{Question} & \textbf{Possible Answers} \\ \midrule
The reply from the bot was useful to me & Strongly agree, Agree, Neither agree nor disagree, Disagree, Strongly disagree \\ \hdashline
How accurate did you find the answers from the bot & Very accurate, Somewhat accurate, Neither accurate nor inaccurate, Somewhat inaccurate, Very inaccurate \\ \hdashline
How would you rate the speed of the bot's reply? & Very quick, Quick, Moderate, Slow, Very slow \\
\bottomrule
\end{tabularx}
}
\vspace{2mm}
\caption{Questions asked to all study participants after receiving the first response in chat number 2, 4, 6.}
\label{tab:inserted_questions_2_4_6}
\end{table}



\renewcommand{\arraystretch}{1.5}
\begin{table}[H]
\centering
{\small
\begin{tabularx}{\textwidth}{@{}X X@{}}
\toprule
\textbf{Question} & \textbf{Answers} \\ \midrule
Overall, the information the bot provided to me has been useful & Strongly agree, Agree, Neither agree nor disagree, Disagree, Strongly disagree \\ \hdashline
Overall, how effectively has the bot been able to answer your questions? & Extremely effectively, Very effectively, Moderately effectively, Slightly effectively, Not at all effectively \\ \hdashline
Overall, the answers from the bot have been correct & Strongly agree, Agree, Neither agree nor disagree, Disagree, Strongly disagree \\ \hdashline
Overall, the answers from the bot contained all the information I needed & Strongly agree, Agree, Neither agree nor disagree, Disagree, Strongly disagree \\ \hdashline
How would you compare the ease of use of the bot with retrieving information from the canvas room yourself? & Much easier, Easier, Neither easier nor more difficult, More difficult, Much more difficult \\ \hdashline
How would you compare the time it takes to ask the bot about the canvas room with retrieving information from canvas yourself? & Much faster, Somewhat faster, About the same, Somewhat slower, Much slower \\
\bottomrule
\end{tabularx}
}
\vspace{2mm}
\caption{Questions asked to all study participants after receiving the first response in chat number 8.}
\label{tab:inserted_questions_8}
\end{table}



\renewcommand{\arraystretch}{1.5}
\begin{table}[H]
\centering
{\small
\begin{tabularx}{\textwidth}{@{}>{\raggedright\arraybackslash}X >{\centering\arraybackslash}X@{}}
\toprule
\textbf{Question} & \textbf{Number of Answers} \\ \midrule
Was this a good reply? & 78 \\ \hdashline
The reply from the bot was useful to me & 28 \\ \hdashline
How accurate did you find the answers from the bot & 26 \\ \hdashline
How would you rate the speed of the bot's reply? & 31 \\ \hdashline
Overall, the information the bot provided to me has been useful & 0 \\ \hdashline
Overall, how effectively has the bot been able to answer your questions? & 0 \\ \hdashline
Overall, the answers from the bot have been correct & 0 \\ \hdashline
Overall, the answers from the bot contained all the information I needed & 0 \\ \hdashline
How would you compare the ease of use of the bot with retrieving information from the canvas room yourself? & 0 \\ \hdashline
How would you compare the time it takes to ask the bot about the canvas room with retrieving information from canvas yourself? & 0 \\
\bottomrule
\end{tabularx}
}
\vspace{2mm}
\caption{Number of answers received to each question}
\label{tab:number_of_answers_received}
\end{table}



\begin{figure}[H]
    \centering
    \includegraphics[width=1\textwidth]{results/plots/assets/feedback-01-frequency-of-answer-for-question-cbfea1.png}
    \caption{The number of answers to each answer for the question \textit{"How would you rate the speed of the bot's reply?"}}
    \label{fig:feedback_01_frequency_of_answer_for_question_cbfea1}
\end{figure}


\begin{figure}[H]
    \centering
    \includegraphics[width=\textwidth]{results/plots/assets/feedback-01-frequency-of-answer-for-question-ead094.png}
    \caption{The number of answers to each answer for the question \textit{"How accurate did you find  the answers from the bot"}}
    \label{fig:feedback_01_frequency_of_answer_for_question_ead094}
\end{figure}


\begin{figure}[H]
    \centering
    \includegraphics[width=1\textwidth]{results/plots/assets/feedback-01-frequency-of-answer-for-question-d474ac.png}
    \caption{The number of answers to each answer for the question \textit{"The reply from the bot was useful to me"}}
    \label{fig:feedback_01_frequency_of_answer_for_question_d474ac}
\end{figure}


Since the number of answers weren't many, the study couldn’t sample between more than two different configurations. These two configurations used different \gls{LLM}s, but shared the same retrieval strategy (vector search) and embedding function (\textit{openai/text-embedding-3-large}). In figures \ref{fig:feedback_02_frequency_of_answer_for_question_per_model_cbfea1}, \ref{fig:feedback_02_frequency_of_answer_for_question_per_model_ead094} and \ref{fig:feedback_02_frequency_of_answer_for_question_per_model_d474ac} how users that were assigned the different \gls{LLM}s answered can be seen.


\begin{figure}[H]
    \centering
    \includegraphics[width=1\textwidth]{results/plots/assets/feedback-02-frequency-of-answer-for-question-per-model-cbfea1.png}
    \caption{The number of answers to each answer for the question \textit{"How would you rate the speed of the bot's reply?"}}
    \label{fig:feedback_02_frequency_of_answer_for_question_per_model_cbfea1}
\end{figure}


\begin{figure}[H]
    \centering
    \includegraphics[width=1\textwidth]{results/plots/assets/feedback-02-frequency-of-answer-for-question-per-model-ead094.png}
    \caption{The number of answers to each answer for the question \textit{"How accurate did you find  the answers from the bot"}}
    \label{fig:feedback_02_frequency_of_answer_for_question_per_model_ead094}
\end{figure}


\begin{figure}[H]
    \centering
    \includegraphics[width=\textwidth]{results/plots/assets/feedback-02-frequency-of-answer-for-question-per-model-d474ac.png}
    \caption{The number of answers to each answer for the question \textit{"The reply from the bot was useful to me"}}
    \label{fig:feedback_02_frequency_of_answer_for_question_per_model_d474ac}
\end{figure}


\section{Qualitative analysis of free-text answers}
\label{sec:qualitative_analysis_of_user_responses}


This section will present an analysis of the free text answers users have provided in the forms that have been presented in the participating courses. The complete answers to these forms can be found in \autoref{appendix:mg2040_form}, \autoref{appendix:ld1000_form} and \autoref{appendix:ld1006_form}. With these answers there is no way to correlate the answers with which chat configuration they had used, such as what \gls{LLM} had been used.


\subsection{The form submitted in MG2040}


\subsubsection{Can you describe a situation where the chatbot was particularly helpful or fell short of your expectations?}


The answers to this question show that users found the chatbot useful when asking questions about specific assignments, such as the Assembly project. One student say that it was surprisingly good at helping them with calculations for the project. Other used it to find certain topics and modules in the courseroom and thought the chatbot helped them to find it quickly.


There were also instances where the chatbot wasn’t particularly helpful. One student said it couldn’t provide the date of the exam. Others said its answers were too generic.


\subsubsection{Which type of questions would you ask the chatbot as opposed to the teacher or teaching assistants?}


The students used it to ask questions they deemed unnecessary to ask the course instructors directly. This could be questions about deadlines, number of lectures or other logistical details. Additionally, users mentioned asking the chatbot for detailed explanations with examples and for help with understanding course material. This could reportedly be late at night when course administrators aren’t available.


Some users also prefer the chatbot for questions that might be too minor or unrecognised by teaching assistants. These could be questions that does not require the expertise of a teacher or TA.


\subsubsection{Has using the chatbot changed the way you access information for your courses? If so, how?}


The chatbot has altered the course room experience for some students. While some students reported minimal change, either because of the chatbot's novelty or lack of interest, others found it helped them to access information quicker. For instance, one student reported they asked the chatbot questions instead of searching through course room documents manually. Users also mentioned that the chatbot provided good explanations of content.


\subsubsection{What has been your overall experience using the bot for course-related queries?}


Overall students who tried the chatbot reported positive, and very positive, responses. One student reported it was surprisingly effective. Others reported that it was fun and easy to use. Although some responses are seemingly very generic, the general consensus seems to be that the chatbot has been a beneficial addition to their study resources.


\subsubsection{Why have you not used the chatbot?}


The students who began the form by indicating they had not tried the chatbot got the opportunity to answer the question why they had not tried it. Most users reported that they had not had time to try it, or it was not a priority of theirs to test it. One user expressed a lack of perceived necessity, seemingly satisfied with the current mechanisms of navigating the course material.


One user reported that the reason they had not tried the chatbot was because it was monitored by the teachers of the course. This indicates that some students might feel uneasy using this technology, because they are not sure what they are allowed to ask the chatbot.


\subsection{The form submitted in LD1000}


\subsubsection{What was your experience using the chatbot for course-related questions?}


Users found the chatbot to be helpful. This was especially true for the course-related queries. Some users noted that the chatbot was very detailed in its answers, others thought it provided very high-level answers. One user noted that the bot included references to where more info could be found, which that user thought was very helpful. Some users expressed that they were surprised how well it worked. Overall the responses suggests that the chatbot was generally well-received.


\subsubsection{Were there situations where the chatbot was particularly helpful or did not meet your expectations?}


Most responses recorded a situation they thought were particularly helpful. No user used this question to report an event they thought the bot did not meet their expectations. Users found the chatbot helpful when it provided detailed information, including links to content in the course room. Some users reported that they found it useful to be able to ask follow-up questions, with more tailored questions depending on the answer to their first questions.


\subsubsection{Are there questions you prefer to ask the chatbot instead of a teacher or teaching assistant? Which ones?}


Answers here generally fell into two categories of seemingly similar size. One category reported that they appreciate the ability to ask the bot simpler questions, perhaps about course logistics, when the teacher isn’t online or available. Some users prefer the immediacy and convenience of asking a chatbot. Some still noted though that they generally prefer the human connection of asking a teacher or TA.


The second category was users who reported they don’t prefer asking any questions to a chatbot over a teacher. The reason here was generally that they preferred the "human-connection" of asking a human teacher. Where one suggested there was a trust issue with asking a chatbot over a human.


\subsubsection{Do you think using the chatbot can change how you access information in a course? How?}


The responses to this question generally indicate students think it would mostly change how they access information in courses with larger canvas rooms. That is canvas rooms with lots of pages and difficult to understand navigation. Some reported that they generally would prefer better structured course rooms, but when that’s not possible the chatbot could be a nice complement.


Some users didn’t think the chatbot would change how they accessed information. This was either due to not having used it much in this course, or because they thought they would prefer talking to a human teacher.


\subsubsection{What do you think about using AI tools in your studies? Do you use any tools today? Which ones, and for what?}


A minority of the responses reported that they used AI tools in their academic studies. Some reported that they used it for programming assignments. Others said that they use it for explanations or clarifications in texts they’ve written. Most did report that they don’t use it though. Some reasons for this were that they hadn’t found it useful, preferred to do stuff themselves because it was more creative and it could be considered cheating.


\subsection{The form submitted in LD1006}


\subsubsection{Can you describe a situation where the chatbot was particularly helpful or fell short of your expectations?}


TODO


\subsubsection{Which type of questions would you ask the chatbot as opposed to the teacher or teaching assistants?}


TODO


\subsubsection{Has using the chatbot changed the way you access information for your courses? If so, how?}


TODO


\subsubsection{What has been your overall experience using the bot for course-related queries?}


TODO


\subsubsection{Why have you not used the chatbot?}


TODO


% \sweExpl{Lite statistik av fördröjningsmätningarna visas i Tabell~\ref{tab:delayMeasurements}. Förseningen har beräknats från den tidpunkt då begäran GET tas emot fram till svaret skickas.}


% \section{Reliability Analysis}


% \sweExpl{Analys av tillförlitlighet\\
% Tillförlitlighet i metod och data}


% \section{Validity Analysis}


% \sweExpl{Analys av validitet\\
% Validitet i metod och data}


\cleardoublepage

\chapter{Discussion}
\label{ch:discussion}


This chapter will discuss the key findings from the research carried out in this thesis. It will evaluate the implications of the findings and compare it to existing literature. The discussion is structured to provide analysis of the results. It will also explore why some questions laid out in the introduction section have been left unanswered. Finally the chapter will include a brief discussion about the role of AI assistants in the broader context of AI in the educational setting.


\section{Summary of key findings}


% Briefly restate the main results and findings of your research.


\section{Explanation of key results}


% Provide interpretations of your findings and explain why certain results were obtained. Discuss the implications of your results for theory, practice, or policy.




% Lessons learned:
% Prompts aren't portable across models


\section{Comparison to existing literature}


% Compare your findings with previous studies and discuss any similarities or differences.
%




% \sweExpl{Diskussion\\
% Förbättringsförslag?}
% \generalExpl{This can be a separate chapter or a section in the previous chapter.}


diskussion här


\cleardoublepage

\chapter{Conclusions and Future work}
\label{ch:conclusionsAndFutureWork}


% \sweExpl{Slutsats och framtida arbete}
% \generalExpl{Add text to introduce the subsections of this chapter.}


This chapter will provide conclusions from the research carried out in this thesis. It will reflect on the goals laid out in section~\ref{sec:goals} and what insights have been gained. It will describe the limitations to the result and discuss some future work.


\section{Conclusions}
\label{sec:conclusions}


% state the goals and research questions


One of the goals for this thesis was to understand the technological efficacy, i.e. speed and accuracy of various different tools and techniques commonly used when developing AI assistants. In addition to that, this thesis tried to understand what users preferred, various tools and the feasibility of operating \gls{LLM}on-premises. Lastly, a goal was to understand how AI assistants impacted education. All of this was encapsulated by the research questions, \textit{Which language model and which retrieval techniques do students prefer using?} and \textit{Is it possible to deploy an AI-assistant using a completely open source toolchain?}. My hypothesis was that the closed source alternatives would be better, however it would be feasible to build a completely open source and self-hosted AI assistant too.


% what was measured closed source models v open sourced


Generally, the hypothesis was correct. The results laid out in \autoref{ch:resultsAndAnalysis}, and section~\ref{sec:survey_questions_injected_into_the_chat} specifically, showed that students preferred the model provided by OpenAI to the Mistral model. Even though that comparison isn’t very fair model-to-model, the fact that self-hosting models that are as large as those available by for profit vendors, means that building an assistant on proprietary models will yield a better assistant. However, the results show generally favourable opinions from the users who had to use the open source model.


% embedding functions and full text search


Due to the size of the experiment, there weren't enough participants to test all the tools and techniques initially intended. The software constructed for the experiments was designed for testing more tools and models, such as some open source embedding functions. However, given these weren’t used in real chats, no conclusions were drawn regarding their effectiveness or effect on student satisfaction. For the same reason, no data was collected on how traditional search techniques, such as fulltext search, affected the metrics collected in the study.


% speed, accuracy and usefulness


The perceived speed, as reported by the users, was just slightly slower for the open source model used, as can be seen in \autoref{fig:feedback_01_frequency_of_answer_for_question_cbfea1}. This was also backed up by the recorded time each response took in \autoref{fig:performance_05_daily_average_response_time_including_pending_time}. The only measurement of accuracy in this study was the perceived accuracy by the users. This is a vert brute metric, however, it did produce a clear result that users perceived \textit{GPT-4} to produce more accurate results than the Mistrals model. Again, this was fairly inline with expectations. Larger open source models might produce even more accurate results. However, a model as capable as \textit{Mistral-7B-Instruct-v0.2}, is not capable enough of building an AI assistant that users overwhelmingly think produce accurate results. The same conclusion can be drawn for the perceived usefulness of the assistants built with the different models.


% understand impact on education


Finally, the research conducted in this thesis did produce some insights on what impact chatbots and AI assistants could have on education. Students appreciated the quick and easy access to information. AI assistants could definitely improve student-teacher communications. This doesn’t mean that all student inquiries should be answered by a chatbot. Students prefer speaking to their actual teacher, however, a chatbot is a useful complement. While implementing such chatbots, the research in this thesis shows the technology selection is important. It is very important the information the bot has access to is accurate. While privacy is paramount for students, it’s still crucial teachers has insight into what the chatbot is telling students.


% re-iterate research question




% \sweExpl{Slutsatser}
% \engExpl{Describe the conclusions (reflect on the whole introduction given in Chapter 1).}




% \engExpl{Discuss the positive effects and the drawbacks.\\
% Describe the evaluation of the results of the degree project.\\
% Did you meet your goals?\\
% What insights have you gained?\\
% What suggestions can you give to others working in this area?\\
% If you had it to do again, what would you have done differently?}


% \sweExpl{Uppfyllde du dina mål?\\
% Vilka insikter har du fått?\\
% Vilka förslag kan du ge till andra som arbetar inom detta område?
% Om du skulle göra detta igen, vad skulle du ha gjort annorlunda?}


\section{Limitations}
\label{sec:limitations}


% \sweExpl{Begränsande faktorer\\Vad gjorde du som begränsade dina ansträngningar? Vilka är begränsningarna i dina resultat?}
% \engExpl{What did you find that limited your efforts? What are the limitations of your results?}


\section{Future work}
\label{sec:futureWork}






% \sweExpl{Vad du har kvar ogjort?\\Vad är nästa självklara saker som ska göras?\\Vad tips kan du ge till nästa person som kommer att följa upp på ditt arbete?}
% \engExpl{Describe valid future work that you or someone else could or should do.\\
% Consider: What you have left undone? What are the next obvious things to be done? What hints can you give to the next person who is going to follow up on your work?}




Due to the breadth of the problem, only some of the initial goals have been
met. In these section we will focus on some of the remaining issues that
should be addressed in future work. ...


\subsection{What has been left undone?}
\label{what-has-been-left-undone}


The prototype does not address the third requirment, \ie a yearly unavailability of less than 3 minutes; this remains an open problem. ...


\subsubsection{Cost analysis}


% \generalExpl{Example of a missing component}


The current prototype works, but the performance from a cost perspective makes this an impractical solution. Future work must reduce the cost of this solution; to do so, a cost analysis needs to first be done. ...


\subsubsection{Security}


% \generalExpl{Example of a missing component}


A future research effort is needed to address the security holes that results from using a self-signed certificate. Page filling text mass. Page filling text mass. ...


\subsection{Next obvious things to be done}


In particular, the author of this thesis wishes to point out xxxxxx remains as a problem to be solved. Solving this problem is the next thing that should be done. ...


\section{Reflections}
\label{sec:reflections}


% \sweExpl{Reflektioner}
% \sweExpl{Vilka är de relevanta ekonomiska, sociala, miljömässiga och etiska aspekter av ditt arbete?}
% \engExpl{What are the relevant economic, social,
%   environmental, and ethical aspects of your work?
% }




One of the most important results is the reduction in the amount of
energy required to process each packet while at the same time reducing the
time required to process each packet.


% The thesis contributes to the \gls{UN}\enspace\glspl{SDG} numbers 1 and 9 by
% xxxx.


\noindent\rule{\textwidth}{0.4mm}


% \engExpl{In the references, let Zotero or other tool fill this in for you. I suggest an extended version of the IEEE style, to include URLs, DOIs, ISBNs, etc., to make it easier for your reader to find them. This will make life easier for your opponents and examiner. \\IEEE Editorial Style Manual: \url{https://www.ieee.org/content/dam/ieee-org/ieee/web/org/conferences/style_references_manual.pdf}}
% \sweExpl{Låt Zotero eller annat verktyg fylla i det här för dig. Jag föreslår en utökad version av IEEE stil - att inkludera webbadresser, DOI, ISBN osv. - för att göra det lättare för läsaren att hitta dem. Detta kommer att göra livet lättare för dina opponenter och examinator.}


\cleardoublepage




% Print the bibliography (and make it appear in the table of contents)
\renewcommand{\bibname}{References}


\phantomsection  % make it include a hyperref - see https://tex.stackexchange.com/a/98995
\addcontentsline{toc}{chapter}{References}
\bibliography{references}


% \warningExpl{If you do not have an appendix, do not include the \textbackslash cleardoublepage command below; otherwise, the last page number in the metadata will be one too large.}

\cleardoublepage
\appendix
\renewcommand{\chaptermark}[1]{\markboth{Appendix \thechapter\relax:\thinspace\relax#1}{}}

% \generalExpl{Here is a place to add supporting material that can help others build upon your work. You can include files as attachments to the PDF file or indirectly via URLs. Alternatively, consider adding supporting material uploaded as separate files in DiVA.}
% \warningExpl{DiVA is limited to $\approx$\SI{1}{\giga\byte} for each supporting file. If you have very large amounts of supporting material, you will probably want to use one of the data repositories. For additional help about this, contact KTH Library via

\chapter{Source Code}
\label{appendix:source_code}


The source code used to conduct the research in this thesis is available in its entirety at \href{https://github.com/nattvara/kth-assistant}{github.com/nattvara/kth-assistant}.


\chapter{All answers to the form submitted in MG2040}
\label{appendix:mg2040_form}


This includes all answers to the questions in the form submitted to the students in the course \textit{MG2040 Assembly Technology 6.0 credits}. The form included logic, so only students who had used the bot got shown questions evaluating the bot. Student who answered they hadn’t used the bot, got shown a single question asking them why they hadn’t used it.


\begin{table}[H]
\centering
{\small
\begin{tabularx}{\textwidth}{@{}lX@{}}
\toprule
\textbf{Form Submission No.} & \textbf{Answer} \\ \midrule
1 & During Assembly project assignment I had some help with the calculations and surprisingly it returned commendable results. \\ \hdashline
3 & To search certain topics from the modules. \\ \hdashline
4 & While searching for the project. \\ \hdashline
5 & Assisted me explaining the notes. \\ \hdashline
8 & It couldn't answer when the exam is. \\ \hdashline
7 & Mostly when I have doubts that I can't find in the canvas. \\ \hdashline
17 & Easy knowledge questions. \\ \hdashline
19 & When I need to find quickly information in the slide. \\ \hdashline
20 & I only used it once for instructions on the line balancing. It gave me the information but it was still quite a generic answer and needed to look into the slides for full understanding. \\
\bottomrule
\end{tabularx}
}
\vspace{2mm}
\caption{Answers to the question: Can you describe a situation where the chatbot was particularly helpful or fell short of your expectations?}
\label{tab:appendix_typeform_table_question_helpful_or_fell_short}
\end{table}

\begin{table}[H]
\centering
{\small
\begin{tabularx}{\textwidth}{@{}lX@{}}
\toprule
\textbf{Form Submission No.} & \textbf{Answer} \\ \midrule
1 & The questions I think would not be necessary to ask course responsible instead I can ask the bot, like when is the deadline of the project submission? How many lectures we have in MG2040? etc. \\ \hdashline
5 & Detailed explanations with examples. \\ \hdashline
8 & Questions in the middle of the night. \\ \hdashline
7 & Mostly unrecognized questions. \\ \hdashline
17 & Easy understanding questions. \\ \hdashline
20 & Repetition. \\
\bottomrule
\end{tabularx}
}
\vspace{2mm}
\caption{Answers to the question: Which type of questions would you ask the chatbot as opposed to the teacher or teaching assistants?}
\label{tab:appendix_typeform_table_question_ask_chatbot_vs_teacher}
\end{table}

\begin{table}[H]
\centering
{\small
\begin{tabularx}{\textwidth}{@{}lX@{}}
\toprule
\textbf{Form Submission No.} & \textbf{Answer} \\ \midrule
1 & Not that much because it's still new for us. \\ \hdashline
3 & Made it easier to get topics and what prompt to use for searching a certain type of information needed. \\ \hdashline
5 & Yes, it explains better. \\ \hdashline
8 & I use the chatbot instead of searching through all documents. \\ \hdashline
7 & No, mostly if I have doubts I can access the chatbot. \\ \hdashline
19 & Yes, it’s faster. \\ \hdashline
20 & Not really as I forgot to use it a lot. \\
\bottomrule
\end{tabularx}
}
\vspace{2mm}
\caption{Answers to the question: Has using the chatbot changed the way you access information for your courses? If so, how?}
\label{tab:appendix_typeform_table_question_changed_access}
\end{table}

\begin{table}[H]
\centering
{\small
\begin{tabularx}{\textwidth}{@{}lX@{}}
\toprule
\textbf{Form Submission No.} & \textbf{Answer} \\ \midrule
1 & I have had pleasant experience. I thought it would be stupid but it was not! \\ \hdashline
3 & It was helpful. \\ \hdashline
4 & Its fun. \\ \hdashline
5 & Learning is easy. \\ \hdashline
7 & Its nice. \\ \hdashline
8 & Very effective. \\ \hdashline
20 & Good. \\
\bottomrule
\end{tabularx}
}
\vspace{2mm}
\caption{Answers to the question: What has been your overall experience using the bot for course-related queries?}
\label{tab:appendix_typeform_table_question_overall_experience}
\end{table}

\begin{table}[H]
\centering
{\small
\begin{tabularx}{\textwidth}{@{}lX@{}}
\toprule
\textbf{Form Submission No.} & \textbf{Answer} \\ \midrule
1 & Yes \\ \hdashline
2 & Yes \\ \hdashline
3 & Yes \\ \hdashline
4 & Yes \\ \hdashline
5 & Yes \\ \hdashline
6 & Yes \\ \hdashline
7 & Yes \\ \hdashline
8 & Yes \\ \hdashline
9 & No \\ \hdashline
10 & No \\ \hdashline
11 & No \\ \hdashline
12 & No \\ \hdashline
13 & No \\ \hdashline
14 & No \\ \hdashline
15 & No \\ \hdashline
16 & No \\ \hdashline
17 & Yes \\ \hdashline
18 & No \\ \hdashline
19 & Yes \\ \hdashline
20 & Yes \\
\bottomrule
\end{tabularx}
}
\vspace{2mm}
\caption{Answers to the question: Have you tried the chatbot?}
\label{tab:appendix_typeform_table_question_tried_chatbot}
\end{table}

\begin{table}[H]
\centering
{\small
\begin{tabularx}{\textwidth}{@{}lX@{}}
\toprule
\textbf{Form Submission No.} & \textbf{Answer} \\ \midrule
9 & It is monitored by the teacher. \\ \hdashline
10 & It was not just a priority of mine till date, but will try using it while navigating in the examination and see how it goes. \\ \hdashline
12 & No need for now. \\ \hdashline
13 & Because I don't take Time to try maybe it's better if you show an example of using during the course. \\ \hdashline
14 & I don’t take the time. \\ \hdashline
16 & Had no reason so far might use it for exam preparation. \\
\bottomrule
\end{tabularx}
}
\vspace{2mm}
\caption{Answers to the question: Why have you not used the chatbot?}
\label{tab:appendix_typeform_table_question_not_used_chatbot}
\end{table}

\begin{table}[H]
\centering
{\small
\begin{tabularx}{\textwidth}{@{}lX@{}}
\toprule
\textbf{Form Submission No.} & \textbf{Answer} \\ \midrule
1 & Strongly agree \\ \hdashline
2 & Agree \\ \hdashline
3 & Agree \\ \hdashline
4 & Neither agree nor disagree \\ \hdashline
5 & Agree \\ \hdashline
6 & Agree \\ \hdashline
7 & Strongly agree \\ \hdashline
8 & Agree \\ \hdashline
17 & Neither agree nor disagree \\ \hdashline
19 & Agree \\ \hdashline
20 & Strongly agree \\
\bottomrule
\end{tabularx}
}
\vspace{2mm}
\caption{Answers to the question: Overall, the information the bot provided to me has been useful}
\label{tab:appendix_typeform_table_question_information_useful}
\end{table}

\begin{table}[H]
\centering
{\small
\begin{tabularx}{\textwidth}{@{}lX@{}}
\toprule
\textbf{Form Submission No.} & \textbf{Answer} \\ \midrule
1 & Very effectively \\ \hdashline
2 & Very effectively \\ \hdashline
3 & Very effectively \\ \hdashline
4 & Moderately effectively \\ \hdashline
5 & Moderately effectively \\ \hdashline
6 & Very effectively \\ \hdashline
7 & Moderately effectively \\ \hdashline
8 & Very effectively \\ \hdashline
17 & Moderately effectively \\ \hdashline
19 & Moderately effectively \\ \hdashline
20 & Very effectively \\
\bottomrule
\end{tabularx}
}
\vspace{2mm}
\caption{Answers to the question: Overall, how effectively has the bot been able to answer your questions?}
\label{tab:appendix_typeform_table_question_effectively_answer}
\end{table}

\begin{table}[H]
\centering
{\small
\begin{tabularx}{\textwidth}{@{}lX@{}}
\toprule
\textbf{Form Submission No.} & \textbf{Answer} \\ \midrule
1 & Strongly agree \\ \hdashline
2 & Agree \\ \hdashline
3 & Agree \\ \hdashline
4 & Neither agree nor disagree \\ \hdashline
5 & Agree \\ \hdashline
6 & Neither agree nor disagree \\ \hdashline
7 & Agree \\ \hdashline
8 & Agree \\ \hdashline
17 & Neither agree nor disagree \\ \hdashline
19 & Agree \\ \hdashline
20 & Agree \\
\bottomrule
\end{tabularx}
}
\vspace{2mm}
\caption{Answers to the question: Overall, the answers from the bot have been correct}
\label{tab:appendix_typeform_table_question_answers_correct}
\end{table}

\begin{table}[H]
\centering
{\small
\begin{tabularx}{\textwidth}{@{}lX@{}}
\toprule
\textbf{Form Submission No.} & \textbf{Answer} \\ \midrule
1 & Neither agree nor disagree \\ \hdashline
2 & Agree \\ \hdashline
3 & Agree \\ \hdashline
4 & Neither agree nor disagree \\ \hdashline
5 & Agree \\ \hdashline
6 & Neither agree nor disagree \\ \hdashline
7 & Neither agree nor disagree \\ \hdashline
8 & Neither agree nor disagree \\ \hdashline
17 & Neither agree nor disagree \\ \hdashline
19 & Disagree \\ \hdashline
20 & Neither agree nor disagree \\
\bottomrule
\end{tabularx}
}
\vspace{2mm}
\caption{Answers to the question: Overall, the answers from the bot contained all the information I needed}
\label{tab:appendix_typeform_table_question_info_needed}
\end{table}

\begin{table}[H]
\centering
{\small
\begin{tabularx}{\textwidth}{@{}lX@{}}
\toprule
\textbf{Form Submission No.} & \textbf{Answer} \\ \midrule
1 & Much easier \\ \hdashline
2 & Much easier \\ \hdashline
3 & Easier \\ \hdashline
4 & Easier \\ \hdashline
5 & Easier \\ \hdashline
6 & Neither easier nor more difficult \\ \hdashline
7 & Neither easier nor more difficult \\ \hdashline
8 & Much easier \\ \hdashline
17 & Easier \\ \hdashline
19 & Easier \\ \hdashline
20 & Easier \\
\bottomrule
\end{tabularx}
}
\vspace{2mm}
\caption{Answers to the question: How would you compare the ease of use of the bot with retrieving information from the canvas room yourself?}
\label{tab:appendix_typeform_table_question_ease_of_use}
\end{table}

\begin{table}[H]
\centering
{\small
\begin{tabularx}{\textwidth}{@{}lX@{}}
\toprule
\textbf{Form Submission No.} & \textbf{Answer} \\ \midrule
1 & Somewhat faster \\ \hdashline
2 & About the same \\ \hdashline
3 & Much faster \\ \hdashline
4 & Somewhat faster \\ \hdashline
5 & About the same \\ \hdashline
6 & Much faster \\ \hdashline
7 & Somewhat faster \\ \hdashline
8 & Much faster \\ \hdashline
17 & Much faster \\ \hdashline
19 & Much faster \\ \hdashline
20 & Somewhat faster \\
\bottomrule
\end{tabularx}
}
\vspace{2mm}
\caption{Answers to the question: How would you compare the time it takes to ask the bot about the canvas room with retrieving information from canvas yourself?}
\label{tab:appendix_typeform_table_question_time_comparison}
\end{table}



\chapter{All answers to the form submitted in LD1000}
\label{appendix:ld1000_form}


This includes all answers to the questions in the form submitted to the students in the course \textit{LD1000 Lär dig lära online 2,0 hp}.


\begin{table}[h]
\centering
{\small
\begin{tabularx}{\textwidth}{@{}lX@{}}
\toprule
\textbf{ID} & \textbf{Answer} \\ \midrule
1 & Den svarar bra på frågor som handlar om kursen och den hjälper en att hitta i canvas om det är nån specifik sak man söker. Det gör den bra tycker jag. \\ \hdashline
2 & Upplevelsen var bra, den svarade korrekt på alla kurs relaterade frågor som jag ställde. \\ \hdashline
3 & Min upplevelse av att använda chattboten för kursrelaterade frågor var mycket positiv och över min förväntan. Den var skalad och konkret. Gissar att just dens informationsutbud här matchade men utmärkte sig i saklighet och korrekt svar när man kontrollerade mot fakta. \\ \hdashline
4 & Den svarade lite långsamt, men verkade ge korrekt svar. \\ \hdashline
5 & Bra fast alltför grunt. \\ \hdashline
6 & Fick känslan av att kapaciteten finns att besvara frågor om kursen på ett relevant sätt, men något ytligt. \\ \hdashline
7 & Min upplevelse var positiv, det känns kul när man skriver en fråga och man får svar 😊 \\ \hdashline
8 & Jag tycker att chattbotten kan svara på frågor om hur många quiz man måste göra osv. \\ \hdashline
9 & Den svarar informativt och detaljerat och avslutar med information om var jag kan läsa mer. Användbart eftersom man får svar så snabbt. \\
\bottomrule
\end{tabularx}
}
\vspace{2mm}
\caption{Answers to the question: Vad var din upplevelse av att använda chattboten för kursrelaterade frågor?}
\label{tab:appendix_typeform_table_question_experience}
\end{table}


\begin{table}[h]
\centering
{\small
\begin{tabularx}{\textwidth}{@{}lX@{}}
\toprule
\textbf{ID} & \textbf{Answer} \\ \midrule
1 & Jag testade att fråga hur många quiz som ingick i kursen och den beskrev utförligt hur många den bestod av, hur många rätt man måste få för att bli godkänd samt hur många försök man har på sig. Jag tyckte den var extra hjälpsam genom att den gav en länk till "Examinerande moment" så att man själv kan läsa på mer om de olika momenten i kursen! \\ \hdashline
2 & Jag prövade att ställa "större" frågor som hur bestämt bestämma struktureringen av inlärningen av kursen. Där jag tyckte att svaret var väldigt utförligt och kan klassas som extra hjälpsamt. Svaret var en 6 stegs plan om hur hanteringen och bearbetningen av alla moment och moduler skulle gå till. \\ \hdashline
3 & Chatboten var särskilt hjälpsam och den levde upp till dina förväntningar mer än väntat då den var väldigt konkret i sitt svar och snabb på att hitta det exakta svaret jag sökte. \\ \hdashline
4 & Det som kändes extra hjälpsamt var att man fick en länk till mer detaljerad information (gissningsvis källan den inhämtade information från). \\ \hdashline
5 & Nej. \\ \hdashline
6 & Testade att ställa ett par frågor rent allmänt om examinerande moment och upplevde svaren var tillfredsställande. \\ \hdashline
7 & Jag ställde en bredd fråga bland annat och då svarade den med ett par påstående som jag fick välja bland för att få bättre anpassat svar vilket jag uppfattade som hjälpsamt. \\ \hdashline
8 & Den svarade på vad man behövde göra för att bli godkänd i kursen. \\ \hdashline
9 & Den är bra på att ta mig vidare när jag kört fast. Då kan jag fråga boten vad jag ska skriva nu, vilket den förstår inte svarar på men den förtydligar ämnet och säger "tänk till exempel såhär", och så är jag igång igen. \\
\bottomrule
\end{tabularx}
}
\vspace{2mm}
\caption{Answers to the question: Uppstod det situationer där chatboten var särskilt hjälpsam eller där den inte levde upp till dina förväntningar?}
\label{tab:appendix_typeform_table_question_helpful_or_fell_short}
\end{table}

\begin{table}[h]
\centering
{\small
\begin{tabularx}{\textwidth}{@{}lX@{}}
\toprule
\textbf{ID} & \textbf{Answer} \\ \midrule
1 & Det finns egentligen inte någon speciell fråga jag hellre ställer till boten än en lärare. För mig handlar det kanske mer om att jag hellre skriver till boten för att få ett svar på direkten och slipper vänta! \\ \hdashline
2 & Till viss del, särskilt om det är relativt simpla frågor som jag bara vill ha ett snabbt svar på. \\ \hdashline
3 & Jag uppskattar generellt mer att ställa frågor till lärare men tycker att en chattbot är som fungerar som denna skulle vara användbar för att studera på tider när läraren ej är aktiv. Då gällande alla frågor. \\ \hdashline
4 & Man vill ju inte störa kursassistenter eller lärare i onödan för att fråga frågor om kursupplägg, deadlines, etc., så då kändes chattboten som rätt ställe att söka på först. \\ \hdashline
5 & Frågan, varför då? \\ \hdashline
6 & Nej! I alla situationer föredrar jag levande människor. Många onlinekurser har så pass få fysiska onlineträffar, så dessa är oerhört värdefulla. Intressant med mänskliga möten i sig, att få bekantskap med ansiktena bakom en kurs, och att vid behov kunna ställa frågor direkt till dem. \\ \hdashline
7 & Nej, det tror jag inte, frågor kan jag ställa till både lärare, lärarassistent och chatboten? Det som är viktigt är att man kan lita på att det blir rätt svar och att inte chatboten svarar på vad som gäller för andra kurser tex om man frågar vilka examinerande moment som finns. \\ \hdashline
8 & Nej, det tror jag inte. Kanske så kallade "dumma" frågor som borde vara självklara kanske man hellre frågar chatbotten. \\ \hdashline
9 & Är den här diskussionen obligatorisk? \\
\bottomrule
\end{tabularx}
}
\vspace{2mm}
\caption{Answers to the question: Finns det frågor du hellre ställer till chatboten än till lärare eller lärarassistent? Vilka?}
\label{tab:appendix_typeform_table_question_chatbot_vs_teacher}
\end{table}

\begin{table}[h]
\centering
{\small
\begin{tabularx}{\textwidth}{@{}lX@{}}
\toprule
\textbf{ID} & \textbf{Answer} \\ \midrule
1 & Det tror jag. Jag hade använt mig av den för att snabbt få fram den informationen jag vill ha. Vilket kan vara störande i kursen som är stora och har många moduler att navigera igenom. Då hade jag definitivt använt mig av boten. \\ \hdashline
2 & Den kan nog hjälpa till med effektiviteten en del med ställandet av simpla frågor och kolla upp vad som examineras på och så vidare. \\ \hdashline
3 & Om jag visste att det var tillåtet skulle jag använda chattboten 100\% till att ta in information då den verkar kunna ta ut det viktiga och formulera text som ger bilden av vad materialet gör. \\ \hdashline
4 & För min del föredrar jag nog att ha all relevant information koncist beskriven och samlad på en sida på Canvas, men om informationen är väldigt spridd kan jag tänka mig att chattboten skulle kunna vara till hjälp. \\ \hdashline
5 & Ja. Om den kan bli en gruppkompis som aldrig är sur och har oändligt tålamod. \\ \hdashline
6 & En genomarbetad studiehandledning, vilket ingår i de flesta onlinekurser är fullt tillräckligt, då behövs ingen chatbot. Tilltalas personligen av att ha all information samlad om en kurs i en studiehandledning. \\ \hdashline
7 & Nej det tror jag inte, men det är egentligen svårt att svara få jag använt det för lite för att ha någon egentlig uppfattning. \\ \hdashline
8 & Nej, inte väsentligt. Men det kan göra att man snabbare får svar på sina frågor än om man frågar en lärare. \\ \hdashline
9 & Om det är något jag inte förstår kan jag fråga och få ett sammanfattat svar. Då kan jag välja om jag vill läsa mer utförligt eller inte beroende på ämnets relevans. \\
\bottomrule
\end{tabularx}
}
\vspace{2mm}
\caption{Answers to the question: Tror du användningen av Chatboten kan ändra hur du tar till dig information i en kurs? Hur då?}
\label{tab:appendix_typeform_table_question_changed_access}
\end{table}

\begin{table}[h]
\centering
{\small
\begin{tabularx}{\textwidth}{@{}lX@{}}
\toprule
\textbf{ID} & \textbf{Answer} \\ \midrule
1 & Jag tycker att AI-verktyg är en bra grej och använder det i mina studier. Främst använder jag det för att få saker förklarade för mig på andra sätt som kanske är lättare för mig att förstå. Vissa lärare kanske undervisar på ett sätt som inte passar mig. Då är det smidigt att kunna lägga in deras föreläsningar i ett sådant verktyg och be verktyget att formulera om det eller förtydliga otydliga delar. \\ \hdashline
2 & Jag kan se det positiva i att utnyttja AI-verktyg i mina studier men använder det för tillfället inte i mina studier idag. \\ \hdashline
3 & Jag använder endast AI ifall jag vill hitta något jag inte direkt gör i min litteratur men använder den mera privat för att ställa frågor om diverse saker. \\ \hdashline
4 & Jag använder inga AI-verktyg i mina studier och känner nog inte riktigt att det är där de gör mest nytta heller (och då arbetar jag ändå med AI, och har en generellt positiv inställning till det). \\ \hdashline
5 & Chat gtp för programmering. \\ \hdashline
6 & Jag använder inga AI-verktyg i mina studier. Tycker om att vara kreativ själv. \\ \hdashline
7 & Jag använder inte några verktyg idag men jag tycker att det är intressant och testar gärna olika. \\ \hdashline
8 & Jag tycker det är bra, jag använder det ganska ofta för att få en förklaring på något som jag tyckte behövdes förtydligas. Jag använder det också när jag skriver rapporter som ett avancerat rättstavningsprogram som kan hjälpa så att det blir en bra meningsuppbyggnad och låter bra. \\ \hdashline
9 & Jag har inte haft för vana att använda AI-verktyg. Jag vet att de används för att formulera text ibland, vilket jag tycker är lite fusk, men samtidigt en ganska bra idé eftersom så många har svårt för just det. \\
\bottomrule
\end{tabularx}
}
\vspace{2mm}
\caption{Answers to the question: Hur tänker du kring användandet av AI-verktyg i dina studier? Använder du några verktyg idag? Vilka, till vad?}
\label{tab:appendix_typeform_table_question_AI_tools}
\end{table}



\chapter{All answers to the form submitted in LD1006}
\label{appendix:ld1006_form}


This includes all answers to the questions in the form submitted to the students in the course \textit{LD1006 Kognitiv psykologi för lärare: Matematikundervisning 3,0 hp}.


\begin{table}[h]
\centering
{\small
\begin{tabularx}{\textwidth}{@{}lX@{}}
\toprule
\textbf{ID} & \textbf{Answer} \\ \midrule
1 & För enklare frågor fungerar botten bra. \\ \hdashline
2 & Jag är helt ny användare så det är svårt att säga något nu :) \\ \hdashline
3 & Använde dem bara för att svara på AI verktyg frågorna. \\ \hdashline
4 & Jag är relativt ny i användandet av AI, därför prövade jag lite olika inlägg i AI-Copilot. \\ \hdashline
5 & Jag frågade om zoom-mötena spelades in och lades upp på sidan - jag fick svaret ja. Så där ger den mig fel information. \\ \hdashline
6 & Det fungerade bra tycker jag. Ställde bara enkla frågor som jag redan hade svaret på. \\ \hdashline
7 & Den svarar väldigt svepande och allmänt. \\ \hdashline
8 & Frågade och fick ett korrekt svar. \\ \hdashline
9 & En relativt positiv upplevelse. \\ \hdashline
10 & Den var oväntat positiv! För mig var det inte självklart med kopplingen mellan hämtningskapacitet och lagringskapacitet vilket jag ställde en fråga om. Jag fick ett utmärkt svar och förstod bättre än innan. \\ \hdashline
11 & Aldrig använt AI chat tidigare, men tycker det fungerade riktigt bra. Jag frågade bland annat om hur man undervisar lågpresterande elever och fick då både sammanhängande text och även en lista med råd på hur man kan arbeta med dessa elever. Jag frågade också om "den tysta läraren" och fick då svar på min fråga. \\ \hdashline
12 & Bra. Jag fick svar på min fråga. \\ \hdashline
13 & Bra upplevelse. Den hjälpte mig att besvara på frågor kring sena inlämningar. \\
\bottomrule
\end{tabularx}
}
\vspace{2mm}
\caption{Answers to the question: Vad var din upplevelse av att använda chattboten för kursrelaterade frågor?}
\label{tab:appendix_typeform_table_question_experience}
\end{table}

\begin{table}[h]
\centering
{\small
\begin{tabularx}{\textwidth}{@{}lX@{}}
\toprule
\textbf{ID} & \textbf{Answer} \\ \midrule
1 & För frågor som är relaterade till självaste kursmaterialet, dvs teoretiska frågor, är den näst intill inkapabel. \\ \hdashline
2 & Det är bra att få bra verktyg, men det är bra tiden som visar om det är bra/effektivt eller inte. \\ \hdashline
3 & Ser ut som det kan bli hjälpsamt när det gäller definitioner. Men för djupare frågor var kritisk tänkande krävs, chatbot svarar igen baserat på kursens definitioner. \\ \hdashline
4 & Jag skrev några olika inlägg med fråga om lärstilar kopplat till forskning och önskade svar skrivet av en 15-åring och senare ett svar skrivet av en forskare. Det var intressant att se hur boten anpassade texterna utifrån min önskan. Texten var välformulerade och anpassade utifrån vem som enligt min önskan skulle vara författaren. Dock fick jag endast en källhänvisning till forskning. \\ \hdashline
5 & Jag frågade om zoom-mötena spelades in och lades upp på sidan - jag fick svaret ja. Så där ger den mig fel information. \\ \hdashline
6 & Jag kunde söka efter ett specifikt avsnitt i boken för att få veta vilket kapitel jag skulle leta i. Det hade jag inte förväntat mig. \\ \hdashline
7 & Jag har inga direkta förväntningar. \\ \hdashline
8 & Kan inte säga att jag tog det vidare. Fick inte svar på djupare frågor i kursen. \\ \hdashline
9 & Det enda var väl att den någon gång svarade på engelska fast frågan ställdes på svenska. Sen svarade den antingen ganska kort och inte så uttömmande som andra chatbotar på vissa frågor och andra frågor svarade den inte på hela frågan. Exempelvis ställde jag fråga om den kunde nämna 3 inre samt 3 yttre motivationer, då nämnde den enbart 3 inre. \\ \hdashline
10 & Nej det var jag inte, den var precis så hjälpsam som jag behövde. \\ \hdashline
11 & Chatboten klarade att svara på alla de frågor jag ställde på ett tillfredsställande sätt. \\ \hdashline
12 & Vet ej, har bara testat 1 gång. \\ \hdashline
13 & den var mycket hjälpsam. \\
\bottomrule
\end{tabularx}
}
\vspace{2mm}
\caption{Answers to the question: Uppstod det situationer där chatboten var särskilt hjälpsam eller där den inte levde upp till dina förväntningar?}
\label{tab:appendix_typeform_table_question_helpful_or_fell_short}
\end{table}

\begin{table}[h]
\centering
{\small
\begin{tabularx}{\textwidth}{@{}lX@{}}
\toprule
\textbf{ID} & \textbf{Answer} \\ \midrule
1 & Teoretiska frågor från kursmaterialet. \\ \hdashline
2 & Svårt att svara på den frågan nu. \\ \hdashline
3 & Nej, hittills inte alls. \\ \hdashline
4 & Svårt att säga eftersom chattboten är helt ny för mig. \\ \hdashline
5 & Nej - är väl allmänt lite skeptisk till olika chatbotar (även på andra diverse sidor där man kan chatta med en chatbot). Jag mailar eller ringer hellre till en person för att få svar. \\ \hdashline
6 & Detta kan jag inte svara på i nuläget. \\ \hdashline
7 & Nej. \\ \hdashline
8 & Nej. \\ \hdashline
9 & Inga specifika frågor men jag kanske skulle välja att ställa frågor till chatboten i första hand då jag tänker att jag får svar fortare av den då den inte blir lika överbelastad som lärare kan vara. \\ \hdashline
10 & Kanske om det var frågor jag kände mig dum att fråga. \\ \hdashline
11 & Åh, svårt, men det finns det säkert. Frågor som man kanske tycker att man ska ha koll på ställer man säkert hellre till en chatbot än till en lärare. \\ \hdashline
12 & Frågor om kursen, administrativa frågor är bra att ställa till chatboten. Man får ju svar direkt vilket underlättar mycket. \\ \hdashline
13 & var inte helt säker på till vem man skulle vända sig till för frågor så då var det bra att den här fanns som kunde svara istället. \\
\bottomrule
\end{tabularx}
}
\vspace{2mm}
\caption{Answers to the question: Finns det frågor du hellre ställer till chatboten än till lärare eller lärarassistent? Vilka?}
\label{tab:appendix_typeform_table_question_chatbot_vs_teacher}
\end{table}

\begin{table}[h]
\centering
{\small
\begin{tabularx}{\textwidth}{@{}lX@{}}
\toprule
\textbf{ID} & \textbf{Answer} \\ \midrule
1 & Nej. \\ \hdashline
2 & Det är svårt att säga eftersom chattboten är helt ny för mig. \\ \hdashline
3 & Jo, kanske om, i framtiden, jag kommer inte att hinna läsa hela kursens material inom kort tid. \\ \hdashline
4 & Det är svårt att säga eftersom chattboten är helt ny för mig. \\ \hdashline
5 & Nej - inte som den fungerar nu. \\ \hdashline
6 & Kanske framöver när jag lärt mig mer om det. Just nu litar jag inte fullt ut på att det fungerar. \\ \hdashline
7 & Nej. \\ \hdashline
8 & Nej egentligen inte. Jag har personligen ringa intresse för detta förutom hur mina elever kan tänkas använda det. Så i min profession är jag intresserad men för mitt eget lärande - nej. \\ \hdashline
9 & Beror på vilken slags kurs det är och hur gammal chatboten är. Den kan vara användbar om man ställer frågor kring ex marknadsföring men jag ställde en enkel mattefråga till den nyss och den svarade tyvärr fel, liksom ChatGPT 3,5 gjorde. \\ \hdashline
10 & Det vet jag inte i dagsläget. \\ \hdashline
11 & Ja, det kan ju eventuellt förtydliga begrepp som man tycker är svåra att förstå, samt evt underlätta informationssökning. \\ \hdashline
12 & Vet inte, har inte så mycket erfarenhet. \\ \hdashline
13 & Den kan förtydliga saker och ting för mig vid arbetet med case och quiz. \\
\bottomrule
\end{tabularx}
}
\vspace{2mm}
\caption{Answers to the question: Tror du användningen av Chatboten kan ändra hur du tar till dig information i en kurs? Hur då?}
\label{tab:appendix_typeform_table_question_changed_access}
\end{table}

\begin{table}[h]
\centering
{\tiny
\begin{tabularx}{\textwidth}{@{}lX@{}}
\toprule
\textbf{ID} & \textbf{Answer} \\ \midrule
1 & Jag ser inte speciellt positivt på AI-verktygen just när det kommer till matematik, för i slutändan behöver man "plöja" problem som man själv behöver förstå, applicera teori och tänka kring. Men det är möjligt att det är användbart i andra ämnen och områden. Därför ser jag personligen ingen användning av det idag. \\ \hdashline
2 & Jag gick på en kurs för en månad sedan där jag fick veta om AI-verktyg, det låter spännande och skrämmande samtidigt :) \\ \hdashline
3 & Jag är inte ett stort fan av chatbots. Jag märkte en gång att några av mina elever använde det för att lösa ordproblem; det gjorde inte ett bra jobb så det fortsatte inte med det. \\ \hdashline
4 & Min fundering är hur man som lärare ska kunna se eller kontroller om en text är författad av en student eller av en AI? Jag testade att skriva in samma fråga två gånger och fick olika svar. Som jag ser det är det inte en fråga om användandet av AI eller inte. Det kommer att användas även i studiesammanhang vare sig vi vill eller inte. Jag tänker att via AI-verktyget kan vara en hjälp för att få en snabb koll kring en fråga. Den kan också hjälpa till att förstärka det man läser och lär sig, men den kan inte, i alla fall för mig, ersätta den lärprocess som sker när man tar till sig kunskap jag läser (helst analog) i kurslitteratur eller studier samt det som sker i kunskapsutbyte med andra personer. Jag är ännu för okunnig om AI-användning för att tänka kring hur det kan användas i undervisning inom matematik. \\ \hdashline
5 & Jag undervisar i en kurs på gymnasiet för våra åk 3 elever som heter Ledarskap och organisation. Där är AI inte med i det centrala innehållet, men något som jag tycker bör lyftas i och med att det tar större och större del i vår vardag men även arbetsliv. De får diskutera och intervjua (i samband man annat område med) hur ett företag/organisation använder sig av AI - men även nu på slutseminariet får de diskutera utifrån vad de själva tycker och om de t.ex. vill byta ut chefen mot en AI (utifrån en artikel de läst). Även viktigt att förklara för dem hur t.ex. chat GPT fungerar (för alla har använt den till skolarbete på olika sätt), att lära dem hur de kan använda den på ett smart sätt som en "studybudy", men att vara lite mer vaksam när det gäller information m.m. Men en jättebra källa om man söker inspiration till något, skapa quizar m.m. \\ \hdashline
6 & Jag provade att tillverka läsförståelsefrågor åt eleverna med hjälp av ChatGPT. Det såg väldigt bra ut till en början men när man läste mer noggrant fanns det ingen som helst substans i texten. Förmodligen för att jag ställt frågan på helt fel sätt. Jag är inte negativ till AI-verktyg men känner att jag måste lära mig mer innan jag kan/vågar använda det i undervisningen. \\ \hdashline
7 & Ja, jag har använt AI för att skapa flera varianter av enklare matteprov. Men man måste alltid modifiera. Jag ställer ibland frågor. Eleverna använder det för att svara på frågor ibland, men man kan se att det inte är deras egna ord ganska ofta. Det kommer nog att bli stort i framtiden. Men just nu är vi inte riktigt där, är känslan. \\ \hdashline
8 & Jag vet att många elever, särskilt på gymnasiet, använder det. De använder det mest för att få jobbet gjort och lämnar in. Här är det viktigt att vi lärare lär oss om hur det fungerar. Här blir det liksom en kollaps i lärandet som jag ser det. Vi pratas inte ens om prestation kontra lärande utan en AI som levererar åt dig. Problematiskt skulle jag säga. I matematiken anser jag att det är ett mindre problem men i andra ämnen med tex skriftliga inlämningar så behöver man tänka över sin bedömning som lärare. I matematiken skulle det kunna vara ett stöd för elever i det att de får exempel på lösningar till matematiska problem som de sedan kan överföra till andra uppgifter. Eleverna behöver dock få tillgång till studiestrategier som ger dem faktisk nytta av den information AI kan ge dem. \\ \hdashline
9 & Med mina elever använder jag inte AI-verktyg och då jag mest håller på med matte så tror jag inte att jag kommer göra det heller inom en snar framtid. Däremot brukar jag själv använda mig av ChatGPT 3,5 (tycker USD \$20/mån är lite väl dyrt för 4,0, men ska nog inom en snar framtid börja använda den istället). Håller på att starta upp ett företag och då tycker jag ChatGPT är användbar vid frågor kring hur man kan attrahera kunder, marknadsföring, mm. \\ \hdashline
10 & Idag använder inte mina elever några sådana verktyg och det beror nog på att jag är ovan vid och okunnig om det själv. Men jag ska lära mig mer! Först då kan jag svara bättre på den frågan. \\ \hdashline
11 & Jag undervisar just nu i en årskurs 1 så AI används inte av mina elever. Inte heller av mig i min yrkesroll som lärare. Jag skulle dock absolut kunna tänka mig att använda mig av AI-verktyg om jag fick utbildning i hur det ska användas och vilka fallgropar man bör undvika. \\ \hdashline
12 & Jag använder i nuläge väldigt sällan AI-verktyg, men jag har testat att göra en planering till en schemabrytande dag med fokus på motivation. Det gick ganska bra. Jag fick bra tips på aktiviteter, och det gick snabbare än att googla själv. Eleverna får inte använda AI-verktyg just nu, men vi har haft diskussion om det. Jag är öppen för att testa nya AI-verktyg. \\ \hdashline
13 & Jag använder det ibland till hjälp med att formulera uppgifter. Jag vet att också elever använder det bland annat för att söka svar på uppgifter som de sedan kan öva på. \\ \hdashline
14 & Jag missade att använda denna chattbott men brukar uppskatta när det finns. Finns det frågor du hellre ställer till chatboten än till lärare eller lärarassistent? Vilka? Är det frågor men relativ enkla svar så skulle jag gärna ställa dem till en chatbott. Känslan av att fråga något som redan sagts av lärare minskar om man har en chatt att fråga. Jag brukar uppskatta att ta reda på saker själv och genom en chattbott så kan frågorna /svaren utveckla sig jämfört med bara färdiga fråga och svar. I detta läget tror jag inte att man kommer att ta åt sig information på annat sätt. Vi använder i nuläget inget AI-verktyg. Bland mina elever så tror jag att det är viktigt att de förstår hur AI -verktyg fungerar och har ett källkritiskt tänkande. Vidare så tror jag att arbetsmarknaden kommer att ändras och att det tex blir viktigt att kunna /förstå att skriva tex promptar. \\
\bottomrule
\end{tabularx}
}
\vspace{2mm}
\caption{Answers to the question: Hur tänker du kring användandet av AI-verktyg både av dig själv och dina elever? Använder du och/eller dina elever några verktyg idag? Vilka, till vad?}
\label{tab:appendix_typeform_table_question_AI_tools}
\end{table}



\cleardoublepage

% Information for authors
%\include{README_author}
%\subfile{README_author}

\cleardoublepage
% information about the template for everyone
%\input{README_notes/README_notes}

\begin{comment}
% information for examiners
\ifxeorlua
\cleardoublepage
\input{README_notes/README_examiner_notes}
\fi
\end{comment}

\begin{comment}
% Information for administrators
\ifxeorlua
\cleardoublepage
\input{README_notes/README_for_administrators.tex}
\fi
\end{comment}

\begin{comment}
% Information for Course Coordinators
\ifxeorlua
\cleardoublepage
\input{README_notes/README_for_course_coordinators}
\fi
\end{comment}

%% The following label is necessary for computing the last page number of the body of the report to include in the "For DIVA" information
\label{pg:lastPageofMainmatter}

\cleardoublepage
\clearpage\thispagestyle{empty}\mbox{} % empty page with backcover on the other side
\kthbackcover
\fancyhead{}  % Do not use header on this extra page or pages
\section*{€€€€ For DIVA €€€€}
\lstset{numbers=none} %% remove any list line numbering
\divainfo{pg:lastPageofPreface}{pg:lastPageofMainmatter}

% If there is an acronyms.tex file,
% add it to the end of the For DIVA information
% so that it can be used with the abstracts
% Note that the option "nolol" stops it from being listed in the List of Listings

% The following bit of ugliness is because of the problems PDFLaTeX has handling a non-breaking hyphen
% unless it is converted to UTF-8 encoding.
% If you do not use such characters in your acronyms, this could be simplified.
\ifxeorlua
\IfFileExists{lib/acronyms.tex}{
\section*{acronyms.tex}
\lstinputlisting[language={[LaTeX]TeX}, nolol, basicstyle=\ttfamily\color{black},
commentstyle=\color{black}, backgroundcolor=\color{white}]{lib/acronyms.tex}
}
{}
\else
\IfFileExists{lib/acronyms-for-pdflatex.tex}{
\section*{acronyms.tex}
\lstinputlisting[language={[LaTeX]TeX}, nolol, basicstyle=\ttfamily\color{black},
commentstyle=\color{black}, backgroundcolor=\color{white}]{lib/acronyms-for-pdflatex.tex}
}
{}
\fi


\end{document}
